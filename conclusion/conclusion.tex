\documentclass[../master_thesis.tex]{subfiles}
\begin{document}
We developed lots of theoretical tools in the first chapter. We introduced 
differential forms, singular homology and Hilbert complexes to provide the 
necessary tools to prove the existence and uniqueness of the magnetostatic 
problem on the exterior domain of a torus where we reformulated the problem 
using the assumption that the first homology group being generated by the homology class 
of the curve that we are integrating over. We utilized existence and uniqueness results 
on the level of homology and applied the de Rham isomorphism to obtain analogous 
results for differential forms. These were then used together with the Hodge 
decomposition to prove existence and uniqueness.

In the second part, we looked at the 2D magnetostatic problem with curve integral 
constraint. We derived a convenient variational formulation and 
proved well-posedness as well as an a-priori estimate. We saw that the numerical results reach the 
theoretically derived convergence rates. 

One immediate possibility to extend these results would be to look for solutions on domain with more complicated 
topologies and different topological constraints. For example the first Betti number 
i.e. the number of generators of the first homology group could be increased
combined with more curve integrals. This generalization should be 
straightforward. 

We only showed the existence and uniqueness of the homogeneous magnetostatic problem 
with $\mathbf{J}=0$. Of course, the uniqueness for $\mathbf{J} \neq 0$ follows from 
that immediately because the problem is linear. With the existence however 
one has to be careful. The main issue is the application of the regularity results, but 
this would be possible if the assumptions on $\mathbf{J}$ are chosen carefully.

Another way to go would be to generalize the magnetostatic problem itself 
and pose it to find a differential form with exterior derivative and codifferential 
(see \cite[Sec.\,6.2.6]{arnold})
equal to zero.
One would have to be careful 
then to obtain the regularity and density results that we relied on which 
will most likely come down to certain regularity assumptions on the manifold itself.
This generalization would certainly not be easy, but would be a way to put 
this result into the general framework of differential forms together with 
the exterior derivative and the codifferential. To come back to the very
first sentence of this thesis, this would fit in a way to the core of 
mathematics: The beauty of abstraction.
\end{document}