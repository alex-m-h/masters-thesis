\documentclass[../master_thesis.tex]{subfiles}
\begin{document}
We developed lots of theoretical tools in the first chapter. We introduced 
differential forms, singular homology and Hilbert complexes to provide the 
necessary tools to prove the existence and uniqueness of the magnetostatic 
problem on a non-trivial exterior domain. 

In the second part, we looked at the 2D magnetostatic problem with curve integral 
constraint. We derived a convenient variational formulation for it and 
proved well-posedness and a-priori estimate. The numerical estimates reach the 
theoretically derived convergence rates. 

One immediate possibility to extend these results in on more complicated 
topologies with different topological constraints. For example the first Betti number 
could be a higher number combined with more curve integral. This generalization should be 
straightforward. 

The well-posedness of the magnetostatic could be put into abstraction using the 
exterior derivative and the codifferential (cf \cite) and then posed on arbitrary 
Riemannian manifolds. This would be another possible direction to go to.
\end{document}