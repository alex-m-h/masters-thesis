\documentclass[12pt,a4paper]{article}
\usepackage[utf8]{inputenc}
\usepackage[english]{babel}
\usepackage{amsmath}
\usepackage{amsfonts}
\usepackage{amssymb}
\usepackage{amsthm}
\usepackage{xcolor}
\usepackage{csquotes}
\usepackage{enumerate}
\usepackage{faktor}
\usepackage{mathrsfs}
\usepackage{mathtools}
\usepackage{showlabels}

\usepackage{biblatex}
\addbibresource{bibliography.bib}
% \bibliography{bibliography}
% \bibliographystyle{ieeetr}

\numberwithin{equation}{subsection}

\newtheorem{lemma}{Lemma}[section]
\numberwithin{lemma}{subsection}
\newtheorem{corollary}[lemma]{Corollary}
\newtheorem{proposition}[lemma]{Proposition}
\newtheorem{theorem}[lemma]{Theorem}

\theoremstyle{definition}
\newtheorem{assumption}[lemma]{Assumption}
\newtheorem{definition}[lemma]{Definition}
\newtheorem{example}[lemma]{Example}
\newtheorem{remark}[lemma]{Remark}
\newtheorem{problem}[lemma]{Problem}

\DeclareMathOperator*{\esssup}{ess\,sup}
\DeclareMathOperator{\curl}{curl}
\DeclareMathOperator{\diver}{div}
\DeclareMathOperator{\grad}{grad}
\DeclareMathOperator{\Ima}{im}
\DeclareMathOperator{\interior}{int}
\DeclareMathOperator{\sgn}{sgn}
\DeclareMathOperator{\supp}{supp}
\DeclareMathOperator{\vol}{vol}

\newcommand{\aop}{\mathscr{A}}
\newcommand{\alternating}[2]{ {\text{Alt}^{#1}\,#2} }
\newcommand{\integers}{\mathbb{Z}}
\newcommand{\smoothcompforms}[2]{C_0^\infty \Lambda^{#1}(#2)}
\newcommand{\lpcoho}{H^k_{p,dR}}
\newcommand{\naturalnum}{\mathbb{N}}
\newcommand{\norm}[2]{\lVert #1 \rVert_{#2}}
\newcommand{\omegabar}{\overline{\Omega}}
\newcommand{\rational}{\mathbb{Q}}
\newcommand{\real}{\mathbb{R}}
\newcommand{\rop}{\mathscr{R}} % short for R operator


\begin{document}

\section{Introduction}
Our goal is to study the homogeneous magnetostatic problem on the exterior 
domain of a triangulated torus. That means 
that for the unbounded domain  $\Omega \subseteq \real^3$ we have
$\real^3 \setminus \Omega$ is a triangulated torus. We also need a 
piecewise straight (i.e. triangulated) closed curve around the torus.
%%% TBD: Picture
{\color{red} (TBD: Define the "triangulated torus" more rigorous)}

Let $B$ be a magnetic field on the domain $\Omega$.
We the have the following boundary value problem:

\begin{align}
    \curl \, B &= 0, \\ 
    \diver \, B  &= 0 \text{ in } \Omega \\
    B \cdot n &= 0 \text{ on } \partial \Omega \text{ and }\\
    \int_\gamma B \cdot dl &= C_0
\end{align}
where $n$ is the outward normal vector field on $\partial \Omega$ and 
$C_0 \in \real$. We want to prove existence and uniqueness of 
solutions. In order to do so we will need to introduce Sobolev spaces of 
differential forms and basics from
simplicial topology {\color{red} among other things...}

\section{Differential forms}


\subsection{Alternating maps} \label{sec:alternating_maps}

For the introduction of alternating maps we follow
the short section in Arnold's book
\cite[Sec. 6.1.]{arnold} combine it 
with material from \cite[Sec.\,V.1]{topology_and_geometry}.
However, more arguments and additional details are provided especially  
in the the part about scalar and vector proxies.


Let $V$ be a real vector space with $\text{dim}\,V = n$.
Then $k$-linear maps are of the form
\begin{align*}
    \omega: \underbrace{V \times V \times ... \times V}_{k \text{ times}}
    \rightarrow \real
\end{align*}
that are linear in every component. We call a $k$-linear form 
\textit{alternating} if the sign switches when two arguments are exchanged i.e.
\begin{align*}
    \omega(v_1,...,v_i,...,v_j,...,v_k)
    = - \omega(v_1,...,v_j,...,v_i,...,v_k), \text{ for } 1\leq i < j \leq k,
    \quad v_1,...,v_k \in V.
\end{align*}
The sign $\sgn(\pi)$ of a permutation 
$\pi: \{ 1,2,...,n\} \rightarrow \{ 1,2, ..., n\}$ is equal to $(-1)^p$ where 
$p \in \naturalnum$ is the number of transpositions required to achieve the 
permutation. For example, the permutation $(1,2,3,4) \mapsto (2,3,1,4)$ 
can be built by performing the transpositions $(1,2)$ and $(1,3)$ so the 
sign of this permutation would be $1$. 
This also means for any permutation 
$\pi: \{1,2,...,k\} \rightarrow \{1,2,...,k\}$
\begin{align*}
    \omega(v_{\pi(1)},v_{\pi(2)},...,v_{\pi(k)})
    = \sgn(\pi)\, \omega(v_1, v_2,..., v_k).
\end{align*}
Denote the space of alternating maps by $\text{Alt}^k\,V$. 

For $\omega \in 
\alternating{k}{V}$, $\mu \in 
\alternating{l}{V}$ we define the wedge product $\omega \wedge \mu \in 
\alternating{k+l}{V}$ 
\begin{align*}
    (\omega \wedge \mu) (v_1,...,v_k,v_{k+1},...,v_{k+l}) =
    \sum\limits_\pi
    \text{sgn}(\pi) \omega(v_{\pi(1)},...,v_{\pi(k)}) 
    \nu(v_{\pi(k+1)},...,v_{\pi(k+l)})
\end{align*}
where we sum over all permutations 
$\pi: \{1,...,k+l\} \rightarrow \{1,...,k+l\}$ 
s.t. $\pi(1) < ... < \pi(k)$ and $\pi(k+1) < ... < \pi(k+l)$.
This definition is not very intuitive.
{\color{red} TBD: Examples in 3D}.

Let us mention some important properties of the wedge product. It is 
associative, but not commutative. But we 
have for $\omega \in 
\alternating{k}{V}$, $\mu \in 
\alternating{l}{V}$
\begin{align}
    \omega \wedge \mu = (-1)^{kl} \mu \wedge \omega. \label{eq:commutativity_wedge_product}
\end{align}
Recalling the definition of the sign of a permutation $\pi \in \mathcal{S}_k$ 
we get for linear forms $\omega_1, \omega_2, ..., \omega_k \in V'$
\begin{align*}
    \omega_{\pi(1)} \wedge \omega_{\pi(2)} \wedge ... \wedge \omega_{\pi(k)}
    = \sgn(\pi) \, \omega_1 \wedge \omega_2 \wedge ... \wedge \omega_k
\end{align*}
and if a linear form appears twice then the expression is zero.

There is a useful formula for computing the wedge product of linear forms.
For $1$-forms i.e. linear functionals $\omega_1,...,\omega_k$ 
, $k \leq n$ we have the formula (\cite[p.260]{topology_and_geometry})
\begin{align}
    \omega_1 \wedge ... \wedge \omega_k (v_1,...,v_k)
    = \det (\omega_s(v_t))_{1\leq s,t \leq n}. 
    \label{eq:wedge_product_of_one_forms}
\end{align}

Let $\{ b_i\}_{i=1}^n$ be any basis of $V$ and $\{ b^i\}_{i=1}^n$ the 
correspoding dual basis i.e. 
$b^i \in V'$, $b^i(u_j) = \delta_{ij}$ for $i,j = 1,2,..., n$. Then 
\begin{align*}
    \{b^{i_1} \wedge b^{i_2} \wedge ... \wedge b^{i_k} | \, 
    1 \leq i_1 < ... < i_k \leq n \}
\end{align*}
is a basis of $\alternating{k}{V}$. In particular, 
$\dim\, \alternating{k}{V} = \binom{n}{k}$.

Given a inner product $\langle\cdot, \cdot \rangle_V$ on $V$ we obtain an inner 
product on the dual space $V'$ by using the Riesz isomorphism $\Phi$ 
\begin{align*}
    \langle \Phi v, \Phi w \rangle_{V'} \vcentcolon= \langle v, w \rangle_V.
\end{align*}
Now we can define an inner product on $\alternating{k}{V}$ by defining
\begin{align*}
    \langle b^{i_1} \wedge b^{i_2} \wedge... \wedge b^{i_k}, 
    b^{j_1} \wedge... \wedge b^{j_k} \rangle_{\alternating{k}{V}} 
    \vcentcolon= \det  ( \langle b^{i_k}, b^{i_l} \rangle_V )_
    {1\leq k,l \leq n} 
\end{align*}
which is then extended to all of $\alternating{k}{V}$ by linearity. 
We denote with $\lvert \cdot \rvert _\alternating{k}{V}$ the induced norm.
For an orthonormal basis 
$u_1$, ..., $u_n$ the corresponding basis 
$u^{i_1} \wedge v^{i_2} \wedge ... \wedge u^{i_k}$, 
$1\leq i_1 < ... < i_k \leq n$ is an orthonormal basis of $\alternating{k}{V}$.


Next, we want to introduce the \textit{pullback} as the most natural 
mapping between alternating maps. Let $V$ and $W$ be finite-dimensional real
vector spaces with ordered bases $(b_i)_{i=1}^n$ and $(c_j)_{j=1}^m$ 
respectively. We write a basis in standard brackets $(\cdot)$ if the basis is 
ordered. Let $L \in \mathcal{L}(V,W)$ where $\mathcal{L}(V,W)$ is the 
space of linear mappings from $V$ to $W$. For $\omega \in \alternating{k}{W}$
we define the pullback $L^* \omega$ via 
\begin{align*}
    (L^* \omega)(v_1,...,v_k) = \omega(L\,v_1,...,L\,v_k).
\end{align*}
It is then easy to see that 
$L^*$ is a linear mapping from $\alternating{k}{W}$ to 
$\alternating{k}{V}$. 

It is obvious from the definitions of the exterior product and 
the pullback that we have 
\begin{align*}
    L^*(\omega \wedge \nu) = L^*\omega \wedge L^* \nu \quad \forall 
        \omega \in \alternating{k}{W}, \; \nu \in \alternating{l}{W}.
\end{align*}

Let $A \in \real^{m\times n}$ be the matrix representation of $L$ in the 
above bases. Because $\{ b^{i_1}\wedge b^{i_2}\wedge ... \wedge b^{i_k} 
 \mid 1 \leq i_1 < ... < i_k  \leq n \}$ and 
 $\{ c^{j_1}\wedge c^{j_2}\wedge ... \wedge c^{j_k} 
 \mid 1 \leq j_1 < ... < j_k  \leq m \}$ are bases for $\alternating{k}{V}$ 
and $\alternating{k}{W}$ respectively, we can find $\lambda_{i_1 ... i_k}$ s.t.
\begin{align}
    L^* (c^{j_1} \wedge c^{j_2} \wedge ... \wedge c^{j_k})
    = \sum\limits_{1 \leq i_1 < ... < i_k \leq n} 
    \lambda_{i_1 ... i_k} b^{i_1} \wedge b^{i_2} \wedge ... \wedge b^{i_k}
    \label{eq:basis_representation_pullback_unspecified_coefficents}
\end{align}
Now recall the formula for the wedge product of $1$-forms 
$\nu_i \in \alternating{1}{V} = V'$
\begin{align*}
    \nu_1 \wedge ... \wedge \nu_k (v_1,...,v_k) = 
    \det \big( \nu_s(v_t) \big)_{1 \leq s,t \leq n}.
\end{align*}
Fix now $1 \leq l_1 < ... <  l_k \leq n$. Then we get from this formula
$b^{i_1} \wedge b^{i_2} \wedge ... \wedge b^{i_k} ( b_{l_1},...,b_{l_k}) = 1$ 
iif. $(i_1,...,i_k) = (l_1,...,l_k)$. Here it is important to remember that 
these indices are ordered. Plugging this in 
(\ref{eq:basis_representation_pullback_unspecified_coefficents}) gives us 
\begin{align*}
    \lambda_{i_1 ... i_k}  &= 
        L^* (c^{j_1} \wedge c^{j_2} \wedge ... \wedge c^{j_k})
        (b_{l_1},...,b_{l_k})
    \\ &= c^{j_1} \wedge c^{j_2} \wedge ... \wedge c^{j_k} 
        (L\,b_{l_1},...,L\,b_{l_k})
    \\ &= \det \big( c^{j_s}(\sum_{r_t = 1}^m A_{r_t,l_t}c_{r_t}) 
        \big)_{1 \leq s,t \leq k}
    \\ &= \det \big( \sum_{r_t = 1}^m A_{r_t,l_t} \delta_{j_s,r_t}
        \big)_{1 \leq s,t \leq k}
    \\ &= \det \big( A_{j_s,l_t} \big)_{1 \leq s,t \leq k}
    \\ &= \det A_{(j_1,...,j_k),(i_1,...,i_k)}
\end{align*}
where $A_{(j_1,...,j_k),(i_1,...,i_k)}$ is the matrix we get 
by choosing the rows $j_1,...,j_k$  and the columns $i_1$, ... , $i_k$. 
Plugging this in (\ref{eq:basis_representation_pullback_unspecified_coefficents}) 
we arrive 
at the basis representation of the pullback 
\begin{align*}
    L^* (c^{j_1} \wedge c^{j_2} \wedge ... \wedge c^{j_k})
    = \sum\limits_{1 \leq i_1 < ... < i_k \leq n} 
        \det A_{(j_1,...,j_k),(i_1,...,i_k)} \,
        b^{i_1} \wedge b^{i_2} \wedge ... \wedge b^{i_k}
\end{align*}

We want to emphasize the special case of the pullback of a $n$-linear map 
with $m = n$. So take $\omega \in \alternating{n}{W}$. Then we know that 
$\omega = \lambda\, c^1 \wedge ... \wedge c^n$. The above forumla becomes
\begin{align}
    L^* \omega = \lambda \, \det A \, b^1 \wedge ... \wedge b^n
    \label{eq:pullback_alternating_nlinear_map}
\end{align}

We want to examine $\alternating{n}{V}$ a bit closer. 
$\alternating{n}{V}$ is one-dimensional and so we can choose a basis by fixing 
a specific non-zero element. We want to choose one specific element 
called the \textit{volume form} which will play a crucial role 
when we define integration on a manifold in Sec. \ref{}. We also need it to 
define the Hodge star operator below.

The choice of this volume form will depend on the orientation. 
We say that two bases of $V$ 
have the same orientation if the change of basis has positive determinant. 
That divides the bases into two classes with different orientation.
We choose one of these classes and call these bases positively
oriented. In $\real^n$, the convention is to define the class as 
positively oriented which includes the standard orthonormal basis.

Let $(b_i )_{i=1}^n$ be any positively oriented
basis. Let $G$ be the Gramian matrix i.e. $G_{ij} = \langle b_i, b_j \rangle$ 
which is always a symmetric positive definite matrix.
Then we define the \textit{volume form}
\begin{align*}
    \vol \vcentcolon= \sqrt{\det G} \, b^1 \wedge b^2 \wedge ... \wedge b^n.
\end{align*}
The claim is now that for any
orthonormal basis $\{u_i\}_{i=1}^n$ we have 
\begin{align*}
    \vol (u_1,\,u_2,...,\,u_n) = (-1)^s.
\end{align*}
with $s=0$ if $(u_1,...,u_n)$ has the same orientation as $(b_i)_{i=1}^n$ 
and $s=1$ otherwise.
Let us define the matrix $B \in \real^{n\times n}$, $B_{k,i} = \langle b_i, u_k
\rangle_V$ which is just the change of basis matrix from $( b_i )_{i=1}^n$ 
to $( u_i )_{i=1}^n$. 
Then using basic linear algebra we get
$ G = B^\top B$ and 
$\sqrt{\det G} = (-1)^s \det B$. Let now $\Psi$ be the linear map with 
$\Psi b_i = u_i$. In the basis $( b_i )_{i=1}^n$ this has the matrix 
representation $B^{-1}$ and so by using 
(\ref{eq:pullback_alternating_nlinear_map}) we get
\begin{align*}
    \vol (u_1,\,u_2,...,\,u_n) &= \sqrt{ \det G} \, b^1 \wedge ... \wedge b^n 
    ( \Psi b_1,..., \Psi b_n )
    \\ &= (-1)^s \det B \, \Psi^* (b^1 \wedge ... \wedge b^n )( b_1,..., b_n) 
    \\ &= (-1)^s \det B \, \det B^{-1} \, (b^1 \wedge ... \wedge b^n )( b_1,..., b_n)
    = (-1)^s.
\end{align*}
In particular, $\vol (u_1,\,u_2,...,\,u_n) = 1$ for any positively oriented 
ONB $(u_i)_{i=1}^n$.
This property also defines the volume form uniquely so it is independent of the 
chosen basis. It only depends on the orientation.
It also shows that $\vol$ is non-zero and thus 
\begin{align*}
    \alternating{n}{V} = \text{span} \{ \vol \}.
\end{align*}

Note that if we choose $\{ b_i \}_{i}$ to be an orthonormal basis to begin 
with the Gramian matrix is just the identity and 
$\vol = b^1 \wedge ... \wedge b^n$. Especially in the case of $\real^n$ if 
we denote the standard dual basis by $\{ dx^i \}_{i=1}^n$ then 
\begin{align*}
    \vol = dx^1 \wedge ... \wedge dx^n.
\end{align*}

We will from now on assume that we fixed a orientation on $V$. 
Using the resulting volume form on $V$
we can now define the \textit{Hodge star operator}.

Let denote with $\vol'$ the dual basis of $\vol$ i.e.
$\vol'(\vol) = 1$. Let us fix $\omega \in \alternating{k}{V}$. 
Then we can define the following linear form on $
\alternating{n-k}{V}$
\begin{align*}
    \mu \mapsto \vol' (\omega \wedge \mu).
\end{align*}
Let $\Phi$ be the Riesz isomorphism for $\alternating{n-k}{V}$. Then we define
$\star \omega$ as the Riesz representative of this linear form that means we 
have $\vol'(\omega \wedge \mu) = 
\langle \star \omega, \mu \rangle_{\alternating{k}{V}}$ for all 
$\mu \in \alternating{n-k}{V}$ i.e.
\begin{align}
    \omega \wedge \mu = \langle \star \omega, \mu \rangle_{\alternating{k}{V}}
    \vol \quad \forall \mu \in \alternating{n-k}{V}.
    \label{eq:hodge_star_definition}
\end{align}
It is also clear from the uniqueness of the Riesz representative that 
the $\star\omega$ is uniquely determined by the above condition.

For an orthonormal basis $\{u_i \}_{i=1}^n$ we 
can have simple expressions for the Hodge star applied to the basis elements 
of alternating forms which are 
\begin{align}
    \star u^{i_1} \wedge u^{i_2} \wedge ... \wedge u^{i_k}
    = \sgn(i_1,i_2,...,i_n) u^{i_{k+1}} \wedge ... \wedge u^{i_{n}}
    \label{eq:hodge_star_orthonormal_basis}
\end{align}
with $\{i_1, i_2 , ... , i_n\} = \{1,2,...,n\}$ and $\sgn(i_1,i_2,...,i_n)$ 
is the sign of the permutation $j \mapsto i_j$.
For a three dimensional 
space this gives us
\begin{align*}
    \star u^1 = u^2 \wedge u^3, \; \star u^2 = - u^1 \wedge u^3
    \; \star u^3 = u^1 \wedge u^2. 
\end{align*}

We see that the Hodge star maps orthonormal bases to orthonormal bases 
and is thus an isometry.
We can then derive from the defining property of the Hodge star 
that $\star\star \omega= (-1)^{k(n-k)}\omega$ 
for $\omega \in \alternating{k}{V}$ and 
\begin{align*}
    \omega \wedge \star\mu = \langle \omega, \mu \rangle_{\alternating{k}{V}} 
    \text{vol} \quad \forall \omega, \mu \in \alternating{k}{V}.
\end{align*}
In particular in $\real^3$, we have 
$\star\star = \text{Id}$ i.e. $\star$ is self-inverse.

Let us quickly derive the expression in a basis 
for the Hodge star applied to linear 
forms which we will need later. Let $\omega = \sum_i \omega_i b^i \in 
\alternating{1}{V} = V'$. Let us denote $g^{ij} = \langle b^i, b^j \rangle$.
Then we claim
\begin{align}
    \star \omega = \sqrt{\det G} \sum_{i,j=1}^n \omega_i (-1)^{j-1} 
        g^{ij} b^1 \wedge
        b^2 \wedge ... \wedge \widehat{b^j} \wedge b^n 
\end{align}
where $\widehat{b^j}$ means that this term is left. The proof is very 
simple in this case. For any $1 \leq l \leq n$ we get
\begin{align*}
    &\left( \sqrt{\det G} \sum_{i,j=1}^n \omega_i (-1)^{j-1} g^{ij} b^1 \wedge
        b^2 \wedge ... \wedge \widehat{b^j} \wedge b^n\right) \wedge b^l 
    \\ &= \sqrt{\det G} \sum_{j=1}^n \langle b^j,\sum_{i=1}^n \omega_i  b^i\rangle
        (-1)^{j-1}  b^1 \wedge
        b^2 \wedge ... \wedge \widehat{b^j} \wedge b^n \wedge b^l
    \\ &= \sqrt{\det G} \langle b^l,\omega \rangle
        (-1)^{2(l-1)}  b^1 \wedge
        b^2 \wedge ... \wedge b^l  \wedge ... \wedge b^n 
    \\ &= \langle b^l,\omega \rangle \vol.
\end{align*}
In the second step, we used that if $j\neq l$ then one basis element must 
appear twice in the wedge product which is then zero. If $j=l$ then 
$\sgn (1,2, ..., \hat{l},...,n,l) = (-1)^{l-1}$ because we need $(l-1)$
transpostions to bring the indices into order.

Then by linearity we obtain (\ref{eq:hodge_star_definition}) and thus 
the given expression is indeed equal to $\star b^i$ as claimed.


\subsection{Scalar and Vector proxies}
Now we want to relate alternating maps to elements of the 
vector space $V$ itself or to scalars. Let us start with the easiest 
case. $\alternating{0}{V}$ are already scalars by definition. Now we can use 
the Hodge star operator which is an isometry 
$\star: \alternating{0}{V} \rightarrow \alternating{n}{V}$ with
$\star (c) = c \vol$. 
We call the real number that is associated with an element of 
$\alternating{n}{V}$ \textit{scalar proxy} i.e. the scalar proxy of 
$c \in \real$ is just $c \vol \in \alternating{n}{V}$.

Next, we will move on to $\alternating{1}{V}$ and $\alternating{n-1}{V}$. 
Let $\Phi: V \rightarrow V'$ denote the Riesz isomorphism which is an isometry.
Because $V' = \alternating{1}{V}$ this gives us the correspondence of 
vectors and linear forms. Now we can once again use the Hodge star and obtain 
the isometry $\star \Phi: V \rightarrow \alternating{n-1}{V}$. We call 
the vectors associated with an alternating $1$- or $(n-1)$-linear map 
\textit{vector proxy}.

These way to identfy alternating maps with scalars and vectors gives us the 
ability to look at the notions defined above in the context of scalars 
and vectors.

Let us look at the wedge product. We have for $v,w \in V$
\begin{align*}
    \Phi v \wedge \star \Phi w = \langle \Phi v , \Phi w \rangle_{V'} \vol
    = \langle v , w \rangle_{V} \vol
\end{align*}
which means that the wedge product of a linear from and an alternating $(n-1)$-
linear map corresponds in proxies to the inner product.

Note that for $n=2$ the situation 
is slightly ambiguous, see \cite[p.67]{arnold}. But this case will not be
relevant in this thesis. % TBD: Are you sure?

In the case of $V= \real^3$ with the standard basis
vectors $e_1$, $e_2$ and $e_3$. Denote the resulting elements 
of the dual basis with $e^1$, $e^2$ and $e^3$ respectively. 
Take $v = v_1 e_1 + v_2 e_2 + v_3 e_3 \in \real^3$ 
and recall that for a orthonormal basis the Riesz isomorphism maps basis 
elements to their dual basis elements i.e. $\Phi e_i = e^i$. Hence, 
we get $\Phi v = v_1 e^1 + v_2 e^2 + v_3 e^3$. Take another $w \in \real^3$.
Then using the $e^i \wedge e^j = - e^j \wedge e^i$ we get 
\begin{align}
    \Phi v \wedge \Phi w = \star \Phi (v \times w). 
    \label{eq:cross_product_as_wedge_product}
\end{align}
That means in $3D$ in terms of vector proxies, the wedge product 
of two linear forms corresponds to the cross product. Note that 
(\ref{eq:cross_product_as_wedge_product}) is formulated without using a 
specific basis and can therefore be computed using any basis i.e. 
if we have $v = \tilde{v}_1 b_1 + \tilde{v}_2 b_2 + \tilde{v}_3 b_3$ 
and analogous for $w$ we could still calculate the cross product directly as 
\begin{align*}
    v \times w = \Phi^{-1} \star^{-1} \big(\Phi v \wedge \Phi w \big).
\end{align*}
One has to take care though because if the basis is not orthonormal 
the Riesz isomorphism does not map basis elements $b_i$ to their respective 
dual basis elements $b^i$. We have 
\begin{align*}
    \Phi b_i = \sum_{i=1}^{n} \langle b_j, b_i \rangle b^j
\end{align*}
i.e. it has the gramian matrix $G$ as basis representation. 
This is easy to 
see. Let $\Phi b_i = \sum_j \lambda_j b^j$. Then
\begin{align*}
    \lambda_j = \Phi b_i (b_j) = \langle b_i, b_j \rangle.
\end{align*}
As derived above, the Hodge star is not as trivial to compute either.

This illustrates the idea of a coordinate free description which is a important
notion in differential geometry (see \cite{}). 

Similarly, we want to explore the pullback in terms of vector proxies as well. 
These will be important in the next section when we talk about the pullback 
of differential forms and apply these to the tranformation of integrals.
In order to avoid complicated computations we will stick to orthonormal bases.
Let $\{ b_i \}_{i=1}^n$ be an ONB of $V$ and 
$\{ c_j \}_{j=1}^m$ be an ONB of $W$. Let $L : V \rightarrow W$ 
again be a linear map and $A$ be the basis representation of it w.r.t. 
the two bases given i.e. $L b_i = \sum_j A_{ji} c_j$. 
Then we get the pullback of linear forms in terms 
of vector proxies 
as $\Phi_V^{-1} L^* \Phi : W \rightarrow V$. Let us apply 
formula (\ref{eq:pullback_alternating_nlinear_map}) to compute the basis 
representation of it in the corresponding dual bases.
\begin{align*}
    \Phi_V^{-1} L^* \Phi c_j = \Phi_V^{-1} L^* c^j 
    = \Phi_V^{-1} \sum_{i=1}^n A_{j,i} b^i = \sum_{i=1}^n A_{j,i} b_i
\end{align*}
so the matrix representation of the pullback is $A^\top$. 

For $m=n$ let us look at the pullback of alternating $(n-1)$-linear maps.
In terms of vector proxies this can then be expressed as 
$\Phi_V^{-1} \star^{-1} L^* \star \Phi_W$. Note that we used the same symbol 
$\star$, but it is once applied in $W$ and then the inverse in $V$. It 
can be shown with the same ideas and \ref{eq:hodge_star_orthonormal_basis}
that the matrix representation is the adjugate matrix 
$\text{ad}(A)$ defined as 
\begin{align*}
    \text{ad}(A)_{ij} = (-1)^{i+j} \det A_{-j,-i}
\end{align*} 
where $A_{-j,-i}$ is the matrix without the $j$-th row and $i$-th column.
If $A$ is invertible then $(\det A)\,A^{-1} =  \text{ad}(A)$. 

The pullback of $n$-linear mappings in terms of vector proxies is 
$\star^{-1} L^* \star$. Again n the case of $n=m$ we get for $c \in \real$
\begin{align*}
    \star^{-1} L^* \star c = \star^{-1} L^* c \vol 
    = \star^{-1} c\,\det L \,\vol = \det L.
\end{align*}

\subsection{Differential forms}\label{sec:differential_forms}

Before we define differential forms, let us start by revising some basics
from differential geometry. We follow the approach from 
\cite[Sec. II]{topology_and_geometry}. 

In order to formulate the definition of a manifold, let us recall the 
definition of a topological space.
\begin{definition}[Topological space]
    A topological space is a set $X$ together with collection of subsets of 
    $X$ denoted by $\mathcal{T}$ s.t.
    \begin{itemize}
        \item $U,V \in \mathcal{T} \Rightarrow U \cap V \in \mathcal{T}$
        \item for $\{ U_i \in \mathcal{T} \mid i \in \mathcal{I} \}$
            for any index set $\mathcal{I}$, 
            $\bigcup_{i\in \mathcal{I}} U_i \in \mathcal{T}$ and
        \item $\emptyset, X \in \mathcal{T}$.
    \end{itemize}
    The sets contained in $\mathcal{T}$ are called \textit{open}.
\end{definition}
For example, a metric space together with its usual open sets is a topological
space. Another well known example of topologies which do not arise from a metric
are the weak and weak-$\star$ topology on infinite dimensional spaces.

\begin{definition}[Second countable topological space]
    Let $(X,\mathcal{T})$ be a topological space. Then we call 
    $\mathcal{B}\subseteq \mathcal{T}$ a basis for the topology of $X$ if 
    every open set (i.e. every set in $\mathcal{T})$ is a union of sets 
    in $\mathcal{B}$. If a topological space has a countable basis it is called
    \textit{second countable}.
\end{definition}
$\real^n$ is an example of a second countable topological space. Consider 
the countable set of balls $\{ B_r(x) \mid r \in \rational, x \in \rational^n\}$
where $ B_r(x)$ are the balls with center $x$ and radius $r$.
Then it is trivial to show that any open set in $\real^n$ is a union of 
of these balls. Hence, $\real^n$ is second countable.

Let us denote $\real_- \vcentcolon= \{ x \in \real \mid x \leq 0 \}$. 
Let us in the following equip $\real_- \times \real^{n-1} \subseteq \real^n$ with the 
subspace topology i.e. we call a set $V \subseteq \real_- \times \real^{n-1}$ 
open iif. there exists an open set $V' \subseteq \real^n$ s.t. 
$V = V' \cap \real_- \times \real^{n-1}$. This means e.g. that 
$B_1(0) \cap \real_- \times \real^{n-1}$ is open which is not an open set 
in the standard topology of $\real^n$.

\begin{definition}[Manifold with boundary]
    A smooth $n$-dimensional manifold with boundary is a second 
    countable Hausdorff 
    space $M$ with an open cover $\{ U_i \} _{i\in I}$ with some index set $I$ 
    and a collection of maps called \textit{charts} $\phi_i$, $i\in I$ s.t.
    \begin{itemize}
        \item $\phi_i: U_i \rightarrow V_i \subseteq \real_- \times \real^{n-1}$
            are homeomorphisms
        \item for two charts $\phi_i$, $\phi_j$ the 
            \textit{change of coordinates} $\phi_j \circ \phi_i^{-1}: 
            \phi_i(U_i \cap U_j) \rightarrow \phi_j(U_i \cap U_j)$ 
            is a $C^\infty$ diffeomorphism.
    \end{itemize}
    When we write $(U_i, \phi_i)$ we mean the chart $\phi_i$ has domain $U_i$.
\end{definition}
% TBD: I am not entirely sure about maximality, Definition of boundary

\begin{definition}[Orientation of a manifold]
    We call an atlas \textit{oriented} if the Jacobian of the coordinate
    changes has positive determinant. A manifold that can be equipped with 
    an oriented atlas is called \textit{orientable}.
\end{definition}

The next important concept we will recall are tangent spaces. 
It should be noted that there are different definitions of tangent space, but
these lead to isomorphic notions 
(see e.g. \cite[Sec.\,1.B]{riemannian_geometry}).
Let $M$ be an $n$-dimensional smooth manifold with boundary.
For a point $p \in M$ and a neighorhood $U$ we call a function 
$f: U \rightarrow \real$ differentiable at $p$ if for a local chart 
$\phi: U \rightarrow \real^k$ we have that $f \circ \phi^{-1}$ is differentiable
at $\phi(p)$.

This defintion is independent of the chart. Let $(V,\psi)$ with $p \in V$ be 
another chart. 
Then $f\circ \psi^{-1} = f \circ \phi^{-1} \circ \phi \circ \psi^{-1}$ and 
because $\phi \circ \psi^{-1}$ is a diffeomorphism it is differentiable as well.

These type of definitions via local charts on a manifold are frequent in
differential geometry. This is a proper definition if it is independent of the 
chosen chart. Because we do not want to bother with the technicalities of 
differential geometry too much we will very often leave out these types of
proofs.  

Let $I \subseteq \real$ be an interval containing $0$ and 
$\gamma: I \rightarrow M$ be a differentiable curve with $\gamma(0) = p \in M$.
For a differentiable $f: U \rightarrow \real$ 
we define the the directional derivative 
$D_\gamma(f) \vcentcolon= \frac{d}{dt} f(\gamma(t)) |_{t=0}$.
We call the functional $D_\gamma: C^1(U) \rightarrow \real$ 
tangent vector. The vector space of all tangent vectors is called the 
\textit{tangent space} and denoted by $T_p M$

We define 
\begin{align}
    \frac{\partial f}{\partial x_i} (p) 
    = \frac{\partial (f \circ \phi^{-1})}{\partial x_i}(\phi(p)).
    \label{eq:derivative_on_manifold}
\end{align}
Let us emphasize that this depends on the chosen chart.

We can now 
express a tangent vector $D_\gamma$ by 
\begin{align}
    D_\gamma(f) &=  \frac{d}{dt} f(\gamma(t)) \big|_{t=0}
    =  \frac{d}{dt} (f \circ \phi^{-1} \circ \phi)  (\gamma(t)) \big|_{t=0}
    \\ &= \sum\limits_{i=1}^k \frac{\partial (f \circ \phi^{-1})}{\partial x_i} 
        (\phi(p))
        \, (\phi_i\circ \gamma)'(0)
    = (\sum\limits_{i=1}^k v_i  \frac{\partial}{\partial x_i}\Big|_p )(f).
\end{align}
Here $\phi_i$ is the $i$-th component of the chart $\phi$.

Thus we can express 
\begin{align*}
    D_\gamma = \sum\limits_{i=1}^k 
    (\phi_i \circ \gamma)' (0) \, \frac{\partial}{\partial x_i}\bigg|_p.
\end{align*}
So we have that 
\begin{align*}
    T_p M = \text{span}\, 
        \bigg\{ \frac{\partial}{\partial x_i}\bigg|_p \bigg\}_{i=1}^n.
\end{align*}
We will show that this indeed a basis. 
Assume we have $\sum_{i=1}^n \lambda_i \, \partial/\partial x_i|_p = 0$
Then because $\phi_j \circ \phi^{-1} (x) = x_j$ for $x \in \phi(U)$ and 
$1 \leq j \leq n$. Then we have 
\begin{align*}
    0 = \left( \sum\limits_{i=1}^n \lambda_i \frac{\partial}{\partial x_i}|_p
        \right) (\phi_j)
    = \sum\limits_{i=1}^n \lambda_i \frac{\partial x_j}{\partial x_i}(\phi(p))
    = \lambda_j
\end{align*}
so $\frac{\partial}{\partial x_i}|_p $ are
linearly independent and thus a basis of $T_p M$. From now on we will often 
leave out the reference to the specific point $p$ if the context allows it.

If we now take a different chart $\psi$ and let us denote the resulting 
basis of $T_p M$ by $\frac{\partial}{\partial y_j}$. Then the question arises 
what the change of basis is between these bases. Using the chain rule 
we can easily compute that 
\begin{align*}
    \frac{\partial (f \circ \phi^{-1})}{\partial x_i} (\phi(p))
    =\sum_{j=1}^n \frac{\partial (f \circ \psi^{-1})}{\partial y_j} (\psi(p))
        \frac{\partial (\psi \circ \phi^{-1})_j}{\partial x_i}(\phi(p)) 
\end{align*}
and we recognize that the change of basis matrix is the Jacobian of 
the chart transition $D(\psi \circ \phi^{-1})(\phi(p))$.

A \textit{vector field} $X$ maps every point $p$ to a tangent vector 
in the corresponding tangent space i.e. by using local coordinates
\begin{align*}
    X(p) = \sum_{i=1}^n X_i(p) \frac{\partial}{\partial x_i}
\end{align*}
with $X_i(p) \in \real$. We call a vector field differentiable if 
the $X_i$ are differentiable. Using the change of basis above we see that 
for a smooth manifold we the notion of differentiable is well-defined because
the Jacobian of the chart transition $D(\psi\circ \phi^{-1})$ is smooth.

\begin{definition}[Differential forms]
    A differential $k$-form $\omega$ maps any point $p \in M$ to a 
    alternating $k$-linear mapping $\omega_p \in \alternating{k}{T_p M}$.
    We denote the space of differential $k$-forms on $M$ as $\Lambda^k M$.
\end{definition}

Let $T_p^* M$ be the dual space of $T_p M$ which is usually called 
\textit{cotangent space}.
As before let us choose a local chart $\phi: U \rightarrow \real^n$ with 
$p \in U$ and define $\frac{\partial}{\partial x_i}|_p$ as before. 
Denote the corresponding
dual basis as $dx^i$, $i = 1,...,n$. From the consideration about
alternating maps from section \ref{sec:alternating_maps} we can now write any 
$\omega \in \Lambda^k M$ with 
\begin{align*}
    \omega_p = \sum\limits_{1\leq i_1 < ... < i_k \leq n} 
        a_{i_1,...,i_k}(p) dx^{i_1} \wedge dx^{i_2} \wedge ... \wedge dx^{i_k}
\end{align*}
with $a_{i_1,...,i_k}(p) \in \real$. The regularity of differential forms 
is then defined via the regularity of these coefficents i.e. we call 
a differential form smooth if all the $a_{i_1,...,i_k}$ are smooth 
and we call a differential form differentiable if all the $a_{i_1,...,i_k}$
are differentiable and so on.

Here we can now apply the result \ref{eq:pull}

We denote the space $C^\infty \Lambda^k M$ the 
space of smooth differential $k$-forms and analogous for other regularity.

$\smoothcompforms{k}{M}$ are the smooth differential forms
with compact support contained in $M$ i.e. 
$\supp \omega = \overline{\{ p \in M \mid \omega_p \neq 0  \}} 
\subseteq M \setminus \partial M$ where the closure is w.r.t. 
the topology on $M$.
These will become very crucial later when we discuss Sobolev spaces 
of differential forms (see Sec. \ref{}). 

In order to define the Hodge star and an inner product on differential forms
we need that 
$T_p M$ is an inner product space.
A Riemannian metric gives us at every point $p \in M$ 
a symmetric, positive definite bilinear form 
$g_p: T_p M \times T_p M \rightarrow \real$. Additionally, a Riemannian metric 
is assumed to be smooth in the sense that for smooth vector fields 
$X$ and $Y$ we have $p \mapsto g_p(X(p),Y(p))$ is a smooth function. The degree 
of smoothness depends on the smoothness of the manifold. 
{\color{red} More details... }
Manifolds on which a Riemannian metric is defined are called 
\textit{Riemannian manifolds}. The Riemannian metric provides us with the 
inner product on every tangent space $T_p M$. 

We will from now on assume that $M$ is a Riemannian manifold. 
We denote the Riemannian metric by $g$.
%TBD: Lipschitz
Let $p \in M$ and $T_p M$ be the tangent space at the point $p$. 
Due to our assumptions on $M$, this is an inner product space of 
dimension $n$ and we can apply 
all of the constructions from the previous chapter. 
Let us go through them one by one. 
Let us fix a point $p$ and a chart $\phi$ 
at this point with local coordinates denoted by $x_i$, $i=1,...,n$. 
First, we have to check that we choose a orientation on $T_p M$.
We can do so by only considering charts with the same orientation then 
we know that the change for the resulting basis of the tangent space 
the change of basis -- the Jacobian of the chart transition -- 
has positive determinant so all these bases have the same orientation.

The resulting gramian matrix is
$(G_p)_{ij} = g_p(\frac{\partial}{\partial x_i},\frac{\partial}{\partial x_j})$.
So we have a volume form vol on $M$ 
\begin{align*}
    \vol_p = \sqrt{\det G_p} dx^1 \wedge ... \wedge dx^n.
\end{align*}
Because we only consider charts that have the same orientation the volume form 
will be the same independent of the chart.

Let $\{(U_i,\phi_i)\}_{i=1}^\infty$ be an oriented atlas of $M$.
For $p \in U_i$ we can define using local coordinates
\begin{align*}
    g_{kl}^{(i)}(p) \vcentcolon= g_p(\frac{\partial}{\partial x^{(i)}_k}\bigg|_p, 
        \frac{\partial}{\partial x^{(i)}_l}\bigg|_p) 
\end{align*}
where we use the superscript $^{(i)}$ to mean the local coordinates for 
chart $\phi_i$.
Then we define the matrix $G^{(i)} \vcentcolon= 
(g_{kl}^{(i)})_{k,l} \in \real^{n \times n}$ which is just the resulting Gramian 
matrix. 
Then we obtain the volume form 
\begin{align*}
    \text{vol}_p = \det G^{(i)} 
    dx_{(i)}^1 \wedge dx_{(i)}^2 \wedge dx_{(i)}^n
\end{align*}
where $dx_{(i)}^k$ are the dual basis corresponding to 
$\partial/\partial x^{(i)}_k$ the local coordinates 
of chart $\phi_i$. 

We know that this works if we choose any positively oriented basis. 
Because our manifold is orientable and we chose a oriented atlas 
$(U_i,\phi_i)$ we get that if we choose a different chart $(U_j,\phi_j)$ 
then the 
corresponding basis of the tangent space
$\{ \partial / \partial x^{(j)}_k \}_k$ is also positively oriented because
the change of basis matrix $D(\phi_j \circ \phi_i^{-1})(\phi(p))$ has positive
determinant.

We define the Hodge star operator to differential forms 
$\star: \Lambda^k(\Omega) \rightarrow \Lambda^{n-k}(\Omega)$ simply by applying it 
pointwise i.e. $(\star \omega)_p = \star \omega_p$. In order for 
the Hodge star to be well-defined the assumption of an orientation on our 
manifold is crucial. We do the same for the exterior product to get
$\wedge: \Lambda^k M \times \Lambda^l M \rightarrow \Lambda^{k+l} M$. 

We want to apply two important concepts from the previous section about 
alternating maps -- vector proxies and pullbacks -- to differential forms. 
Recall, that for a real $n$-dimensional vector space $V$ we had two 
ways to identify a vector $v\in V$ with an alternating map. Either as a 
linear form $\Phi v$ where $\Phi$ is the Riesz isomorphism or as a 
$(n-1)$-linear alternating map $\star \Phi v$. 

A vector field $X$ maps every point $p \in M$ to a tangent vector 
$X(p) \in T_p M$. We can now identfy every vector field with a $1$-form or 
a $(n-1)$-form. $p \mapsto \Phi_{T_p M} X(p)$ defines a $1$-form and  
$p \mapsto \star \Phi_{T_p M} X(p)$ gives us a $(n-1)$-form. In differential 
geometry, the usual notation is
$\Phi_{T_p M} X(p) = X^\flat(p)$. The inverse of $^\flat$ is $^\sharp$ i.e. 
$X = (X^\flat)^\sharp$.
The isomorphisms $^\flat$ and $^\sharp$ 
are fittingly called \textit{musical isomorphisms}. 
With these musical isomorphisms we can identify $X$ with the $1$-form 
$X^\flat$ or the $(n-1)$-form $\star X^\flat$. Vice versa, we find 
for $\omega \in \Lambda^1 M$ the \textit{vector proxy} $\omega ^\sharp$ and for 
and $(n-1)$-form $\nu \in \Lambda^{n-1} M$ we get $(\star^{-1} \nu)^\sharp$.

Next, let us have a look at how we can extend pullbacks to differential forms.
Recall again, that a linear map $L: V \rightarrow W$ with an 
$n$-dimensional real vector space $V$ and an 
$m$-dimensional real vector space $W$ we define its pullback 
$L^*: \alternating{k}{W} \rightarrow \alternating{k}{V}$ via 
\begin{align*}
    L^*\omega (v_1, ..., v_k) = \omega (Lv_1, ..., L v_k).
\end{align*} 
We wish to do the analogous thing with differential forms. However, we have 
so far not discussed the necessary linear maps between tangent spaces, 
the \textit{pushforwards}. 

Let $M$, $N$ be $n$- and $m$-dimensional manifolds respectively. 
Let $F: M \rightarrow N$ be a smooth map between these manifolds (recall from 
above that smoothness is here defined via the charts). 
For $D_\gamma \in T_p M$ with $\gamma$ an appropriate curve we define the
pushforward $F_*: T_p M \rightarrow T_{F(p)} N$
\begin{align*}
    F_* D_\gamma \vcentcolon= D_{F\circ \gamma}.
\end{align*}
It is easy to see that this is indeed linear and well-defined. If 
we choose charts and the corresponding bases $\partial/\partial x_i |_p$, 
$i=1,...,
n$ and $\partial/\partial y_j |_{F(p)}$, $j=1,...,
m$ then the matrix representation of the pushforward is just the Jacobian
of $F$ i.e. 
\begin{align*}
    F_* \left(\sum_{i=1}^n v_i \frac{\partial}{\partial x_i}\big |_p\right)
    = \sum_{j=1}^n v_i \frac{\partial F_j}{\partial x_i}(p) 
    \frac{\partial}{\partial y_j}\big |_{F(p)}.
\end{align*}

For $\omega \in \Lambda^k N$ we now define the pullback $F^*\omega$ as
\begin{align*}
    (F^*\omega)_p = (F_*)^* \omega_{F(p)}
\end{align*}
or written differently for all $v_1,...,v_k \in T_p M$
\begin{align*}
    (F^*\omega)_p (v_1,...,v_k) = \omega_{F(p)}(F_* v_1, ..., F_* v_k).
\end{align*}

Now we can finally use the vector proxies and connect it to what we have done 
for alternating maps above. So let $X$ be a vector field on $N$. We want to 
investigate how the pullback of $X$ looks like if we identify $X$ with 
an $1$-form or with a $(n-1)$-form. For the sake of simplicity, we will
assume that on $M$ we have $g_p (\partial /\partial x_i, \partial /\partial x_j)
= \delta_{ij}$ and equally for the Riemannian metric at the point $T(p)$. 
This is the case if the manifolds are subdomains of $\real^n$ and $\real^m$ 
and we choose the standard Euclidian coordinates. Then after recalling 
that the matrix representation of $F_*$ is the Jacobian of $F$ we can 
apply the result from Sec.\,\ref{sec:alternating_maps} to get 
\begin{align*}
    (F^* X^\flat)^\sharp 
    = \Phi_{T_p M}^{-1} (F_*)^* \Phi_{T_{F(p)} N} 
    \sum_{j=1}^{m}X_j(F(p)) \frac{\partial}{\partial y_j}
    = \sum_{i=1}^n \sum_{j=1}^m X_j(F(p)) \frac{\partial F_j}{\partial x_i}(p)
        \frac{\partial }{\partial x_i}
\end{align*}
i.e. the matrix representation of the the pullback at point $p$
is given as the $DF(p)^\top$.

In an analogous way by using the corresponding result assuming $n = m$ 
we get that the 
matrix representation of the pullback of the corresponding $(n-1)$-form 
is $\text{ad}(DF(p))$. If $F$ is a diffeomorphism then we can write the
coefficent vector of the transformed vector field as 
$(\det DF(p)) \, DF(p)^{-1} \mathbf{X}(F(p))$ where 
$\mathbf{X} = (X_1, X_2, ..., X_n)^\top$ i.e. the vector of the coefficents 
of $X$. This is widely known as the Piola transformation \cite{Ern Guermond}.

Now let us move on to scalar proxies. So let $\rho: N \rightarrow \real$ be 
just a scalar field i.e. a $0$-form. In this case, we 
have the simple expression for the pullback $F^* \rho = \rho \circ F$. 

But $\rho$ could also be the scalar proxy of the $n$-form 
$\star \rho = \rho \vol_N$ (we assume once again 
as above
$n=m$ and that our basis of the tangent spaces are orthonormal w.r.t. the 
Riemannian metric). Then in scalar proxies 
\begin{align*}
    \star ^{-1}F^* \star \rho = \star^{-1} (\rho \circ F) (\det DF) \vol_M
        = (\rho \circ F) \det DF.
\end{align*}
This is strikingly similar to the integrand in the standard transformation 
of integrals formula which will become crucial in  
Sec.\,\ref{sec:integration_differential_forms} where we talk about the
integration of differential forms.

Let $\omega \in \Lambda^k (M)$ be given 
in local coordinates with some chart $(U,\phi)$ s.t. $p\in U$ as above,
\begin{align*}
    \omega_p = \sum\limits_{1\leq i_1 < ... < i_k \leq n} 
        a_{i_1,...,i_k}(p) dx^{i_1} \wedge dx^{i_2} \wedge ... \wedge dx^{i_k}
\end{align*}
Then we define the exterior derivative $d: \Lambda^{k}(M) \rightarrow 
\Lambda^{k+1}(M)$. By
\begin{align*}
    (d\omega)_p = \sum\limits_{1\leq i_1 < ... < i_k \leq n} \sum\limits_{i=1}^n
    \frac{\partial a_{i_1,...,i_k}}{\partial x_i}(p) 
    dx^i \wedge dx^{i_1} \wedge dx^{i_2} \wedge ... \wedge dx^{i_k}
\end{align*}
We remind of the that the derivative of $a_{i_1,...,i_k}$ is meant w.r.t. 
the chart as defined at (\ref{eq:derivative_on_manifold}).
It can be shown that the $d\omega$ is independent of the chosen charts. 

Let us investigate the exterior derivative in the case when 
$M = U \subseteq \real^n$ is an open subdomain. It turns out that by using 
scalar and vector proxies as introduced above we can identify the exterior 
derivative with well-known differential operators. We will use standard 
Euclidian coordinates. 

Let us start with a differentiable function $f: U \rightarrow \real$ i.e. 
$f$ is a $0$-form. Then
\begin{align*}
    (df)^\sharp = \left( \sum_{i=1}^n \frac{\partial f}{\partial x_i} dx^i 
        \right)^\sharp
    = \sum_{i=1}^n \frac{\partial f}{\partial x_i} \frac{\partial}{\partial x_i}
\end{align*}
which we identify with the gradient $\nabla f$. In other words, 
the exterior derivative is just the gradient in vector proxies.

Let $\mathbf{X}$ be a differentiable vector field on $U$ 
with components $\mathbf{X}_i$. This corresponds to the vector field 
$X = \sum_i \mathbf{X}_i \frac{\partial}{\partial x_i}$. Let us say first that 
$X$ is the vector proxy 
of an $(n-1)$-form. The hat symbol used for
$\widehat{dx^i}$ means that this term is left out.
\begin{align*}
    \star^{-1} d\star X^\flat &= \star^{-1} d\star \sum_{i=1}^n \mathbf{X}_i dx^i
    = \star^{-1} d \sum_{i=1}^n \mathbf{X}_i (-1)^{i-1} 
        dx^1 \wedge ... \wedge \widehat{dx^i} 
        \wedge ... 
        \wedge dx^n
    \\ &= \star^{-1} \sum_{i=1}^n \frac{\mathbf{X}_i}{\partial x_i} (-1)^{i-1} 
        dx^i \wedge 
        dx^1 \wedge ... \wedge \widehat{dx^i} \wedge ... \wedge dx^n
    \\ &= \star^{-1} \vol \sum_{i=1}^n \frac{\mathbf{X}_i}{\partial x_i} 
        (-1)^{2(i-1)}
    = \sum_{i=1}^n \frac{\mathbf{X}_i}{\partial x_i}
    = \diver \mathbf{X}.
\end{align*}

In the case $n=3$ if we identify $\mathbf{X}$ with a $1$-form then 
we obtain using similiar computations
\begin{align*}
    (\star d X^\flat)^\sharp = \mathbf{\curl} \mathbf{X}.
\end{align*}
So in $3$D we can identify all the exterior derivatives with known differential
operators and thereby putting them into a more general framework.

A very nice conclusion can be seen directly from the above computations. The 
expression on the left hand side does not use any coordinates. Hence, 
we can use any coordinate system we want and can then compute e.g. the 
divergence in any coordinates we need. Note however that the computations are
more cumbersome when the bases are not orthonormal.

Let us give an application of this fact. This example is not taken from the 
references. We will derive the divergence for arbitrary coordinates. 
So let $\mathbf{X}: U \rightarrow \real^n$ be a usual vector field. By 
identifying this with the vector field $\sum_i X_i \frac{\partial}{\partial x_i}
$ we know computed above that 
\begin{align*}
    \diver \mathbf{X} \vol = d \star X^\flat.
\end{align*}
Now let us express $\mathbf{X}$ using different coordinates. 
Let $\phi: U \rightarrow V$ be a diffeomorphism. Then we can write 
\begin{align*}
    \mathbf{X} = \sum_{j=1}^n \tilde{X}_j 
        \frac{\partial \phi^{-1}}{\partial y_j}.
\end{align*}
In terms of differential geometry, we get representation of the vector field 
$X = \sum_j \tilde{X}_j \frac{\partial}{\partial y_j}$. 
Let $\tilde{\mathbf{X}} = (\tilde{X}_1, \tilde{X}_2, ..., \tilde{X}_n)^\top$ 
and 
\begin{align*}
    G_{kl} = g(\frac{\partial}{\partial y_k}, \frac{\partial}{\partial y_l})
        = \frac{\partial \phi^{-1}}{\partial y_k} \cdot 
            \frac{\partial \phi^{-1}}{\partial y_l}
\end{align*}
Then we compute 
\begin{align*}
    (\diver \mathbf{X}) \vol 
    &= d \star X^\flat 
    = d \star \sum_{j=1}^n (G \tilde{\mathbf{X}})_j dy^j 
    \\ &= d \sum_{j=1}^n \sum_{k=1}^n (G \tilde{X})_j 
        \sqrt{ |\det G| } g^{jk} (-1)^{k} dy^1 \wedge dy^2 \wedge ... \wedge 
        \widehat{dy^k} \wedge ... \wedge dy^n 
    \\ &= \sum_{k=1}^n \frac{\partial
        (\sqrt{ |\det G|} \sum_{j=1}^n g^{jk} (G \tilde{X})_j )}
        {\partial y_k} (-1)^k dy^k \wedge dy^1 \wedge dy^2 \wedge ... \wedge 
        \widehat{dy^k} \wedge ... \wedge dy^n 
    \\ &= \sum_{k=1}^n \frac{\partial (\sqrt{ |\det G|}  \tilde{\mathbf{X}})_k }
        {\partial y_k} (-1)^{2k} dy^1 \wedge dy^2 \wedge ... \wedge 
        \widehat{dy^k} \wedge ... \wedge dy^n
    \\ &= \left[ \frac{1}{\sqrt{ |\det G|}} \sum_{k=1}^n 
        \frac{\partial (\sqrt{ |\det G|}  \tilde{\mathbf{X}})_k }{\partial y_k}
        \right] \vol
\end{align*}
and we find the well-known expression for the divergence in general coordinates
(cf. e.g. \cite{})
\begin{align*}
    \diver \mathbf{X} = \frac{1}{\sqrt{ |\det G|}} \sum_{k=1}^n 
        \frac{\partial (\sqrt{ |\det G|} \, \tilde{X}_k )}{\partial y_k}.
\end{align*}
% TBD: Some stuff I used here was not introduced before


Let us mention some important properties of the exterior derivative. 
At first, we have $d\circ d = 0$ which will be important later on 
when we talk about cochain complexes in Section \ref{}. This corresponds in 
$3D$ to $\diver \curl= 0$ and $\curl \grad = 0$.

The relation to the wedge product is described by a Leibniz formula. 
Let $\omega \in C^1 \Lambda^k M$ and $\nu \in C^1 \Lambda^l M$. Then
\begin{align}
    d (\omega \wedge \nu) = d\omega \wedge \nu + (-1)^l \omega \wedge d\nu.
    \label{eq:leibniz_formula}
\end{align}

The second property is that the exterior derivative commutes with pullback i.e. 
for manifolds $M$ and $N$ a differentiable mapping $F:M \rightarrow N$ 
and $\omega \in \Lambda^k N$ we have $dF^* \omega = F^* d\omega$. 
In terms of proxies this is related to very interesting results. 

Let $F: \widehat{\Omega} \rightarrow \Omega$ be a diffeomorphism with 
$\widehat{\Omega}, \Omega \in \real^n$. Let us again consider Euclidian 
coordinates on the domain and codomain. Then observe
\begin{align*}
    & (\diver \mathbf{X})(F(\hat{x}))  \det DF(\hat{x}) \widehat{\vol}_{\hat{x}} 
    = ( F^* (\diver \mathbf{X}))_{\hat{x}} \wedge (F^* \vol)_{\hat{x}}
    \\ &= \left( F^* \big( (\diver \mathbf{X}) \vol \big) \right)_{\hat{x}}
    \\ &= \left( F^* d \star X^\flat  \right)_{\hat{x}}
    = \left( d F^* \star X^\flat  \right)_{\hat{x}}.
\end{align*}
and then 
\begin{align*}
    d F^* \star X^\flat = d \star \star^{-1} F^* \star X\flat 
    = d \star \hat{X}^\flat = \widehat{\diver} \mathbf{\widehat{X}}
\end{align*}
by defining 
\begin{align*}
    ( \star^{-1} F^* \star X^\flat)^\sharp 
    =\vcentcolon \hat{X} = \sum_i \mathbf{\hat{X}}_i(\hat{x})  
     \frac{\partial}{\partial \hat{x_i}}
\end{align*}
which we identify with the vector field 
$\mathbf{\widehat{X}}(\hat{x}) = (\widehat{X}_1(\hat{x}), \widehat{X}_2(\hat{x})
, ..., \widehat{X}_n(\hat{x}) )^\top$.
We then know from Sec.\,\ref{sec:alternating_maps} and because we assume
that we use orthonormal coordinates that 
\begin{align*}
    \hat{X}(\hat{x}) 
    &= ( \star^{-1} F^* \star X\flat)^\sharp (\hat{x})
    = \sum_{i,j=1}^n \text{ad}(DF(\hat{x}))_{ij} X_j(F(x)) 
        \frac{\partial}{\partial \hat{x}_i}
    \\ &= \sum_{i,j=1}^n \det DF(\hat{x}) \big( DF(\hat{x})^{-1}\big)_{ij} X_j(F(x)) 
        \frac{\partial}{\partial \hat{x}_i}
\end{align*}
and so we identify this with the vector field
\begin{align*}
    \mathbf{\widehat{X}} = \det DF(\hat{x}) DF(\hat{x})^{-1} \mathbf{X}.
\end{align*}
and recognize the widely known Piola transformation (cf. \cite{Ern, Guermond}).

If the manifold is oriented and we have thus a Hodge star operator.
Then we define the \textit{codifferential operator}
$\delta \vcentcolon= (-1)^{n(k-1)+1} \star d \star$ which is then 
an operator $\Lambda^{k}(M) \rightarrow 
\Lambda^{k-1}(M)$.


The exterior derivative and the codifferential both require the differential 
form to be differentiable. Later we will extend this in weak sense so 
classical differentiabilty is no longer required (see \ref{}).


\subsection{Sobolev spaces of differential forms}

In order to rigorously define Sobolev spaces we have to define $L^p$-spaces 
of differential forms first. But before we can do that we should have a look
how integration of a function can be defined on a smooth orientable 
Riemannian manifold $M$. 

We want to define integration in the framework of usual measure and integration
theory which means defining it as a Lebesgue integral w.r.t. a measure on $M$
which we have to define first along with a $\sigma$-algebra on $M$. 

It is well known that
the borel $\sigma$-algebra $\mathcal{B}$ on $\real^n$ is generated
by all open sets. This idea can be applied to any topological space $X$ by 
defining the borel $\sigma$-algebra $\mathcal{B}(X)$ as the $\sigma$-algebra 
generated by all open sets. So we can simply use the topology on $M$ to define 
our $\sigma$-algebra $\mathcal{B}(M)$. Because we know that all our 
charts $\phi_i: U_i \rightarrow \real^n$ are homeomorphisms it is very 
straightforward to show that a set $E \in \mathcal{B}(M)$ i.i.f.
$\phi_i(E \cap U_i)$ is Borel-measurable. 

Now, we need to define a measure on the manifold. This measure will 
also involve the charts and it requires the manifold to be Riemannian 
and orientable. Let $\{ \chi_i \}_{i=1}^\infty$ be partition of unity 
subordinate to the $U_i$. {\color{red} I need stronger 
requirements for the manifold. We need a countable basis. Or does any 
Riemannian orientable manifold have that?}  Then we define the 
\textit{Riemannian measure} for any $E \in \mathcal{B}(M)$
\begin{align*}
    V(E) \vcentcolon= \sum\limits_{i=1}^\infty \int_{\phi_i(U_i) \cap E}
        \chi_i(\phi_i^{-1}(x)) \sqrt{\det G^{(i)}(\phi_i^{-1}(x))} dx 
        \in [0,\infty]
\end{align*}
It can be shown that it is independent of the chosen oriented atlas 
using the transformation behaviour of $G^{(i)}$. But the orientation is 
crucial for it to be well-defined. 

Now that we have the measure space $(M, \, \mathcal{B}(M), \, V)$
we can define integration in the usual Lebesgue way.
It is easily shown that a function $f: M \rightarrow \real$ is measurable 
i.i.f. $f \circ \phi_i^{-1}$ is measurable for every chart $\phi_i$. 
It is then simply an application of the definition of Lebesgue integration to
show that for any measurable $f \geq 0$ we can express the integration as
\begin{align*}
    \int_M f \, dV = \sum\limits_{i=1}^\infty \int_{\phi_i(U_i)} 
        \chi_i(\phi_i^{-1}(x)) f(\phi_i^{-1}(x)) 
        \sqrt{\det G^{(i)}(\phi_i^{-1}(x))} dx.
\end{align*}
By introducing the integral as a Lebesgue integral w.r.t. the Riemannian 
measure we inherit the theoretical framework of Lebesgue integration. 
For example, we know that the the spaces $L^p(M,V)$ for $1\leq p < \infty$
, i.e. the $p$-integrable 
real-valued functions w.r.t. the Riemannian measure, are Banach spaces.

\subsubsection{Integration of differential forms}\label{sec:integration_differential_forms}

Once again, we should first ask ourselfs what a measurable differential form 
should be. We know that we can express our differential form locally 
for $p \in U_i$ using the local coordinates as 
\begin{align*}
    \omega_p = \sum\limits_{1\leq i_1 < ... < i_k \leq n} 
        a_{i_1,...,i_k}(p) dx^{i_1} \wedge dx^{i_2} \wedge ... \wedge dx^{i_k}.
\end{align*}
In the spirit of the above steps we call a $\omega \in \Lambda^k (M)$
measurable if for every chart $\phi_i$ the coefficent functions 
are measurable. This is once again independent of the chosen atlas. 
This can be proven by identifying $\omega$ as a section of the vector bundle 
of $k$-linear alternating maps over $M$. We did not define vector bundles
however and will skip the proof.
For now assume our manifold to be smooth and orientable, but not necessarily
Riemannian.

Next, we will define integration of an $n$-form over an $n$ dimensional 
manifold. At first, we do so for an open set $U \subseteq \real^n$.
This is the simplest example of an $n$-dimensional manifold where 
we only have one chart which is the identify and the local coordinates are 
just our standard coordinates. Let $\omega$ be a measurable 
$n$-form on $U$ so we can 
write 
\begin{align*}
    \omega_x = f(x) dx_1 \wedge dx_2 \wedge ... \wedge dx_n
\end{align*}
for $x \in U$ with $f:U \rightarrow \real$ being measurable. 
We can now simply define 
\begin{align*}
    \int_U \omega = \int_U f(x) dx.
\end{align*}

With this definition at hand we can now extend this definition to 
any smooth oriented $n$-dimensional manifold $M$. As it is often done in 
differential geometry we will work locally first and then extend this 
construction globally by using a partition of unity.

Let $(U,\phi)$ be a chart on $M$ and assume $\supp \omega \subseteq U$. 
Then $(\phi^{-1})^* \omega$ {\color{red} Define the pullback}
is a $n$-form on $\phi(U) \subseteq \real^n$ and 
we can apply our prior definition. So now we just define 
\begin{align*}
    \int_M \omega \vcentcolon= \int_{\phi(U)} (\phi^{-1})^* \omega.
\end{align*}
Once again, it can be shown that this definition does not depend on the chart if 
we choose the atlas corresponding to the orientation of the manifold. 

Now let us move on to the global definition. Let $\{(U_i,\phi_i)\}_{i=1}^\infty$
be an oriented atlas and let $\{ \chi_i \}_{i=1}^\infty$ be a partition 
of unity subordinate to it. 
Then $\supp \chi_i \omega \subseteq U_i$ 
and we define 
\begin{align*}
    \int_M \omega \vcentcolon= \sum_{i=1}^\infty \int_M \chi_i \omega.
\end{align*} 
This definition is also independent of the chosen chart and partition of 
unity. We will omit the proof. 

We will now have a look how the integration of functions and of 
differential forms are related to each other. 
We know that for $p \in U_i$ we can write the volume form as
\begin{align*}
    \vol_p = \sqrt{ \det G^{(i)}(p)} 
        dx_{(i)}^1 \wedge dx_{(i)}^2 \wedge ... \wedge dx_{(i)}^n.
\end{align*}
Because $(\phi_i^{-1})^* dx_{(i)}^k = dx^k$ i.e. the standard dual basis
in $\real^n$ we have
\begin{align*}
    \big( (\phi_i^{-1})^* \vol)_x =  
        \sqrt{ \det G^{(i)}(\phi_i^{-1}(x))} 
        dx^1 \wedge dx^2 \wedge ... \wedge dx^n.
\end{align*}
That means for a $n$-form $f \vol$ we can write the integral as
\begin{align*}
    \int_M f \vol = \sum_{i=1}^\infty \int_{\phi_i(U_i)} 
    \chi_i(\phi_i^{-1}(x)) f(\phi_i^{-1}(x))
        \sqrt{ \det G^{(i)}(\phi_i^{-1}(x))} dx^1 \wedge ... \wedge dx^n
\end{align*} 
and we see 
\begin{align*}
    \int_M f dV = \int_M f \vol
\end{align*}
with the two different notions of integration. So we see that the
the two definitions are essentially equivalent.
The big advantage of considering these two
approaches is that we know the integration of 
differential forms is within the framework of Lebesgue integration. 
It is then also 
clear how integrability for $n$-forms should be defined. We call an 
$n$-form $f \vol$ integrable if $f$ is integrable w.r.t. the Riemannian measure.

\subsubsection{Stokes' theorem and integration by parts} 

One of the most important results about the integration of differential forms
is Stokes' theorem which we will state in this section. From it, we will 
obtain a integration by parts formula.

But before we do so we have to check how to define the restriction of a
differential form to a submanifold $N \subseteq M$. 
We have the inclusion $\iota: N \hookrightarrow M$. Then for a 
smooth differential form $\omega \in C^\infty \Lambda^k (M)$ we define 
the restriction just via the pullback of the inclusion operator i.e. 
$\iota^* \omega \in C^\infty \Lambda^k(N)$. 

For a $k$-dimensional submanifold $N$ we then denote the integration of an 
$k$-form $\omega$ as
\begin{align*}
    \int_N \iota^*\omega = \int_N \omega.
\end{align*} 

\begin{theorem}[Stokes]
    Let $M$ be a smooth oriented manifold with boundary 
    $\partial M$. Let $\omega$ be 
    a smooth compactly supported $(n-1)$-form. Then we have 
    \begin{align*}
        \int_M d\omega = \int_{\partial M} \omega.
    \end{align*}
\end{theorem}
This theorem gives us a relation of the boundary and the exterior derivative 
which will be crucial in the topological context of differential forms 
which we will investigate in Sec.\,\ref{sec:de_rhams_theorem}. 

We can also derive a form of the integration by parts formula from it. 
Let $\omega \in C^1 \Lambda^k (M)$ and $\mu \in C^1 \Lambda^{n-k-1}(M)$. 
Then $\omega \wedge \mu \in C^1 \Lambda^{n-1} (M)$. Recall the 
Leibniz rule for the exterior derivative 
\begin{align*}
    d(\omega \wedge \mu) = d\omega \wedge \mu + (-1)^k \omega \wedge d\mu. 
\end{align*}
By integrating both sides over $M$ and applying Stokes' theorem we obtain
\begin{align*}
    \int_{\partial M} \omega \wedge \mu 
    = \int_M d\omega \wedge \mu + (-1)^k \int_M \omega \wedge d\mu.
\end{align*}

The integration by parts motivates the definition of another differential
operator. Let $\omega \in C^1 \Lambda^k (M)$ be integrable. 
We define the \textit{codifferential operator} or simply 
codifferential 
\begin{align*}
    \delta \omega \vcentcolon= 
    (-1)^{n(k-1)+1} \star d \star \omega \in C^0 \Lambda^{k-1} M.
\end{align*}
By using the Leibniz rule and \ref{} we compute for $\omega \in 
C^1 \Lambda^k (M), \nu \in C^1 \Lambda^{k+1} (M)$
\begin{align*}
    d(\omega \wedge \star \nu) 
    &= d\omega \wedge \star \nu + 
        (-1)^k \omega \wedge d \star \nu 
    \\ &= d\omega \wedge \star \nu + (-1)^k (-1)^{(n-k)k} 
        \omega \wedge \star \star d \star \nu 
    \\ &= d\omega \wedge \star \nu + (-1)^k (-1)^{(n-k)k} (-1)^{nk + 1}
        \omega \wedge \star \delta \nu 
    \\ &= d\omega \wedge \star \nu -
        \omega \wedge \star \delta \nu.
\end{align*}
In the last step, we used 
\begin{align*}
    (-1)^k (-1)^{(n-k)k} (-1)^{nk + 1} 
    = (-1)^{2kn - k(k-1)+1} = -1
\end{align*}
because $2kn - k(k-1)$ is always even. 
Now we once again integrate on both sides assuming $\omega$ and $\nu$ are 
integrable
\begin{align*}
    \int_M \langle d\omega_p , \nu_p \rangle_{\alternating{k+1}{T_p M}} \,\vol_p
    &= \int_M d\omega \wedge \star \nu 
    = \int_M \omega \wedge \star \delta \nu  
        + \int_{\partial M} \omega \wedge \star \nu
    \\ &= \int_M \langle \omega_p , \delta \nu_p \rangle
        _{\alternating{k}{T_p M}} \, \vol
        + \int_{\partial M} \langle \omega , \delta \nu 
        \rangle_{\alternating{k}{T_p \partial M}} \, \vol_{\partial M}.
\end{align*}
Some clarification on the notation used. In the first integral we integrate 
the $n$-form $p \mapsto 
\langle d\omega_p , \nu_p \rangle_{\alternating{k+1}{T_p M}} \, \vol_p$
with the inner product of alternating maps introduced in 
Sec.\,\ref{sec:alternating_maps} and analogous in the last line. 
From now on, we will leave out the reference to the space of the inner product
when it is clear from the context.

This integration by parts will be used in the next section to extend the 
exterior derivative in the weak sense in analogy to the usual introduction 
of the weak derivative.

\subsubsection{Sobolev spaces of differential forms}

So far, we used $p$ to refer to a point on the manifold and to emphasize 
the fact that it is not in $\real^n$. However, this causes a collision of 
notation when we talk about $L^p$-spaces. So from now on, $x$ will denote 
a point on a manifold $M$.

We define the $L_p$-norm of a measurable $k$-form $\omega$ for 
$1\leq p < \infty$
as (cf. \cite{goldshtein})
\begin{align*}
\lVert \omega \rVert _{L_p^k(M)}\vcentcolon=
\left(\int_M \lVert \omega_x \rVert _{\text{Alt}^k T_x M}^p 
    \,\vol_x \right)^{1/p}.
\end{align*}
Let us briefly argue, why $\lVert \omega_x \rVert _{\text{Alt}^k T_x M}$ 
is measurable. Since $\omega$ is measurable we have 
\begin{align*}
    \omega_x = \sum_{1 \leq i_1 < ... < i_k \leq n}
        a_{i_1...i_k}(x) dx^{i_1} \wedge ... \wedge dx^{i_k}.
\end{align*}
with $a_{i_1...i_k}: M \rightarrow \real$ measurable. Then the expression 
for the norm is
\begin{align*}
    &\lVert \omega_x \rVert^2 _{\text{Alt}^k T_x M}
    \\ &= \langle \sum_{1 \leq i_1 < ... < i_k \leq n} 
        a_{i_1...i_k}(x) dx^{i_1} \wedge ... \wedge dx^{i_k},
        \sum_{1 \leq j_1 < ... < j_k \leq n} 
        a_{j_1...j_k}(x) dx^{j_1} \wedge ... \wedge dx^{j_k} \rangle
        _{\text{Alt}^k T_x M}
    \\ &= \sum_{\substack{1 \leq i_1 < ... < i_k \leq n \\ 
            1 \leq j_1 < ... < j_k \leq n}}
        a_{i_1...i_k}(x)\,a_{j_1...j_k}(x) 
        \langle  dx^{i_1} \wedge ... \wedge dx^{i_k},
         dx^{j_1} \wedge ... \wedge dx^{j_k} \rangle _{\text{Alt}^k T_x M}
    \\ &= \sum_{\substack{1 \leq i_1 < ... < i_k \leq n \\ 
        1 \leq j_1 < ... < j_k \leq n}}
        a_{i_1...i_k}(x)\,a_{j_1...j_k}(x) 
        \det \big(\langle dx^{i_s},dx^{i_t}\rangle_{T^*_x M})_{1\leq s,t \leq k}.
\end{align*}
Now $\langle dx^i, dx^j\rangle _{T^*_x M} = g^{ij} = \big( G^{-1} \big)_{ij}$
is measurable because we assume the Riemannian metric to be at least 
differentiable. So the pointwise norm is indeed a measurable function 
on $M$.

%%% TBD: Is this independent of coordinates i.e. does this notation make sense?
$L_p^k(M)$ is the spaces of measurable $k$-forms 
s.t. the corresponding $L_p$-norm is finite.
For $p=2$ we obtain a Hilbert space (cf. \cite[Sec. 6.2.6]{arnold}) 
with the $L_2$ inner product  
\begin{align}
    \langle \omega, \nu \rangle_{L_2^k(M)} \vcentcolon= 
    \int_M \langle \omega_x, \nu_x \rangle _{\text{Alt}^k T_x M} \,dx
    = \int_\Omega \omega \wedge \star \nu
    \label{eq:def_inner_product} 
\end{align}
We will leave out the reference to the $L_2$-space at the inner product from now 
on.

\begin{proposition}
    The Hodge star operator $\star:L^k_2(\Omega) \rightarrow L^{n-k}_2(\Omega)$ is a
    Hilbert space isometry.
\end{proposition}
\begin{proof}
    This follows directly from the definition of the inner product 
    (\refeq{eq:def_inner_product}) and the fact that $\star$ is an isometry 
    when applied to alternating forms $\text{Alt}^k T_x M$.
\end{proof}


Our next goal is to extend the exterior derivative $d$ 
of smooth differential forms in the weak sense (cf. \cite{goldshtein}). 
Let $\mathring{d}: L^k_2(\Omega) \rightarrow L^{k+1}_2(\Omega)$ be the exterior
derivative as an unbounded operator with domain 
$D(\mathring{d}) = \smoothcompforms{k}{\Omega}$ 
which are the smooth compactly supported differential forms $\varphi$ 
of degree $k$
with $\supp \varphi \subseteq \interior M$.

Note that when we talk about smoothness or regularity of differential forms we
always mean the regularity of the coefficents when the form is expressed 
via local charts. 

Analogous, let 
$\mathring{\delta}: L^k_2(\Omega) \rightarrow L^{k-1}_2(\Omega)$ be the 
codifferential operator $\mathring{\delta} \vcentcolon= 
(-1)^{n(k-1)+1}\star\mathring{d}\star$ 
also with domain $\smoothcompforms{k}{\Omega}$. 

Then the exterior derivative $ d\omega \in L^{k+1}_p(\Omega)$ is defined as
the unique $(k+1)$-form in $L^{k+1}_p(\Omega)$ s.t. 
\begin{align*}
\int_\Omega d\omega \wedge \star\phi = \int_\Omega \omega \wedge 
\star\mathring{\delta}\phi
\quad \forall \phi \in C_0^\infty \Lambda^{k}(\Omega).
\end{align*}
Just as in the usual Sobolev setting we define the following spaces:
\begin{align*}
W^k_p(\Omega) &= \left\{ \omega \in L^k_p(\Omega) \mid 
     d\omega \in L_p^{k+1}(\Omega) \right\}, \\ 
W^k_{p,loc}(\Omega) &= \left\{ \omega \ k \text{-form} \mid 
\omega|_A \in W^k_p(A) \text{ for every open } A \subseteq \Omega 
\text{ s.t. } \overline{A} \subseteq \Omega \text{ is compact} %TBD: Check that this is right
\right\}.
\end{align*}
For $\omega \in W^k_p(\Omega)$ for $p<\infty$ we define the norm 
\begin{align*}
\lVert \omega \rVert _{W^k_p(\Omega)} &\vcentcolon= 
\left( \norm{\omega}{L^k_p(\Omega)}^p 
    + \norm{d\omega}{L^k_p(\Omega)}^p \right)^{1/p}.
\end{align*}

\begin{remark} \label{rem:identification_sobolev_spaces}
    Throughout this thesis, we will mostly deal with open subdomains $\Omega 
    \subseteq \real^n$ with Lipschitz boundary. Then $\Omega$
    is a smooth submanifold of $\real^n$ and $\omegabar$ is a 
    Lipschitz manifold with boundary. If we now assume that
    $\Omega = \interior \omegabar$ then $W^k_p(\Omega)$ and $W^k_p(\omegabar)$
    are essentially the same. Take $\omega \in W^k_p(\Omega)$ and extend 
    it arbitrarily to $\overline{\omega} \in W^k_p(\omegabar)$. Then because 
    $\partial \Omega$ is a null set $\overline{\omega} \in L_2^k(\omegabar)$ and 
    because the definition of the exterior derivative uses only smooth 
    funtions with compact support contained in $\interior \omegabar = \Omega$ 
    we get that $d\overline{\omega} = d\omega \in L^k_2(\omegabar)$ (again 
    by choosing arbitrary values on the boundary). From now on we will in the
    assumed setting treat the spaces $W^k_p(\omegabar)$ and $W^k_p(\Omega)$ 
    as the same.
\end{remark}
    
\begin{definition}[$L^p$-cohomology]
    We define the following subspaces of $W^k_p(\Omega)$, $1\leq p \leq\infty$:
    \begin{align*}
        \mathfrak{B}_k &\vcentcolon= dW^{k-1}_p(\Omega) \text{ and} \\
        \mathfrak{Z}_k &\vcentcolon= \{ \omega \in W^k_p(\Omega)| 
        \, d\omega = 0\}.
    \end{align*}
    We call the $k$-forms in $\mathfrak{B}_k$ exact and the forms in 
    $\mathfrak{Z}_k$ closed. Because $d \circ d=0$ we always have  
    $\mathfrak{B}_k \subseteq \mathfrak{Z}_k$.
    Then we define the de Rham- or $L^p$-cohomology space $\lpcoho(\Omega)$ as 
    the quotient space
    \begin{align*}
        \lpcoho (\Omega) \vcentcolon= \faktor{\mathfrak{Z}_k}{\mathfrak{B}_k}.
    \end{align*}
\end{definition}
\vspace{0.5cm}
We want to examine the Hilbert space $L^k_2(\Omega)$ more closely
(see \cite[Sec. 6.2.6]{arnold} for more details).  
%%% TBD: This is only for bounded domains. What changes?
% Picard defines it as completion of smooth functions \cite[p.37]{picard}
We denote $H^k(d;\Omega) \vcentcolon= W^k_2(\Omega)$. If the domain is clear
we will leave it out. Note that the above definition of the exterior derivative
is in the Hilbert space setting equivalent to defining $d$ as the adjoint
of $\mathring{\delta}$. 



In order to extend $\mathring{\delta}$ as well, we will need the following

\begin{definition}[Codifferential operator]
    Analogous to the smooth case, 
    we define the \textit{codifferential operator} for any $k$ as an unbounded
    operator $\delta: L^k_2(\Omega) \rightarrow L^{k-1}_2$ as
    \begin{align*}
        \delta \vcentcolon= (-1)^{n(k-1)+1}\star d\,\star
    \end{align*}
    with domain
    \begin{align*}
        D(\delta) = \{ \omega \in L^k_2(\Omega) | \,
        \star\omega \in H^{n-k}(d) \} 
        =\vcentcolon H^k(\delta; \Omega).
    \end{align*}
\end{definition}

\begin{proposition}
    $\delta = \mathring{d}^*$ i.e. $\delta$ is the adjoint of $\mathring{d}$.
\end{proposition}
\begin{proof}
    Denote with
    $D(\mathring{d}^*) \subseteq L_2^{k-1}(\Omega)$ the domain of the adjoint.
    Now take $\omega \in H^k(\delta)$ and $\phi \in 
    \smoothcompforms{k}{\Omega}$.
    Then 
    \begin{align*}
        &\langle \delta \omega, \phi \rangle = 
        (-1)^{nk+1} \langle \star d\star \omega, \phi \rangle \\  
        &= (-1)^{nk+1} (-1)^{k(n-k)} \langle d\star \omega, \star\phi \rangle =
        (-1)^{nk+1} (-1)^{k(n-k)} 
            \langle \star \omega, \mathring{\delta}\star\phi \rangle \\
        &= (-1)^{nk+1} (-1)^{k(n-k)} (-1)^{n(n-k-1)+1}
            \langle \star \omega, \star\mathring{d}\star\star\phi \rangle \\
        &=(-1)^{n(n-1)+2} (-1)^{k(n-k)} \langle \omega, 
            \mathring{d}\star\star\phi \rangle\\
        &= \langle \omega, \mathring{d}\phi \rangle
    \end{align*}
    where we used repeatedly that $\star$ is an isometry and 
    $\star\star = (-1)^{k(n-k)}\text{Id}$. 
    This shows that $H^{k+1}(\delta) \subseteq 
    D(\mathring{d}^*)$ and that $\mathring{d}^* \omega = \delta \omega$. Now for 
    the other inclusion assume that $\omega \in D(\mathring{d}^*)$ and take 
    $\phi \in \smoothcompforms{n-k}{\Omega}$ arbitrary.
    \begin{align*}
        \langle \star\omega, \mathring{\delta}\phi \rangle = \pm 
        \langle \omega, \mathring{d}\star\phi \rangle = 
        \pm \langle \mathring{d}^* \omega, \star\phi \rangle = 
        \pm \langle \star\mathring{d}^* \omega, \phi \rangle.
    \end{align*} 
    Here we use $\pm$ to mean that we choose the sign correctly, s.t. all the
    operations are correct. Then by choosing the sign appropriately we find that
    $ \pm \star\mathring{d}^* \omega = d\star\omega$ and therefore 
    $\star\omega \in H^{n-k-1}(d)$ so we proved $D(\mathring{d}^*) \subseteq 
    H^{k+1}(\delta)$ and we are done.    
\end{proof}

In order to deal with the boundary of our domain we introduce Homogeneous
boundary conditions for these Sobolev spaces of differential forms.

\begin{definition}[Zero boundary condition] \label{def:zero_boundary_condition}
    We say that $\omega \in H^k(d;\Omega)$ has zero boundary condition if
    \begin{align*}
        \langle d\omega,\chi \rangle _{L^{k+1}_2(\Omega)}
        =  \langle \omega,\delta\chi \rangle _{L^k_2(\Omega)}
        \quad \forall \chi \in H^{k+1}(\delta;\Omega). 
    \end{align*}
    Denote $\mathring{H}^k(d;\Omega) \vcentcolon= \{ \omega \in H^k(d;\Omega)|\,
    \omega \text{ has zero boundary condition} \}$. 
\end{definition}
Of course we should justify why this is a reasonable definition. If $\Omega$ 
is lipschitz and bounded we have the integration by parts formula 
(cf. \cite[Thm.~6.3]{arnold})
\begin{align*}
    \int_\Omega d\omega \wedge \mu 
    = (-1)^k \int_\Omega \omega \wedge d \mu  + \int_{\partial\Omega} 
    \text{tr}\,\mu \wedge \text{tr}\,\omega \quad \text{for } \omega \in 
    H^1\Lambda^k(\Omega),\ \mu \in H^{n-k-1}(d;\Omega)  
\end{align*} 
where $H^1\Lambda^k(\Omega)$ are the differential forms with all coefficients
being in $H^1(\Omega)$ (here we mean just the standard Sobolev space). 
Let now $\omega \in \mathring{H}^k(d)$. Then if we use
the integration by parts formula and 
$\langle d\omega,\mu \rangle_{L^{k+1}(\Omega)} = 
\langle \omega, \delta \mu \rangle_{L^k(\Omega)}$ we get 
after some computation using the Hodge star
\begin{align*}
    \langle \text{tr}\,\omega, 
        \star\,\text{tr}\,\star\mu \rangle_{L^k(\Omega)}
    = 0 \quad \forall \mu \in H^1\Lambda^k(\Omega).
\end{align*}
The trace operator $\text{tr}: H^1\Lambda^k(\Omega) 
\rightarrow H^{1/2}\Lambda^k(\Omega)$ is surjective \cite[Thm.~6.1]{arnold}.
\begin{align*}
    \star\text{tr}\star H^1\Lambda^{k+1}(\Omega) 
    = \star \text{tr} H^1\Lambda^{n-k-1}(\Omega)
    = \star H^{1/2}\Lambda^{n-k-1}(\Omega)
    = H^{1/2}\Lambda^k(\Omega)
\end{align*}
is dense in $L^k_2(\Omega)$. Thus $\text{tr}\,\omega = 0$. So in the case 
of bounded Lipschitz domains this definition is reasonable. The reason why we
chose to define it as in Def.\,\ref{def:zero_boundary_condition} is that 
is easily extendible to unbounded domains and the regularity of the boundary
is not an issue. 


Then we define the spaces 
\begin{align*}
    H^k_0(d;\Omega) &\vcentcolon= \{ \omega \in H^k(d;\Omega) 
    | d\omega = 0 \} \\
    \mathring{H}^k_0(d;\Omega) &\vcentcolon= \{ \omega \in \mathring{H}^k(d;\Omega) 
    | d\omega = 0 \}
\end{align*}
i.e. the spaces of closed forms. We will use the analogous definition for 
$H^k_0(\delta;\Omega)$ and $\mathring{H}^k_0(\delta;\Omega)$ which we call 
coclosed forms. We then define the spaces of harmonic forms
\begin{align*}
    \mathring{H}^k_0(d,\delta;\Omega) \vcentcolon= 
    \{ \omega \in \mathring{H}^k(d;\Omega) 
    | \, d\omega = 0, \delta\omega = 0 \}.
\end{align*}
With this one can prove the Hodge decomposition (\cite[Lemma 1]{arnold})
\begin{align}
    L_2^k(\Omega) = \overline{d\mathring{H}^{k-1}(d)} \stackrel{\perp}{\oplus} 
    \mathring{H}^k_0(d,\delta) \stackrel{\perp}{\oplus} 
    \overline{\delta H^{k+1}(\delta)} \label{hodge_decomposition}
\end{align}
and furthermore for the closed and coclosed forms respectively,
\begin{align}
    \mathring{H}^k_0(d) &= \overline{d\mathring{H}^{k-1}(d)} 
    \stackrel{\perp}{\oplus}
    \mathring{H}^k_0(d,\delta) \label{decomposition_closed_forms} \\
    H^k_0(\delta) &= \overline{\delta H^{k+1}(\delta)} \stackrel{\perp}{\oplus}
    \mathring{H}^k_0(d,\delta). \label{decomposition_coclosed_forms}
\end{align}

\section{Singular homology}

The curve integral constraint from the magnetostatic problem is very topological % TBD: Reference
in nature and strongly related to the topology of the domain. 
In order to deal with this constraint and obtain the desired existence and
uniqueness we require some tools from algebraic topology which we will introduce 
in this section. 
This material is taken from
\cite{topology_and_geometry} where a lot more details and results can be found.

\subsection{Homology groups}

Denote with $\real^\infty$ the vector space of all real-valued sequences. Let 
$e_i \in \real^\infty$ for $i \in \naturalnum$ denote the sequences that 
that are zero for every index 
unequal to $i$ and $1$ for the index $i$. Note that in this thesis 
the natural numbers start at zero. Then we define the standard $k$-simplex 
$\Delta_k$ as
\begin{align*}
    \Delta_k \vcentcolon= \Big\{ \sum\limits_{i=1}^k  \lambda_i e_i \mid 
    \sum\limits_{i=0}^k \lambda_i = 1 , \; 0\leq \lambda_i \leq 1\Big\}
    = \text{conv} \{e_0,\,...,\, e_k\}.
\end{align*}
where $\text{conv}$ is the usual convex combination. 

\begin{definition}[$k$-simplex]
Let $X$ be a topological space. Then a \textit{singular $k$-simplex} is a continuous 
map $\sigma_k: \Delta_k \rightarrow X$. We will frequently leave out the term 'singular'
and refer to them just as $k$-simplices.
\end{definition}

As the term 'singular' implies these simplices can be degenerated. For example, $\sigma_k$ could 
just be constant for any $k$, so the object in the topological space corresponding to the 
$k$-simplex is just a point.

We can now introduce a algebraic structure by looking at finite formal sums of the form 
\begin{align*}
    \sum_{\text{$\sigma$ $k$-simplex }} n_\sigma \sigma.
\end{align*}
These formal sums form an abelian group which we refer to as the 
\textit{singular $k$-chain group} $C_k(X)$. 

We will now introduce an important homomorphism between these groups called the \textit{boundary}.
\begin{definition}
    Let $v_0, ... , v_k \in \real^n$. 
    We define \textit{affine singular $k$-simplex} as a special singular $k$-simplex denoted by
    \begin{align*}
        [v_0,...,v_k]: \Delta_k \rightarrow \real^n, \, 
        \sum\limits_{i = 0}^k \lambda_i e_i \mapsto \sum\limits_{i = 0}^k \lambda_i v_i.
    \end{align*}
\end{definition}
As in the general case, the image can be a degenerated simplex in $\real^n$ since the 
$v_i$ are not assumed to be affine independent. 

We call the affine singular simplex 
$[e_0,...,\hat{e}_i,...,e_k]: \Delta_{k-1} \rightarrow \Delta_k$ the $i$-th face map. 
The $\hat{ }$ means this vertex is left out. Here we tacitly used the 
natural inclusion $\real^{k+1} \subseteq \real^\infty$ so we have 
$\Delta_k \subseteq \real^{k+1}$. But this is just a way of representation.

With the face map we can now define the boundary operator.
\begin{definition}
    For a singular $k$-simplex $\sigma: \Delta_k \rightarrow X$ we define its $i$-th face 
    $\sigma^{(i)} \vcentcolon= \sigma \circ F_i^k$ which is a $(k-1)$-simplex. 
    We then define the \textit{boundary} of $\sigma$ as 
    $\partial_k \sigma \vcentcolon= \sum_{i=0}^k (-1)^i \sigma^{(i)}$. We extend this 
    to a homomorphism between the chain groups
    \begin{align*}
        \partial_k: C_k(X) \rightarrow C_{k-1}(X), 
        \sum_\sigma n_\sigma \sigma \mapsto \sum_\sigma n_\sigma \partial_k \sigma.
    \end{align*}
    In the case of $k=0$, we set $\partial_0 = 0$.
\end{definition}
We will frequently leave out the subscript and just write $\partial$ for the boundary 
if it is clear from the context.

A straightforward computation (cf. \cite[Lemma 1.6]{topology_and_geometry}) shows the 
important property 
\begin{align*}
    \partial_k \circ \partial_{k+1} = 0.
\end{align*}
This property implies that $\ker \partial_{k-1} \subseteq \Ima \partial_k$ is 
a subgroup. We call a chain $c \in C_k(X)$ \textit{$k$-cycle} if $\partial_k c = 0$ and 
we call it \textit{$k$-boundary} if $c \in \Ima \partial_{k+1}$. 
Denote the group of $k$-cycles as $Z_k(X)$ and the $k$-boundaries as $B_k(X)$.
Since we are in the abelian setting this motivates us to define 
the resulting factor groups.
\begin{definition}[Homology groups]
    We define the \textit{$k$-th homology group} of the topological space $X$ as
    \begin{align*}
        H_k(X) \vcentcolon= \faktor{Z_k(X)}{B_k(X)}.
    \end{align*} 
\end{definition}
We denote the elements of the homology groups i.e. the equivalence classes of a 
$k$-cycle $c$ as $[c] \in H_k(X)$.

If the $k$-th homology groups is finitely generated then we 
call the rank i.e. the number of generators the \textit{$k$-th Betti number}. 
These Betti numbers are fundamental properties of the topological space. 
For example, the zeroth Betti number corresponds to the number of 
path-components of the space. In 3 dimensions, the first Betti number of 
a compact domain
corresponds to the number of "holes", the second Betti number to 
number of enclosed "voids" in the domain. E.g. a filled torus 
has the zeroth Betti number one, the first Betti number also equal to one
and the second equal to zero which can be proven using the Meyer-Vietoris sequence
(see \cite[Sec.\,IV.18]{topology_and_geometry}). We will not go into this in further
since we do not want to focus too much on algebraic topology.

This construction can be put in an abstract algebraic framework in the following 
way. We call a collection of abelian groups $C_i$, $i\in \integers$ a graded group.                                                   
Together with a collection of homomorphims $\partial_i: C_i \rightarrow C_{i-1}$ 
called \textit{differentials}
s.t. $\partial_{i-1} \circ \partial_i$ this is called a \textit{chain complex} which 
we will denote by $C_*$. 

\begin{example}
    If we set $C_k(X) = {0}$ for $k < 0$ this gives us a chain complex together with 
    the boundary operator. 
\end{example}

Completely analogous to above, we can define the homology groups 
\begin{align*}
    H_k(C_*) \vcentcolon= \faktor{\ker \partial_k}{\Ima \partial_{k+1}}.
\end{align*}

\begin{definition}[Chain map]
    Let $A_*$ and $B_*$ be chain complexes. With a slight abuse of notation
    let us denote the differentials of 
    both chain complexes just by $\partial$.
    Then a \textit{chain map} $f: A^* \rightarrow B^*$ is a collection 
    of homomorphisms $f_i: A_i \rightarrow B_i$ s.t. 
    $f_{i-1} \circ \partial = \partial \circ f_i$.
\end{definition}
{\color{red} Include commuting diagram.} We will most times leave out the 
indices if it is clear what we mean.
The crucial property of these chain maps is that they induce homomorphisms of
the homology groups denoted as
\begin{align*}
    [f_i]: H_i(A_*) \rightarrow H_i(B_*), \, [f_i]([a]) = [f_i([a])]
\end{align*}


Let $\{ C^i \}_{i\in \naturalnum}$ be a collection of abelian groups
and homomorphisms $\partial^i: C^i \rightarrow C^{i+1}$ with 
$\partial^{i+1} \circ \partial^i = 0$ called \textit{codifferentials}. 
Then we call this sequence a 
\textit{cochain complex}. The only difference to chain complexes
is that the index 
increases when applying the codifferential. Hence, they are 
basically the same from an algebraic point of view.
By convention, 
we use superindices for anything that is related to cochain complexes.

We define \textit{cochain maps} completely analogous to chain maps 
i.e. cochain maps commute with the codifferential.

The main motivation for cochain complexes comes from the 
\textit{singular cochain complexes} which we will define in the next 
section.

\begin{example}[De Rham complex]
    Smooth differential forms provide us with another very important example. 
    Let $M$ be a smooth manifold.
    We will use the the notation introduced in 
    Sec.\,\ref{sec:differential_forms}. Then the smooth 
    differential forms $C^\infty \Lambda^k (M)$ together 
    with the exterior derivative give us a cochain complex which we call 
    \textit{de Rham complex}. Note that we have slightly more structure here
    since the $C^\infty \Lambda^k (M)$ are vector spaces and the exterior
    derivative a linear map i.e. a vector space homomorphism.
\end{example}
It turns out that the de Rham complex is closely related with the 
singular cochain complex. This relation will be investigated later in 
Sec.\,\ref{sec:de_rhams_theorem}. %TBD: Example Hilbert complex

\subsection{Cohomology groups}

Let $G$ be any abelian group and $X$ be a topological space as before. 
Then we define the group 
of \textit{$k$-cochains} $C^k(X;G)$ by
\begin{align*}
    C^k(X;G) \vcentcolon= \text{Hom}(C_k(X),\,G)
\end{align*}
i.e. the group of all homomorphisms from $k$-chains $C_k(X)$ to $G$. 
We generally use the superindex $^k$ if something is related to 
cochains and the subindex $_k$ if it is related to chains. 
Just as for chains we now introduce a homomorphism between the groups of cochains
which transforms this into a cochain complex.
\begin{definition}[Coboundary ]
    We define the operator $\partial^k: C^k(X;G) \rightarrow C^{k+1}(X;G)$ via
    \begin{align*}
        (\partial^k f) (c) \vcentcolon= f(\partial_{k+1} c).
    \end{align*}
    for a $(k+1)$-chain $c$.
    We call a cochain $f \in C^k(K;G)$ \textit{closed} if $\partial^k f = 0$ 
    and we call $f$
    \textit{exact} if there is a $g \in C^{k-1}(K;G)$ s.t. $f = \partial^{k-1} g$.
    As for the boundary map we will frequently leave away the superscript if
    the context is clear.
\end{definition}
From the definition it is obvious that $\partial^{k+1} \circ \partial^{k} = 0$ 
and thus we have indeed a cochain complex which we call \textit{singular cochain 
complex}. If there is no confusion with the general notion of cochain complex
we will leave away the term 'singular'.
\begin{definition}[Cochain cohomology]
    Denote the closed $k$-cochains as $Z^k(X;G)$ and the 
    exact ones with $B^k(X;G)$. 
    We then define the \textit{cochain cohomology groups}
    $H^k(X;G)$ as
    \begin{align*}
        H^k(X;G) \vcentcolon= \faktor{Z^k(X;G)}{B^k(X;G)}.
    \end{align*}
\end{definition}
Note that in the case of $G = \real$ this becomes a vector space.

Now of course there is the question how the homology and cohomology groups 
are related to each other. This question is answered by the
\textit{universal coefficent theorem}. But before we can formulate it we have 
to introduce exact sequences.
\begin{definition}[Exact sequence]
    Let $(G_i)_{i\in \integers}$ be a sequence of groups and 
    $(f_i)_{i \in \integers}$ be a sequence of homomorphisms
    $f_i: G_i \rightarrow G_{i+1}$. Then this sequence of homomorphisms is
    called \textit{exact} if $\text{im}\,f_{i-1} = \text{ker}\,f_i$.
\end{definition}

The universal coefficent theorem in the case of simplicial homology states
that the sequence 
\begin{align}
    0 \rightarrow \text{Ext}(H_{k-1}(K),G) \rightarrow 
    H^k(K;G) \xrightarrow{\beta} \text{Hom}(H_k(K),G) 
    \rightarrow 0 \label{eq:univeral_coefficient_theorem}
\end{align}
is exact. 
$\beta$ is defined via 
\begin{align}
    \beta([F])([c]) \vcentcolon= F(c).
    \label{eq:isomorphism_from_universal_coefficent_theorem}
\end{align}
The definition of Ext can be found in \cite{topology_and_geometry},
but it does not matter for our purpose because from now on we will assume
$G = \real$ and
$\text{Ext}(H_{k-1}(X),\real) = 0$. This follows from the fact that 
$\real$ is a divisible and hence injective abelian group. The definition of
these terms and the connections used can also be found in 
\cite[Sec.\,V.6]{topology_and_geometry}. However, we will not dwelve into the 
algebraic background further. In the case of $G = \real$, 
we can conclude from the exactness of the 
above short sequence that $\text{ker}\,\beta = 0$ and 
$\text{im}\,\beta = \text{Hom}(H_k(X),\real)$. So $\beta$ is an isomorphism.


\subsection{De Rham's theorem} \label{sec:de_rhams_theorem}

It turns out that the cochain cohomology is closely related to the cohomology 
of differential forms (\ref{}). Let us recall Stokes' theorem first which said 
that for a $k$-form $\omega \in \smoothcompforms{k}{M}$ 
for a smooth $k$-dimensional oriented manifold $M$ we have 
\begin{align*}
    \int_M d\omega = \int_{\partial M} \omega.
\end{align*}

The following details are taken from Section V.5 and and V.9 from 
\cite{topology_and_geometry}. We will only focus on the main ideas and avoid 
dwelving into the technical details. The interested reader can find more 
arguments in the given reference.

Let now $\sigma$ be a smooth $k$-simplex i.e. $\sigma: \Delta_k \rightarrow 
M$ is smooth. We now define 
\begin{align*}
    \int_\sigma \omega = \int_{\Delta_k} \sigma^* \omega
\end{align*}
and then we define the integral over a $k$-chain 
$c = \sum_\sigma n_\sigma \sigma$
\begin{align*}
    \int_c \omega \vcentcolon= \sum \limits_\sigma n_\sigma \int_\sigma \omega.
\end{align*}
This motivates us to introduce the homomorphism $I: C^\infty \Lambda^k (M) 
\rightarrow \text{Hom}(C_k;\real)$ defined by 
\begin{align*}
    I(\omega)(c) = \int_c \omega.
\end{align*}
\begin{remark}
    There are some technical details that we will not discuss in details here,
    but that should be mentioned. First, $\Delta_k$ is not a manifold. 
    The $(k-2)$ skeleton are, simply speaking, 
    the simplices of dimension $(k-2)$ i.e. the corners for $k=2$ and the 
    edges for $k=3$. If we remove the $(k-2)$ skeleton 
    then $\Delta_k$ is a manifold 
    with boundary. But since this is a null-set w.r.t. the full simplex and 
    the boundary as well, this does not matter for our arguments. 
    Second, Because we are integrating over the $\Delta_k$ their orientation is 
    important and has to be chosen consistently. We will not present here 
    how this is done and will just assume it from now on.
\end{remark}

Using Stokes's theorem and the fact that the exterior derivative commutes with 
the pullback we observe
\begin{align*}
    I(d\omega)(c) &= \sum\limits_\sigma n_\sigma \int_\sigma d\omega 
    = \sum\limits_\sigma n_\sigma \int_{\Delta_k} \sigma^* d\omega 
    = \sum\limits_\sigma n_\sigma \int_{\Delta_k} d\sigma^* \omega
    = \sum\limits_\sigma n_\sigma \int_{\partial \Delta_k} \sigma^* \omega
    \\ &= \int_{\sum_\sigma n_\sigma \partial \sigma} \omega
    = I(\omega)(\partial c) = \partial \big(I(\omega)\big) (c)
\end{align*}
so we obtain
\begin{align*}
    I(d\omega) = \partial \big(I(\omega)\big).
\end{align*}
We see that $I$ is a cochain map and thus 
induces a homomorphism on cohomology
\begin{align*}
    [I]:H^k_{dR}(M) \rightarrow H^k(M;\real).
\end{align*}

Using the notation and the definition of this map we can now formulate 
de Rham's theorem which will become very important later when proving
existence and uniqueness in Sec.\,\ref{sec:existence_and_uniqueness}.
\begin{theorem}[De Rham's theorem]
    $[I]$ is an isomorphism.
\end{theorem}


\section{Existence and uniqueness of solutions}\label{sec:existence_and_uniqueness}

In this section, we will apply the developed theory of the preceding chapters
to prove the existence and uniqueness of the magnetostatic problem 
on exterior domains. With exterior domain we mean that our domain $\Omega \subseteq \real^3$
is the complement of a compact set. The main motivation for this problem is 
the special case of $\Omega$ being the complement of a torus. 
This is also the motivation behind the
topological assumption that we will give. It might be useful to keep this
example in mind.

Recall the magnetostatic problem we are studying.

\begin{problem}\label{prob:magnetostatic_problem}
    Find $B \in H_0(\diver;\Omega)$ s.t.
    \begin{align}
        \curl \, B &= 0, \\ 
        \diver \, B  &= 0 \text{ in } \Omega \text{ and }\\
        \int_\gamma B \cdot dl &= C_0.
    \end{align}
\end{problem}

Of course in order for the curve integral constraint to be well-defined 
we need to check the regularity of solutions. Then using the tools we developed
in the previous sections we will proof existence and uniqueness.

\subsection{Regularity of solutions}\label{sec:regularity_of_solutions}

We will rely on standard regularity results about elliptic systems 
of the following form. Take $A_{ij}^{\alpha \beta} \in \real$
for $i$, $j$, $\alpha$, $\beta = 1,2,3$ and 
$A_{ij}^{\alpha \beta} = A_{ji}^{\beta \alpha}$. Then we have systems of the 
form 
\begin{align}
    -\sum\limits_{\alpha, \beta, j} \partial_\alpha 
        (A_{ij}^{\alpha \beta} \partial_\beta B_j)
    = f_i - \sum\limits_\alpha \partial_\alpha F_i^\alpha
    \label{eq:elliptic_system}
\end{align}
with data $f_i, F_i^\alpha \in L^2(\Omega)$. We call this system 
\textit{elliptic} if $A$ satisfies the Legendre condition i.e.
\begin{align}
    \sum_{\alpha,\beta,i,j}A_{ij}^{\alpha \beta} \xi_\alpha^i \xi_\beta^j
    \geq c |\xi|^2, \quad \forall \xi \in \real^{3 \times 3} 
    \label{eq:legendre_condition}
\end{align}
with $c > 0$. $|\xi|$ is here the Frobenius norm, but technically the chosen 
norm is irrelevant due to all norms on $\real^{3 \times 3}$ being equivalent.
We then call $B$ a weak solution
of the problem if 
\begin{align}
    \int_\Omega \sum\limits_{\alpha,\beta,i,j} 
        A_{ij}^{\alpha \beta} \partial_\beta B_j \partial_\alpha \varphi_i dx
    = \int_\Omega \left\{ \sum\limits_i f_i \varphi_i + 
        \sum\limits_{\alpha,i} F_i^\alpha \partial_\alpha \varphi_i \right\} dx
    \label{eq:weak_elliptic_system}
\end{align}
for all $\varphi \in C^1_0(\Omega;\real^3)$. This formulation is taken from 
\cite[Sec. 1.3]{lectures_on_elliptic_pdes}. At first we will slightly modify
the notion of weak solution. 

\begin{proposition}
    Assume that we have an elliptic system with constant coefficients i.e. 
    $A$ is constant. Then
    (\ref{eq:weak_elliptic_system}) is fulfilled for all 
    $\varphi \in C^1_0(\Omega;\real^3)$ if and only if it is fulfilled 
    for $\varphi \in C^\infty_0(\Omega;\real^3)$.
\end{proposition}
\begin{proof}
    This follows by a simple density argument. Assume that 
    (\ref{eq:weak_elliptic_system}) is fulfilled for all test functions in 
    $C^\infty_0(\Omega;\real^3)$. Now take $\varphi \in C^1_0(\Omega;\real^3)$
    arbitrary. Because $\varphi \in H_0^1(\Omega)^3$ and 
    $C^\infty_0(\Omega;\real^3)$ is dense in $H_0^1(\Omega)^3$ we can find 
    a sequence $(\varphi^{(l)})_{l \in \naturalnum} \subseteq 
    C^\infty_0(\Omega;\real^3)$ s.t. $\varphi^{(l)} \rightarrow \varphi$
    in $H^1(\Omega)^3$. Thus the partial derivatives converge in $L^2(\Omega)$
    and we get
    \begin{align*}
        &\int_\Omega \sum\limits_{i,j,\alpha,\beta} 
            A_{ij}^{\alpha, \beta} \partial_\beta B_j \partial_\alpha \varphi_i
            dx
        = \sum\limits_{i,j,\alpha,\beta} A_{ij}^{\alpha, \beta}
            \int_\Omega \partial_\beta B_j \lim\limits_{l\rightarrow \infty} 
            \partial_\alpha \varphi^{(l)} dx
        \\ &\stackrel{\text{$L^2$ limit}}{=} 
            \lim\limits_{l\rightarrow \infty} 
            \int_\Omega \left\{ \sum\limits_i f_i \varphi^{(l)}_i + 
            \sum\limits_{\alpha,i} F_i^\alpha \partial_\alpha \varphi^{(l)}_i 
            \right\} dx
        = \int_\Omega \left\{ \sum\limits_i f_i \varphi_i + 
            \sum\limits_{\alpha,i} F_i^\alpha \partial_\alpha \varphi_i 
            \right\} dx.
    \end{align*}
    Since $\varphi \in C_0^1(\Omega;\real^3)$ was arbitrary the first 
    direction of the equivalence is proved. The other direction is trivial.
\end{proof}
So we see that in the case of constant coefficents we can consider 
just smooth compactly supported functions as test functions.

Next, we will state the crucial result about the regularity of elliptic systems
which will give us the desired regularity. This is Theorem 2.13 and Remark 2.16 
in \cite{lectures_on_elliptic_pdes} in slightly less generality.
\begin{theorem}
    Let $\Omega$ be an open domain in $\real^n$. Let $A$ be constant and 
    satisfy the Legendre condition (\ref{eq:legendre_condition}). Then for every 
    $B \in H^1_{loc}(\Omega)^3$ weak solution in the sense of
    (\ref{eq:weak_elliptic_system}) with $f \in H^k_{loc}(\Omega)^3$ and 
    $F \in H^{k+1}_{loc}(\Omega;\real^{m\times n})$ 
    we have $B \in H^{k+2}_{loc}(\Omega)^3$. 
\end{theorem}

\begin{corollary}\label{cor:smooth_solution}
    If under the assumptions of the previous theorem we consider the 
    homogeneous problem, i.e.
    \begin{align*}
        \int_\Omega \sum\limits_{\alpha,\beta,i,j} 
        A_{ij}^{\alpha \beta} \partial_\beta B_j \partial_\alpha \varphi_i dx =0
    \end{align*}
    for all $\phi \in C^\infty_0(\Omega;\real^3)$, then $B$ is smooth.   
\end{corollary}
It should be noted that this does not guarantee us any regularity on the 
boundary. 

Before we can apply this result, we have to check whether a solution of our 
problem $B$ is actually in $H^1_{loc}(\Omega)^3$. 

\begin{theorem}\label{thm:solution_in_H1loc}
    Assume $B \in H(\diver;\Omega) \cap H(\curl;\Omega)$. Then 
    $B \in H^1_{loc}(\Omega)^3$.
\end{theorem}
Note that we did not assume $B$ to be a solution.
\begin{proof}
    We know that for a function $u \in H_0(\curl;U) \cap H(\diver;U)$ 
    for some smooth domain $U$ we have $u \in H^1(U)^3$ 
    (cf. \cite[Remark 3.48]{monk}). Our $\Omega$ is just assumed to be open 
    so we can not apply this result directly.

    Take $Q \subset\subset \Omega$ open and pre-compact. 
    Then we can find an open 
    cover of $\overline{Q}$ with a finite set of open balls $\{K_i\}_{i=1}^N$
    s.t. $K_i \subseteq \Omega$ and 
    \begin{align*}
        \overline{Q} \subseteq \bigcup\limits_{i=1}^N K_i.
    \end{align*}
    As a open cover of a compact set we can find a smooth partition of unity 
    $\{\chi_i\}_{i=1}^N$ subordinate to $\{K_i\}_{i=1}^N$. 
    $(B \chi_i)|_{K_i} \in H_0(\curl;K_i) \cap H(\diver;K_i)$ and thus 
    $(B \chi_i)|_{K_i} \in H^1(K_i)^3$ by the above mentioned result. 
    Also because $B \chi_i$ has compact support in $K_i$ we can extend it by 
    zero to obtain
    $B \chi_i \in H^1(\real^3)^3$ where we abused the notation by denoting 
    the extension the same. Whence,
    \begin{align*}
        B|_Q = \large( \sum\limits_{i=1}^N \chi_i |_Q \large) B|_Q = 
        \sum\limits_{i=1}^N (\chi_i B)|_Q \in H^1(Q)
    \end{align*}
    i.e. $B \in H^1_{loc}(\Omega)^3$.
\end{proof}

The following lemma is a reformulation of the differential operator 
$\grad \diver - \curl \curl$ which will be needed when we write our 
magnetostatic problem in the above standard elliptic form.

\begin{lemma}\label{lem:graddiv_curlcurl_equals_componentwise_laplacian}
    Let $F \in H^2_{loc}(\Omega)^3$. Then 
    \begin{align*}
        \grad \diver F - \curl \curl F 
        = \begin{pmatrix} \Delta F_1 \\ \Delta F_2 
            \\ \Delta F_3  \end{pmatrix}.
    \end{align*}
\end{lemma}
\begin{proof}
    By a simple calculation and changing the order of differentiation
    \begin{align*}
        \grad \diver F = 
            \begin{pmatrix} \partial_1^2 F_1 + \partial_1 \partial_2 F_2
                + \partial_1 \partial_3 F_3
            \\ \partial_1\partial_2 F_2 + \partial_2^2 F_2 + 
                \partial_2\partial_3 F_3
            \\ \partial_1 \partial_3 F_1 + \partial_2\partial_3 F_2
                + \partial_3^2 F_3
            \end{pmatrix}
    \end{align*}
    and 
    \begin{align*}
        &\curl \curl F = \curl \begin{pmatrix} 
            \partial_2 F_3 - \partial_3 F_2 \\ \partial_3 F_1 - \partial_1 F_3 
            \\ \partial_1 F_2 - \partial_2 F_1  \end{pmatrix}
        = \begin{pmatrix} 
            \partial_2 (\partial_1 F_2 - \partial_2 F_1)
                - \partial_3 (\partial_3 F_1 - \partial_1 F_3)
            \\ \partial_3 (\partial_2 F_3 - \partial_3 F_2)
                - \partial_1 (\partial_1 F_2 - \partial_2 F_1)
            \\ \partial_1 (\partial_3 F_1 - \partial_1 F_3)
                - \partial_2 (\partial_2 F_3 - \partial_3 F_2)  
            \end{pmatrix}
        \\ &= \begin{pmatrix}
            \partial_1 \partial_2 F_2 - \partial^2_2 F_1 - \partial_3^2 F_1
            + \partial_1 \partial_3 F_3 
            \\ \partial_2 \partial_3 F_3 - \partial^2_3 F_2 - \partial_1^2 F_2
            + \partial_1 \partial_2 F_3
            \\ \partial_1 \partial_3 F_3 - \partial^2_3 F_2 - \partial_2^2 F_3
            + \partial_2 \partial_3 F_2  
            \end{pmatrix}
    \end{align*}
    and so by subtracting the two expressions
    \begin{align*}
        \grad \diver F - \curl \curl F  
        = \begin{pmatrix}
            \partial_1^2 F_1 + \partial_2^2 F_1 + \partial_3^2 F_1
            \\ \partial_1^2 F_2 + \partial_2^2 F_2 + \partial_3^2 F_3
            \\ \partial_1^2 F_3 + \partial_2^2 F_3 + \partial_3^2 F_3
        \end{pmatrix}
        = \begin{pmatrix}
            \Delta F_1 \\ \Delta F_2 \\ \Delta F_3
        \end{pmatrix}.
    \end{align*}
\end{proof}
We want to rewrite this system in the expression of the elliptic system 
(\ref{eq:elliptic_system}). We can rewrite the Laplacian
\begin{align*}
    - \Delta F_i = - \sum\limits_{\alpha = 1}^3 
        \partial_\alpha \partial_\alpha F_i
    = - \sum\limits_{\alpha,\beta = 1}^3 
    \partial_\alpha \delta_{\alpha,\beta} \partial_\beta F_i
    = - \sum\limits_{\alpha,\beta,j = 1}^3 
    \partial_\alpha \delta_{\alpha,\beta} \delta_{ij} \partial_\beta F_j
\end{align*}
so we get $A_{ij}^{\alpha\beta} = \delta_{ij} \delta_{\alpha \beta}$.
We have to check that the resulting differential operator is indeed
elliptic, but this trivial because for any 
$(\xi_\alpha^i)_{1\leq i,\alpha \leq 3}$
we get 
\begin{align*}
    \sum\limits_{\alpha,\beta,i,j} A_{ij}^{\alpha\beta} \xi_\alpha^i \xi_\beta^j 
    = \sum\limits_{\alpha,\beta,i,j} \delta_{ij} \delta_{\alpha \beta} 
        \xi_\alpha^i \xi_\beta^j 
    = \sum\limits_{\alpha,i} (\xi_\alpha^i)^2 = |\xi|^2
\end{align*}
so  the Legendre condition (\ref{eq:legendre_condition}) 
is fulfilled and the resulting system is elliptic. The weak formulation 
is 
\begin{align*}
    \int_\Omega \sum\limits_{\alpha,\beta,i,j} \delta_{ij} \delta_{\alpha\beta}
        \partial_\beta B_j \partial_\alpha \varphi_i dx 
    = \sum\limits_{i=1}^3 \int_\Omega \nabla B_i \cdot \nabla \varphi_i dx.
\end{align*}
Here we can assume $\varphi \in C_0^\infty (\Omega)^3$ because all coefficients
are constant.

\begin{theorem}[Smoothness of solutions]
    Let $\Omega \subseteq \real^3$ open and 
    $B \in H(\diver;\Omega) \cup H(\curl;\Omega)$ and 
    \begin{align*}
        \curl B &= 0,
        \\ \diver B &= 0.
    \end{align*}
    Then $B$ is smooth.
\end{theorem}
\begin{proof}
    Take $\varphi \in C_0^\infty(\Omega)^3$. Then 
    \begin{align*}
        0 &= \int_\Omega \diver B \diver \varphi + \curl B \cdot \curl \varphi dx
        = - \int_\Omega B \cdot (\grad \diver \varphi - \curl \curl \varphi) dx
        \\ &\stackrel{Lemma 
            \ref{lem:graddiv_curlcurl_equals_componentwise_laplacian}}{=} 
            - \int_\Omega B \cdot 
            \begin{pmatrix}
                \Delta \varphi_1 \\ \Delta \varphi_2 \\ \Delta \varphi_3
            \end{pmatrix}
        = \sum\limits_{i=1}^3 \int_\Omega \nabla B_i \cdot \nabla \varphi_i dx.
    \end{align*}
    Note that the last integration by parts is well defined because 
    $B \in H^1_{loc}(\Omega)$ according to Thm.\,\ref{thm:solution_in_H1loc}. 
    So $B$ is a weak solution 
    of the elliptic system given by 
    $A_{ij}^{\alpha \beta} = \delta_{ij} \delta_{\alpha\beta}$. Because 
    we look at the homogenous problem our right hand side is obviously smooth 
    and thus $B$ is smooth as well due to Cor.\,\ref{cor:smooth_solution}.
\end{proof}

\begin{remark}
    Obviously, the above arguments can be generalized by using a 
    non-zero right hand side of our problem. Then we will in general 
    not obtain a smooth solution, but for a sufficiently regular right hand 
    side the curve integral would still be well-defined.
    {\color{red} How much regularity? Source?}
\end{remark}


\subsection{Reformulation of the problem} 

We will return now to the magnetostatic problem. In order to use the results
above we will reformulate the problem in the notation of differential forms.
From now on we assume $n=3$ i.e. we are in three dimensional space.
There are two ways to identify a vector field with a differential form 
(cf. \cite[Table 6.1 and p.70]{arnold}) either as a 1-form or a 2-form. 
For a vector field $B$ we define
\begin{align*}
    F^1\, B &\vcentcolon= B_1 \, dx_1 + B_2 \, dx_2 + B_3\, dx_3 \text{ and}\\
    F^2\, B &\vcentcolon= B_2 \, dx_2 \wedge dx_3 - B_2 \, dx_1 \wedge dx_3
        + B_3 \, dx_1 \wedge dx_2
\end{align*} 
as the corresponding 1-form and 2-form. 
Then the exterior derivative is $dF^2\,\omega$ corresponds to the divergence,
the codifferential $\delta F^2\,\omega$ 
corresponds to the curl and the normal component
being zero on the boundary corresponds 
to $\omega \in \mathring{H}^2(d)$.\cite{}. 

If we then use the association of
3-forms with scalars we have the corresponding boundary value problem without
the integral condition for 
2-forms: Find $\omega \in \mathring{H}^2(d)$ s.t.
\begin{align}
    \delta \omega &= 0, \\ 
    d\omega  &= 0 \text{ in } \Omega.
\end{align}
Next, we have to add the integral condition. 
We remind the reader that we are in three dimensions so
$**= \text{Id}$ 
and observe
\begin{align*}
    *F^2 \, B  &= B_1 \, **dx_1 + B_2 \, **dx_2 + B_3\, **dx_3 
        = B_1 \, dx_1 + B_2 \, dx_2 + B_3\, dx_3\\ 
    &= F^1 \, B.
\end{align*}
{\color{red}: Actually, we already use the Hodge star to define the vector 
proxies. So of course the vector proxy of the Hodge star will be the same.}
Then we have 
\begin{align*}
    \int_\gamma * F^2\, B = \int_\gamma F   ^1\, B = \int_\gamma B \cdot \text{d}l.
\end{align*}
In the last step we used the fact that the integration of a 1-form over a
curve is equivalent to the curve integral of the associated vector field
(cf. \cite[Sec. 6.2.3]{arnold}). Hence, we can add the integral condition 
\begin{align}
    \int_\gamma *\omega = C_0 \label{integral_condition}.
\end{align}
However, we have  only $\omega \in \mathring{H}_0^2(d,\delta)$ so 
$*\omega \in H^1(d)$ so this integral might not be well defined. In order 
to deal with this, we will again use the operator $\rop$ from 
Sec.\,\ref{sec:isomorphism_cohomology}. 

We know from Thm.~\ref{thm:operators} that $\rop *\omega \in S^1_2(\omegabar)$.
Using the operator $\varphi$ from \ref{eq:} we obtain 
$\varphi \rop *\omega \in S^1_2(K)$. Now we know from Sec.~\ref{sec:} that 
the integration mapping $I: S^1_2(K) \rightarrow C^1_2(K)$ is well-defined.
Denote $\bar{I} \vcentcolon= I\circ \varphi \circ \rop$ and 
let us replace the integral condition (\ref{eq:integral_condition}) with 
\begin{align*}
    \bar{I}(*\omega)(\gamma) = C_0.
\end{align*}

Of course, we have to justify why this is reasonable. 
So let us take $\eta \in S^1_2(\omegabar)$. That means we can integrate it 
directly using the definition from \ref{}. 
Let us also assume that $\eta$ is closed and 
$\int_\gamma \eta = C_0$. %TBD: Define the integral on S^k_p(\omegabar)
Then we know from Thm.~\ref{thm:operators}
\begin{align*}
    \rop \eta = \eta + d\aop \eta + \aop d\eta 
    \stackrel{\eta \text{ closed}}{=} \eta + d\aop \eta.
\end{align*}
We know further that $\aop \eta \in S^0_2(\omegabar)$. 
Apply $\varphi$ on both sides and use that fact that it commutes 
\ref{} with the 
exterior derivative to get
\begin{align*}
    \varphi \rop \eta = \varphi\eta + d \varphi\aop \eta.
\end{align*}
$[I]$ is an isomorphism of cohomology and thus sends exact S-forms to exact 
cochains. $\gamma$ is closed so $I(d \varphi\aop \eta)(\gamma) = 0$ and 
we conclude 
\begin{align*}
    \bar{I}(\eta)(\gamma) = I(\varphi \rop \eta)(\gamma) = I(\varphi \eta)
    \stackrel{\text{by def.}}{=} I(\eta),
\end{align*}
i.e. the integral remains unchanged if we integrate closed forms over 
closed chains. Because $*\omega$ is closed we thus do not change the integral 
if the curve integral $*\omega$ would already have been well defined before.

To summarize we obtain the following problem.
\begin{problem} \label{prob:magnetostatic_reformulated}
    Find $\omega \in \mathring{H}^2(d;\Omega)$ s.t.
    \begin{align*}
        d \omega &= 0, \\
        \delta \omega &= 0 \text{ in $\Omega$}, \\
        \bar{I}(*\omega)(\gamma)  &= C_0.
    \end{align*}
\end{problem}
\noindent We will examine existence and 
uniqueness of this problem in the next section.


\subsection{Existence and uniqueness}

The curve integral condition is closely linked to the topology of our domain 
which we will have to use in our proof.
This will rely on the tools of homology from Sec.\,\ref{}. 
Because of this connection if we want the curve integral
to give us uniqueness of the solution we need to assume certain topological 
properties. In our case, this will be the condition that our first 
homology group is generated by the curve that we are integrating over i.e.
\begin{align*}
    H_1(\Omega) = \integers [\gamma].
\end{align*}

Then we get the following existence and uniqueness result on the 
level of homology.

\begin{proposition}\label{prop:uniqueness_cochain}
    Assume that $H_1(\Omega) = \integers [\gamma]$ i.e. the homology 
    class of the 
    closed $1$-chain $\gamma$ is a generator of the first homology group.
    Then we have the following:
    \begin{enumerate}[(i)]
        \item For any $C_0 \in \real$ there exists a closed $1$-cochain 
            $F \in Z^1(\Omega)$ with $F(\gamma) = C_0$,
        \item any other $G \in Z^1(\Omega)$ with $G(\gamma) = C_0$ 
            is in the same cohomology class i.e. $[F] = [G]$
    \end{enumerate}
    i.e. the cochain is unique up to cohomology.
\end{proposition}
\begin{proof}
    \textbf{Proof of (i)} %TBD: This could be wrong
    Because $[\gamma]$ is a generator of the homology group we  obtain a 
    homomorphism $\hat{F} \in \text{Hom}(H_1(\Omega),\real)$ by fixing
    $\hat{F}([\gamma]) = C_0$. This determines the other values.
    Then we know from (\ref{eq:univeral_coefficient_theorem}) that there exists
    a $[F] \in H^1(\Omega)$ with $\beta([F]) = \hat{F}$ because $\beta$ is a 
    isomorphism. So we obtain
    \begin{align*}
        F(\gamma) = \beta([F])([\gamma]) = \hat{F}([\gamma]) = C_0.
    \end{align*}

    \textbf{Proof of (ii)} %TBD: This could be wrong
    Take $[c] \in H_1(\Omega)$ arbitrary. 
    Then there exists  $n \in \integers$ s.t.
    $[c] = n [\gamma]$.
    Using $\beta$ from (\ref{eq:univeral_coefficient_theorem})
    We have
    \begin{align*}
        \beta([F])([c]) = \beta([F])(n [\gamma]) 
        = n \beta([F])([\gamma]) = n \, F(\gamma) = n \, G(\gamma) = 
        \beta([G])([c])
    \end{align*}
    and thus $\beta([F]) = \beta([G])$. Because $\beta$ is an isomorphism
    we arrive at $[F] = [G]$.
\end{proof}
This abstract topological result can now be linked to the differential 
forms via the de Rham isomorphism from Sec.\,\ref{sec:de_rhams_theorem}. 
We will formulate it in a way 
that demonstrates the connection of differential forms and cochains.
\begin{corollary}\label{cor:existence_uniqueness_1form}
    Assume $H_1(\Omega) = \integers [\gamma]$ as above. Then
    \begin{enumerate}[(i)]
        \item For any $C_0 \in \real$ there exists a closed smooth $1$-form 
            $\theta \in \mathfrak{Z}^1(\Omega)$ with 
            \begin{align*}
                I(\theta)(\gamma) = \int_\gamma \theta = C_0
            \end{align*}
        \item any other $\eta \in \mathfrak{Z}^1(\Omega)$ with 
            \begin{align*}
                I(\eta)(\gamma) = \int_\gamma \eta = C_0
            \end{align*}
            is in the same cohomology class of $H_{dR}^1(\Omega)$ 
            i.e. $[\eta] = [\theta]$.
    \end{enumerate}
\end{corollary}
\begin{proof}
    \textbf{Proof of (i)}
    Recall from Sec.\,\ref{sec:de_rhams_theorem} 
    that the integration of differential forms 
    over chains induces an isomorphism on cohomology 
    $[I]: H_{dR}^1(\Omega) \rightarrow H^1(\Omega)$ which we call 
    de Rham isomorphism. We know from 
    Prop.\,\ref{prop:uniqueness_cochain} that there exists $F\in H^1(\Omega)$ 
    s.t. $F(\gamma) = C_0$. The surjectivity of the de Rham isomorphism 
    now gives us $[\theta] \in H^1(\Omega)$ s.t.
    \begin{align*}
        [I(\theta)] = [I](\theta) = [F]
    \end{align*}
    i.e.
    \begin{align*}
        I(\theta) = F + \partial^0 J
    \end{align*}
    with $J \in C^0$. Then, 
    \begin{align*}
        I(\theta)(\gamma) = F(\gamma) + \partial^0 J(\gamma) 
        = C_0 + J(\partial_1 \gamma) 
        \stackrel{\text{$\gamma$ closed}}{=} C_0. 
    \end{align*}
    \textbf{Proof of (ii)}
    We have $I(\eta)$ is a $1$-cochain with $I(\eta)(\gamma) = C_0$.
    Thus, we can apply Prop.\,\ref{prop:uniqueness_cochain} to get
    \begin{align*}
        [I](\eta)=[I(\eta)] = [I(\theta)]=[I](\theta).
    \end{align*}
    Because $[I]$ is an isomorphism we can conclude $[\eta] = [\theta]$.
\end{proof}

\begin{theorem}[Existence of solution]\label{thm:existence}
    Let $\Omega \subseteq \real^3$ be such that $\real^3 \setminus \Omega$
    is compact. 
    For the topology, we require that $H_1(\Omega) = \integers [\gamma]$ 
    for a $1$-chain 
    $\gamma$. Assume further that there exists an $\epsilon$-neighborhood 
    \begin{align*}
        \Omega_\epsilon \vcentcolon= \{ x \in \real^3 \mid
            d(x,\Omega) < \epsilon \} 
    \end{align*}
    s.t. $H_1(\Omega_\epsilon) = \integers [\gamma]$ as well.
    Then there exists a solution 
    to Problem \ref{prob:magnetostatic_reformulated}.
\end{theorem}
Let us say a view words about the topological assumption 
regarding $\Omega_\epsilon$. This just means that we can slightly increase 
the domain without changing the first homology group. As an example, 
think again of a torus in $\real^3$. Assuming the torus has non-empty interior 
we can slightly reduce the poloidal radius without changing the topology of its 
the exterior domain.
\begin{proof}
    At first we want to find a smooth differential $1$-form 
    $\theta \in \Lambda^1(\Omega)$ with the desired curve integral. 
    In order to do that we will increase the 
    domain slightly.
    We start by referring to Cor.\,\ref{cor:existence_uniqueness_1form} 
    to get a smooth differential form 
    $\tilde{\theta} \in \Lambda^1(\Omega_\epsilon)$ with 
    \begin{align}
        \int_\gamma \tilde{\theta} = C_0. \label{eq:integral_theta_tilde}
    \end{align}
    We now refer back to \ref{} to change back to vector proxies. Let 
    $\tilde{\phi}$ be the vector proxy of $\tilde{\theta}$ 
    i.e. $\tilde{\phi} \in C^\infty(\tilde{\Omega})^3$.
    Because $\tilde{\theta}$ is closed and (\ref{eq:integral_theta_tilde}) holds
    we obtain the corresponding properties 
    of $\tilde{\phi}$ which are 
    \begin{align*}
        \int_\gamma \tilde{\phi} \cdot dl &= C_0 
        \\ \curl \tilde{\phi} &= 0.
    \end{align*}
    We define $\phi$ by restricting $\tilde{\phi}$ to $\Omega$. 
    Let now $K_R$ be the open ball around the origin with radius $R>0$ large
    enough s.t. $\Omega^c \subseteq K_R$ and $\gamma \subseteq K_R$. 
    We now restrict $\theta$ further to
    $\Omega_R \vcentcolon= \Omega \cap K_R$. We denote the restriction with 
    $\phi_R \vcentcolon= \phi|_{\Omega_R}$. 

    We now need to construct a harmonic vector field with zero tangential trace
    s.t. we can extend it by zero. We do this by using the Hodge 
    decomposition for $1$-forms on $\Omega_R$ in the $3$D case \ref{}.
    We project $\phi_R$ onto the harmonic fields to obtain $B_R$ which has zero 
    tangential trace. So there exists a sequence 
    $(\psi_i)_{i \in \naturalnum} \in H^1(\Omega_R)$ s.t.
    \begin{align*}
        B_R = \phi_R - \lim\limits_{i\rightarrow \infty}\nabla \psi_i
    \end{align*}
    We also know that $B_R$ is smooth because 
    it is curl and divergence free from the section about the 
    smoothness of solutions (Sec. \ref{sec:regularity_of_solutions}). 
    We want to check that the curve
    integral did not change. We know from \ref{} that the 
    image of the exterior derivative is closed on bounded domains. 
    Formulated in vector proxies, this
    means that $\nabla H^1(U)$ is closed in $L^2$ if $U$ is a bounded domain. 
    So we get that 
    \begin{align*}
        \lim\limits_{i\rightarrow \infty}\nabla \psi_i = \nabla \psi_R
    \end{align*}
    with $\psi_R \in H^1(\Omega_R)$. Because $B_R$ and $\phi_R$ are smooth
    $\psi_R$ must be smooth as well and so we have   
    \begin{align*}
        \int_\gamma B_R\cdot dl = \int_\gamma \phi_R\cdot dl.
    \end{align*}
    Now extend $B_R$ by zero onto $\real^3 \setminus K_R$.
    Denote this extension as $\overline{B}_R$.
    Because $B_R$ has tangential trace zero and is curl-free its extension 
    $\overline{B}_R \in H(\curl;\Omega)$ and is also curl-free. 
    Here it is important to remember 
    that $\real^3 \setminus K_R \subseteq \Omega$. 
    That means of course that $\overline{B}_R$ 
    might not be smooth on $\partial K_R$ 
    as it might have a jump. 
    Now we can once again use the Hodge 
    decomposition of two forms. This time on the whole domain $\Omega$ to find 
    a harmonic field $B$ and a sequence $(\rho_i)_{i\in \naturalnum}$
    s.t.
    \begin{align*}
        \overline{B_R} = B 
            +\lim\limits_{i\rightarrow \infty}\nabla \rho_i.
    \end{align*}
    Notice that because $B$ is a harmonic vector field 
    and because it is the vector 
    proxy of a harmonic $2$-form it already satisfies 
    the Neumann boundary condition and it is divergence as well as curl free.
    That means $B$ is a solution if the curve integral condition is satisfied.
    In order to see this, note that we have on $K_R$
    \begin{align*}
        B_R = B|_{K_R} +\lim\limits_{i\rightarrow \infty}\nabla \rho_i|_{K_R}.
    \end{align*}
    Because the image of the gradient is closed on bounded domains 
    we have $\rho_R \in H^1(K_R)$ s.t.
    \begin{align*}
        B_R = B|_{K_R} + \nabla \rho_R.
    \end{align*}
    With the same argument as above, $\rho_R$ must be smooth and so
    \begin{align*}
        \int_\gamma B \cdot dl =  \int_\gamma B|_{K_R} \cdot dl 
        =  \int_\gamma B_R \cdot dl = C_0
    \end{align*}
    and thus $B$ is indeed a solution.
\end{proof}

In the proof of uniqueness we will use the following lemma. 
\begin{lemma}\label{lem:gradient_sequence}
    Let $\phi \in L^2_{loc}(\Omega_\epsilon)$ 
    with $\nabla \phi \in L^2(\Omega)^3$. Then 
    there exists a sequence $(\phi_i)_{i \in \naturalnum} \subseteq H^1(\Omega)$
    s.t. $\nabla \phi_i \rightarrow \nabla \phi$ in $L^2(\Omega)^3$.
\end{lemma}
\begin{proof}
    Take $K_R$ the open ball around the origin with $R$ large enough 
    s.t. $(K_R)^c \subseteq \Omega$. 
    Define $\Omega_R \vcentcolon= K_R \cap \Omega$. Then 
    $\overline{\Omega}_R \subseteq K_{R+1}$, where $K_{R+1}$ is the open ball 
    around the origin with radius $R+1$, $\Omega_R$ is a Lipschitz 
    domain and $K_{R+1}$ is pre-compact and
    $\phi|_{\Omega_R} \in W^{1,2}(\Omega_R)$. 
    Note that here we need the fact that $\phi \in L^2_{loc}(\Omega_\epsilon)$
    because then $\Omega_R$ is pre-compact in $\Omega_\epsilon$. 
    Therefore we can find an extension
    $E\phi \in W_0^{1,2}(\Omega_{R+1}) \hookrightarrow W^{1,2}(\mathbb{R}^3)$
    (cf. \cite[Sec.\,1.5.1]{mazya}). So we can now define
    \begin{align*}
    \bar{\phi} \vcentcolon=
    \begin{cases}
        \phi & \mbox{in $\Omega$}\\
        E\phi & \mbox{in $\Omega^c$.}\\
    \end{cases}
    \end{align*}
    Then $\bar{\phi} \in L^2_{loc}(\real^3)$ and 
    $\nabla \bar{\phi} \in L^2(\real^3)^3$. 
    Then there exists a sequence 
    $(\phi_l) _{l \in \naturalnum} \subseteq C^\infty_0(\real^3)$ s.t.
    $\nabla \phi_l \rightarrow \nabla \bar{\phi}$ in $L^2(\real^3)^3$ 
    (cf. \cite[Lemma 1.1]{simader}). By restricting $\phi_l$ to $\Omega$ 
    we obtain the result.
\end{proof}


\begin{theorem}
    Let the same assumptions hold as in Thm.\,\ref{thm:existence}.
    Then the solution of the problem is unique.
\end{theorem}

\begin{proof}
    Let $B$ and $\tilde{B}$ both be solutions and denote with $\omega$ 
    and $\tilde{\omega}$ the corresponding $1$-forms. 
    So we have $I(\omega)(\gamma) = I(\tilde{\omega})(\gamma) = C_0$.
    Then we know from Cor.\,\ref{cor:existence_uniqueness_1form} 
    that $\omega$ and $\tilde{\omega}$ are in the 
    same cohomology in $\mathcal{H}^1$. This is equivalent to saying that 
    there exists a smooth $\mu$ s.t. 
    \begin{align*}
        B - \tilde{B} = \grad \mu.
    \end{align*}
    However, $\mu$ need not be in $L^2$ since $\Omega$ is unbounded.
    But we know that $\grad \mu \in L^2(\Omega)^3$ and 
    $\mu \in L^2_{loc}(\Omega)$. 
    Here we can now apply Lemma \ref{lem:gradient_sequence} and conclude
    \begin{align*}
        B - \tilde{B} \in \overline{\grad H^1(\Omega)}.
    \end{align*}
    Now remembering the Hodge decomposition in the 3D case we know 
    \begin{align*}
        B - \tilde{B} \in \overline{\grad H^1(\Omega)}^\perp
    \end{align*}
    as well and thus $B = \tilde{B}$ which concludes the proof of uniqueness.
\end{proof}


% \begin{proof}
%     Let $\omega, \tilde{\omega}$ both be solutions. 
%     Because  $*\omega$ and
%     $*\tilde{\omega}$ are closed the cochains 
%     $c\mapsto \int_c \rop*\omega$ and 
%     $c\mapsto \int_c \rop*\tilde{\omega}$ are closed.
    
%     Due to $\int_\gamma \rop *\omega = \int_\gamma \rop *\tilde{\omega}$ and the 
%     assumption that $[\gamma]$ spans the homology space we have with 
%     Prop.\,\ref{uniqueness_cochain} 
%     $[I(\rop *\omega)] = [I(\rop *\tilde{\omega})]$
%     and because $[I]$ is an isomorphism 
%     $[\rop *\omega] = [\rop *\tilde{\omega}]$. Hence,
%     \begin{align*}
%     [*\tilde{\omega}] = [\rop *\tilde{\omega}] = 
%     [\rop *\omega] = [*\omega].
%     \end{align*}
%     That is equivalent to the
%     existence of some $0$-form $\phi \in H^0(d)$ s.t.
%     $*\omega = *\tilde{\omega} + d\phi$. We continue by applying the Hodge
%     star operator to both sides and use the definition of the codifferential 
%     $\delta$:
%     \begin{align*}
%         \omega = \tilde{\omega} + *d\phi = \tilde{\omega} + *d**\phi 
%         = \tilde{\omega} + (-1)^{(n-k)(k-1)+1}\delta * \phi.
%     \end{align*}
%     Then because $\omega$ and 
%     $\tilde{\omega}$ are harmonic we have 
%     $\omega, \tilde{\omega} \perp \delta H^{3}(\delta)$ and therefore 
%     \begin{align*}
%     \omega = \tilde{\omega}.    
%     \end{align*}
% \end{proof}
% If we now translate this back to standard vector calculus terms we have found 
% the unique solution of the homogeneous magnetostatic on our domain $\Omega$.

% \section{Application: Homogeneous magnetostatic problem on the 
% exterior domain of a torus}

% As an application of this general boundary value problem we will have a look
% at the following magnetostatic problem. Let $\Omega$ be the exterior domain of
% a triangulated torus i.e. $\real^3 \setminus \Omega$ is a torus with 
% triangulated surface. Let $B$ be the magnetic field. We then have the following
% boundary value problem:
% %%%TBD: Include a picture
% It is natural to identify the magnetic field $B$ with a 2-form $\omega$.
% Then the exterior derivative is $d\omega$ corresponds to the divergence,
% the codifferential $\delta$ corresponds to the curl and the normal component
% being zero is corresponds to $\omega \in \mathring{W}^2_2(\Omega)$.\cite{}. 
% The curve integral is an integration over a one-dimensional manifold and
% corresponds therefore to the integration of a one-form. Therefore the
% condition \ref{curve_integral_contition} is can be expressed with the hodge 
% star operator $*$ as
% \begin{align*}
%     \int_\gamma *\omega = C_0.
% \end{align*}


% Now we want to apply our previous results.
% $\Omega$ fulfills all required assumptions for the domain. Because $\gamma$
% goes around the torus once and the homology space $H^1_c$ is one-dimensional
% {\color{red}(TBD: This has to be referenced or proven)}. Therefore because 
% $\gamma$ is not a boundary $[\gamma]$ spans $H^1_c$. Now all assumptions are
% fulfilled and we can apply our result. We will do so on 1-forms and transfer the
% result to 2-forms using the Hodge star operator.

% \subsection*{Existence}
% Let 
% $\tilde{\omega} \in \mathring{W}^1_2(\Omega) $ be the unique solution of our 
% general problem \ref{} and define $\omega \vcentcolon= *\tilde{\omega}$. Then
% we use $**=(-1)^{k(n-k)}\tilde{\omega} = \tilde{\omega}$ \cite[p.66]{arnold} 
% to get 
% \begin{align*}
%     d\omega = ** d*\tilde{\omega} = * (-1)^{n(k-1)+1} \delta \tilde{\omega}
%     = 0
% \end{align*}
% and
% \begin{align*}
%     \delta \omega = (-1)^{n(k-1)+1} *d*\omega = (-1)^{n(k-1)+1}* d\tilde{omega}
%     = 0.
% \end{align*}



%%% TBD: Nowaczyk's thesis is not a master's thesis
\printbibliography
\end{document}