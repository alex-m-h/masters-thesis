\documentclass[12pt,a4paper]{article}
\usepackage[utf8]{inputenc}
\usepackage[english]{babel}
\usepackage{amsmath}
\usepackage{amsfonts}
\usepackage{amssymb}
\usepackage{amsthm}
\usepackage{xcolor}
\usepackage{csquotes}
\usepackage{mathrsfs}
\usepackage{mathtools}

\usepackage[backend=biber,style=alphabetic,sorting=ynt]{biblatex}
\addbibresource{bibliography.bib}

\newtheorem{assumption}{Assumption}
\newtheorem{definition}{Definition}
\newtheorem{proposition}{Proposition}
\newtheorem{theorem}{Theorem}

\newcommand{\aop}{\mathscr{A}}
\newcommand{\omegabar}{\bar{\Omega}}
\newcommand{\real}{\mathbb{R}}
\newcommand{\rop}{\mathscr{R}} % short for R operator


\begin{document}
The goal is to solve the following problem. Let $\Omega \subseteq \real^n$ be a
an exterior polyhedral domain of a compact set i.e. $\real^n \setminus \Omega$ 
is a compact polyhedron. We then have the following boundary value
problem: For a fixed $C_0 \in \real$ and closed bounded $k$-chain $\gamma$ 
find $\omega \in H\Lambda^k(\Omega)$ s.t.
\begin{align*}
    d\omega &= 0, \\
    \delta\omega &= 0, \\
    \text{tr } \omega &= 0 \text{ and} \\
    \int_\gamma \omega &= C_0
\end{align*}
Because we consider polyhedral domains we assume that $\gamma$ consists of 
finitely many $k$-simplices and that the cohomology class $[\gamma]$ is a 
generator of the simplicial homology group.
Our goal will be to show existence and uniqueness of solutions. In order to show
this, we rely on a result about the isomorphism of a simplicial cohomology 
space $H^k_p(K)$ which will be defined below
and the $L^p$-cohomology space $H^k_{p,dR}(\omegabar)$ ($dR$ short for de Rham).
This result was proven in \cite{goldshtein}. In the diploma thesis by Nikolai
Nowaczyk \cite{nowaczyk}, which mostly is based on this paper, 
many additional details can be found. The first part will be to present this 
result.

\section{Isomorphism of Cohomology}
We assume that the $\omegabar$ admits a smooth triangulation $\tau: |K|
\rightarrow \omegabar$ with $|K|$ being the geometric realization of the
simplicial complex $K$. Because $\omegabar$ is itself a polyhedron we can
assume that $\omegabar$ and $|K|$ are equal as subsets of $\real^n$. 
However, we will use different metrics. We use the Euclidian metric on 
$\omegabar$ and we use the standard simplicial metric on $|K|$ (cf. 
\cite[p.191]{goldshtein}). This metric is defined as follows:

Choose some numbering of the vertices $\{ x_1,\, x_2, ... \}$ 
Take $f: |K| \rightarrow \ell^2$ where $\ell^2$ is the 
Hilbert space of real-valued square-summable sequences s.t. $f$ is affine on 
every simplex. This mapping is unique.%%%TBD: Proof uniqueness%%%%

Then we define the metric on $|K|$ as $g_S = f^*g$ where $g$ is the standard
metric in $\ell^2$. That means for a differential

%%TBD: I know how to define the metric, but I do not know how to define the
%% differential


We have two crucial assumptions.  

\begin{assumption}{Star-boundedness}
The star of every vertex in the mesh contains at most $N$ simplices.
\end{assumption}
This assumption is for example fulfilled if we consider a shape-regular
triangulation of our domain \cite{???}.

In order to formulate the second assumption we have to note that for our
triangulation $\tau$ we have that $M = |K|$. Therefore $\tau$ is just the 
identity in our case. However, we use a different metric on $M$ and $|K|$. 
On our domain $M$ we use the Euclidian metric. On $|K|$ however, we have to use
the standard simplicial metric which is defined as follows (cf. 
\cite[p.191]{goldshtein}). 
We enumerate the vertices of the triangulation as ${ x_1,\, x_2,\, ... }$.
Let $f: |K| \rightarrow \ell^2$ be an embedding 
of the triangulation $|K|$ into $\real^\infty$ s.t. $f(x_i) = e_i$ where $e_i$ 
are the standard unit vectors of $\real^\infty$ and $f$ is affine on every 
simplex. Keeping this in mind, we have the following assumption 

\begin{assumption}
    We have constants $C_1, C_2 > 0$ s.t. for every $x \in M$ we have
    \begin{align*}
    \lVert d\tau|_x \rVert < C_1, \; \lVert d\tau^{-1}|_\tau(x) \rVert < C_2
    \end{align*} 
    where the operator norm is to be understood w.r.t. the metrics on $|K|$ and
    $M$.
\end{assumption}
{\color{red} TO BE DONE}





Take $\tau, \sigma \in K$ s.t. $\tau \leq \sigma$.  We define an extension 
operator $j^*_{\sigma, \tau}:W^*_\infty \rightarrow W^*_\infty $ which is 
bounded (cf \cite[p.191]{goldshtein}). 
\begin{definition}
    Let 
    \begin{align*}
    \theta = \{ \theta(\sigma) \in W^k_\infty(\sigma) | \sigma \in T\}
    \end{align*}
    be a collection of differential k-forms. We call $\theta$ S-form of degree
    k if we have for all for simplices
    $\mu \leq \sigma$ 
    \begin{align*}
    j^*_{\sigma,\mu}\theta(\sigma) = \theta(\mu).
    \end{align*}
    We denote with $S^k(K)$ the space of all S-forms of degree k over the chain
    complex $K$. 
For $\theta \in S^k(K)$ we define $d\theta \vcentcolon= \{ d\theta(\sigma) | 
\sigma \in K \} \in S^{k+1}(K)$. $S^*(K)$ is the resulting cochain complex.
\end{definition}

Using integration we can define define the homomorphism 
(see \cite[p.191]{goldshtein})
\begin{align*}
I: S_p^k(K) \rightarrow C_p^k(K), \; I(\theta)(\sigma) = 
\int_\sigma \theta(\sigma) \text{ for } \sigma \in K
\end{align*}
which induces an isomorphism on cohomology (see Thm. 1 in \cite{goldshtein}
and the proof thereof).

We say that $\omega \in W^k_{\infty,loc}(M)$ 
if $\omega|_A \in W^k_\infty(A) \text{for every} A \subseteq M \text{compact}$.
Then we define 
\begin{align*}
\varphi_\tau: W^k_{\infty,loc}(M) \rightarrow S^k(K), \;
\omega \mapsto \{ \tau|_\sigma^*\omega | \sigma \in K \}.
\end{align*}
This is a well-defined vector space isomorphism (\cite[p.191]{goldshtein}). This
way we can identify $W^k_{\infty,loc}(M)$ with $S^k(K)$. For S-forms of 
degree k we now define the norm
\begin{align*}
\lVert \theta \rVert _{S_p(K)}  \vcentcolon= \sum_{\sigma \in K} 
\lVert \theta(\sigma) \rVert _{W^k_\infty(\sigma)}.
\end{align*} 
$S^k_p(K)$ are the S-forms of degree $k$ s.t. this norm is finite. Using the
isomorphism $\varphi_\tau$ we now define 
$S^k_p(M) \vcentcolon= (\varphi_\tau)^{-1} S^k_p(K)$.

We then have $S^k_p(M) \subseteq W^k_p(M)$ and the inclusion induces an
isomorphism on cohomology \cite[Lemma 4, Corollary]{goldshtein}. 


Above, we defined the integral operator $I$ for $S^k_p(K)$ which can be 
therefore be applied on $S^k_p(M)$ as well. If we fix now a closed finite 
k-chain $\gamma$. Then $I(\cdot)(\gamma) = \int_\gamma$ becomes a functional on
$S^k_p(M)$, but is is a-priori not clear how to extend this to $W_p^k(M)$. 
We know that $\int_\gamma d\eta = 0$ for $\eta \in S^k_p(M)$ because 
otherwise $I$ would not induce a isomorphism on cohomology. We extend this now
by setting $\int_\gamma d\nu = 0$ for all $\nu \in W^{k-1}_p(M)$. 
We have to check whether this is consistent with the definition above. 
Let $\nu \in W_p^k(M)$ s.t. $d\nu \in S^k_p(M)$. Let $A \subseteq M$ be a 
bounded neighborhood of $\gamma$. We can then find 
$\tilde{\nu}$ s.t.  $\tilde{\nu} \in W_q(A)$ for any $q > 1$ and 
$d\tilde{\nu} = d\nu$ \cite[Thm 3.1.1]{schwarz}. We can then apply 
Stoke's theorem \cite[Thm. 9]{goldshtein_integration} to get  
$\int_\gamma d\nu = 0$. This shows consistency.

In the second part of \cite{goldshtein} they construct the operators
$\mathscr{R}$ and $\mathscr{A}$. The precise definition and construction of
these operators is not relevant for our purposes because we will only use
the following properties (cf. \cite[Thm.2]{goldshtein}).

\begin{theorem}\label{operators}
    Assume that the triangulation $\tau$ fulfills the GKS-condition.
    Then there exist linear mappings $\mathscr{R}: L^k_{1,loc} \rightarrow 
    L^k_{1,loc}$, $\mathscr{A}: L^k_{1,loc} \rightarrow L^{k-1}_{1,loc}$ 
    such that
    \begin{enumerate}
        \item $\mathscr{R}\omega - \omega = 
            d\mathscr{A}\omega + \mathscr{A}d\omega$ for 
            $\omega \in W^k_{1,loc}(M)$
        \item for any $1 \leq p \leq \infty$, 
            $\rop(W^k_p(M)) \subseteq S^k_p(M)$.
    \end{enumerate}
\end{theorem}

We can now use this operator $\rop$ to define $\int_\gamma \omega$ for closed
$\omega \in W^k_p(M)$ as
\begin{align*}
\int_\gamma \omega \vcentcolon= \int_\gamma \rop\omega.
\end{align*}
This is consistent because if $\omega \in S^k_p(M)$ closed then due to 
Thm. \ref{operators}
\begin{align*}
\int_\gamma \rop\omega = 
\int_\gamma \omega + d\mathscr{A}\omega + \mathscr{A}d\omega = 
\int_\gamma \omega.
\end{align*}

\subsection{Existence of a solution}

Returning now back to the problem, we are now able to proof existence of a 
solution. Take a closed cochain $F \in C^k_p(K)$ s.t. $F(\gamma) = C_0$ and 
$F(\partial d) = 0$ for (k-1)-chains $d$. Then we know from ??? that 
there exists a unique $[\theta] \in \mathscr{H}_p^k(K)$ s.t. 
$[I]([\theta]) = [F]$. Let us take $\eta \vcentcolon= 
\varphi_\tau^{-1} \theta$. Then $\int_\gamma \eta = C_0$ holds. If we now take
the Hodge decomposition
$L^k_2(M) = \bar{\mathfrak{B^k}} \bigoplus \mathcal{H}^k \bigoplus
\bar{\mathfrak{B^*_k}}$ and define $\omega$ as the projection of $\eta$ onto the
harmonic forms $\mathcal{H}^k$. Then we know that $d\omega = 0$, 
$\delta\omega = 0$ and $\text{tr}\,\omega = 0$. So we only have to show that
\begin{align*}
\int_\gamma \omega = C_0.
\end{align*}

We know from the Hodge decomposition that there exists a sequence
$(\phi_i)_{i\in \mathbb{N}} \subseteq L^{k-1}_2(M)$ s.t. 
$\omega = \eta - \lim_{i \rightarrow \infty} d\phi_i$.
Let now be $R>0$ large enough s.t. $\gamma \subseteq B_R$. Then we know that
$d W^{k-1}_2(B_R)$ is closed in $L_2^k(B_R)$. Therefore there exists 
$\phi_R \in  W^{k-1}_2(B_R)$ s.t. 
$\lim_{i \rightarrow \infty} d\phi_i|_{B_R} = d\phi_R$. So we have
$\omega|_{B_R} = \eta|_{B_R} - d\phi_R$ and 
\begin{align*}
\int_\gamma \omega = \int_\gamma \omega|_{B_R} = \int_\gamma \eta|_{B_R} = C_0.
\end{align*}
This proves existence.

\subsection{Existence of the solution}

The first step is to show that the cochain chosen in the proof of existence
is in fact unique if restricted to closed chains.

\begin{proposition}
    Let $\gamma$ be a $k$-chain s.t. the homology class $[\gamma]$ 
    is a generator of the homology group. Assume for some $C_0 \in \real$ 
    there exist cochains $F,G \in C^k_p(K)$ s.t.
    \begin{align*}
    F(\gamma) = C_0 \text{ and } F(\partial d) = 0 
    \text{ for all } (k-1) \text{-chains } d
    \end{align*}
    and the same for $G$. Then the restriction of $F$ and $G$ to closed 
    chains is the same.
\end{proposition} \label{uniqueness_cochain}
\begin{proof}
    Take any closed $k$-chain $c$. Because $\gamma$ is the generator of the 
    homology group we have $n \in \mathbb{Z}$ s.t. $[c] = [n \, \gamma]$
    where $[\cdot]$ is the corresponding homology class. That means that we have
    some $(k-1)$-chain $d$ s.t. $c = n \, \gamma + \partial d$. Using the 
    properties of $F$ and $G$,
    \begin{align*}
        F(c) = F(n \, \gamma + \partial d) = n F(\gamma) = n \, C_0.
    \end{align*}
    Because the same computation is valid for $G$ $F(c) = G(c)$ follows.
\end{proof}

\begin{theorem}
    Assume that a co-chain as in Prop. \ref{uniqueness_cochain} exists. 
    Then the solution of the problem is unique.
\end{theorem}
\begin{proof}
    
\end{proof}



\end{document}