\documentclass[12pt,a4paper]{article}
\usepackage[utf8]{inputenc}
\usepackage[english]{babel}
\usepackage{amsmath}
\usepackage{amsfonts}
\usepackage{amssymb}
\usepackage{amsthm}
\usepackage{xcolor}
\usepackage{csquotes}
\usepackage{enumerate}
\usepackage{faktor}
\usepackage{mathrsfs}
\usepackage{mathtools}
\usepackage{showlabels}

\usepackage{biblatex}
\addbibresource{bibliography.bib}
% \bibliography{bibliography}
% \bibliographystyle{ieeetr}

\numberwithin{equation}{subsection}

\newtheorem{lemma}{Lemma}[section]
\numberwithin{lemma}{subsection}
\newtheorem{corollary}[lemma]{Corollary}
\newtheorem{proposition}[lemma]{Proposition}
\newtheorem{theorem}[lemma]{Theorem}

\theoremstyle{definition}
\newtheorem{assumption}[lemma]{Assumption}
\newtheorem{definition}[lemma]{Definition}
\newtheorem{remark}[lemma]{Remark}
\newtheorem{problem}[lemma]{Problem}

\DeclareMathOperator*{\esssup}{ess\,sup}
\DeclareMathOperator{\curl}{curl}
\DeclareMathOperator{\diver}{div}
\DeclareMathOperator{\grad}{grad}
\DeclareMathOperator{\Ima}{im}
\DeclareMathOperator{\interior}{int}
\DeclareMathOperator{\supp}{supp}

\newcommand{\aop}{\mathscr{A}}
\newcommand{\alternating}[2]{ {\text{Alt}^{#1}\,#2} }
\newcommand{\integers}{\mathbb{Z}}
\newcommand{\smoothcompforms}[2]{C_0^\infty \Lambda^{#1}(#2)}
\newcommand{\lpcoho}{H^k_{p,dR}}
\newcommand{\naturalnum}{\mathbb{N}}
\newcommand{\norm}[2]{\lVert #1 \rVert_{#2}}
\newcommand{\omegabar}{\overline{\Omega}}
\newcommand{\real}{\mathbb{R}}
\newcommand{\rop}{\mathscr{R}} % short for R operator


\begin{document}

\section{Differential forms}

Before we define differential forms, let us start by revising some basics
from differential geometry. We follow the approach from 
\cite[Sec. II.5]{topology_and_geometry}. 
It should be noted that there are different definitions of tangent space, but
these lead to isomorphic notions 
(see e.g. \cite[Sec.\,1.B]{riemannian_geometry}).
Let $M$ be a smooth manifold without boundary.
For a point $p \in M$ and a neighorhood $U$ we call a function 
$f: U \rightarrow \real$ differentiable if for a local chart 
$\phi: U \rightarrow \real^k$ we have that $f \circ \phi^{-1}$ is differentiable
at $\phi(p)$.

Let $I \subseteq \real$ be an interval containing $0$ and 
$\gamma: I \rightarrow M$ be a differentiable curve with $\gamma(0) = p \in M$.
For a differentiable $f: U \rightarrow \real$ 
we define the the directional derivative 
$D_\gamma(f) \vcentcolon= \frac{d}{dt} f(\gamma(t)) |_{t=0}$.
We call the functional $D_\gamma: C^1(U) \rightarrow \real$ 
tangent vector. With the help of a local chart $\phi$ we can now 
express these derivations by 
\begin{align}
    D_\gamma(f) &=  \frac{d}{dt} f(\gamma(t)) |_{t=0}
    =  \frac{d}{dt} (f \circ \phi^{-1} \circ \phi)  (\gamma(t)) |_{t=0}
    \\ &= \sum\limits_{i=1}^k \frac{\partial (f \circ \phi^{-1})}{\partial x_i} 
        (\phi(p))
        \, \frac{d}{dt} [\phi^{-1}(\gamma(t))]_i |_{t=0}
    = (\sum\limits_{i=1}^k v_i  \frac{\partial}{\partial x_i} )(f).
\end{align}
In the last step, we abused the location by denoting by $f$ also the 
corresponding expression in local coordinates i.e. $f \circ \phi^{-1}$.
This way we identified $D_\gamma$ with $v \in \real^k$ 
by $v_i \vcentcolon= \frac{d}{dt} [\phi^{-1}(\gamma(t))]_i |_{t=0}$.
If we use a curve $\tilde{\gamma}$ s.t. $D_{\tilde{\gamma}} = D_\gamma$ and 
the same local chart $\phi$ then we obtain the same vector so this 
identification is unique.  




Our goal is to study the homogeneous magnetostatic problem on the exterior 
domain of a triangulated torus. That means 
that for the unbounded domain  $\Omega \subseteq \real^3$ we have
$\real^3 \setminus \Omega$ is a triangulated torus. We also need a 
piecewise straight (i.e. triangulated) closed curve around the torus.
%%% TBD: Picture
{\color{red} (TBD: Define the "triangulated torus" more rigorous)}

Let $B$ be a magnetic field on the domain $\Omega$.
We the have the following boundary value problem:

\begin{align}
    \curl \, B &= 0, \\ 
    \diver \, B  &= 0 \text{ in } \Omega \\
    B \cdot n &= 0 \text{ on } \partial \Omega \text{ and }\\
    \int_\gamma B \cdot dl &= C_0
\end{align}
where $n$ is the outward normal vector field on $\partial \Omega$ and 
$C_0 \in \real$. We want to prove existence and uniqueness of 
solutions. In order to do so we will need to introduce Sobolev spaces of 
differential forms and basics from
simplicial topology {\color{red} among other things...}

At first, let us introduce some basic notions about differential forms. 
We follow the brief introduction given by Arnold (cf. \cite[Sec. 6.1]{arnold}), 
but less details will be given. 

\subsection{Alternating maps}

Let $V$ be a real vector space with $\text{dim}\,V = n$.
Then $k$-linear maps are of the form
\begin{align*}
    \omega: \underbrace{V \times V \times ... \times V}_{k \text{ times}}
    \rightarrow \real
\end{align*}
that are linear in every component. We call a $k$-linear form 
\textit{alternating} if the sign switches when two arguments are exchanged i.e.
\begin{align*}
    \omega(v_1,...,v_i,...,v_j,...,v_k)
    = - \omega(v_1,...,v_j,...,v_i,...,v_k), \text{ for } 1\leq i < j \leq k,
    \quad v_1,...,v_k \in V.
\end{align*}
Denote the space of alternating maps by $\text{Alt}^k\,V$.

For $\omega \in 
\alternating{k}{V}$, $\mu \in 
\alternating{l}{V}$ we define the wedge product $\omega \wedge \mu \in 
\alternating{k+l}{V}$ 
\begin{align*}
    (\omega \wedge \mu) (v_1,...,v_k,v_{k+1},...,v_{k+l}) =
    \sum\limits_\pi
    \text{sgn}(\pi) \omega(v_{\pi(1)},...,v_{\pi(k)}) 
    \nu(v_{\pi(k+1)},...,v_{\pi(k+l)})
\end{align*}
where we sum over all permutations 
$\pi: \{1,...,k+l\} \rightarrow \{1,...,k+l\}$ 
s.t. $\pi(1) < ... < \pi(k)$ and $\pi(k+1) < ... < \pi(k+l)$.
This definition is not very intuitive.
{\color{red} TBD: Examples in 3D}.

Let $\{ u_i\}_{i=1}^n$ be any basis of $V$ and $\{ u^i\}_{i=1}^n$ the 
correspoding dual basis i.e. 
$u^i:V \rightarrow \real$, $u^i(u_j) = \delta_{ij}$. Then 
\begin{align*}
    \{u^{i_1} \wedge u^{i_2} \wedge ... \wedge u^{i_k} | \, 
    1 \leq i_1 < ... < i_k \leq n \}
\end{align*}
is a basis of $\alternating{k}{V}$. In particular, 
$\dim\, \alternating{k}{V} = \binom{n}{k}$.

Given a inner product $\langle\cdot, \cdot \rangle_V$ on $V$ we obtain an inner 
product on $\alternating{k}{V}$ by defining
\begin{align*}
    \langle u^{i_1} \wedge u^{i_2} \wedge... \wedge u^{i_k}, 
    u^{j_1} \wedge... \wedge u^{j_k} \rangle_{\alternating{k}{V}} 
    \vcentcolon= \det \left[ ( \langle u_{i_k}, u_{i_l} \rangle_V )_
    {1\leq k,l \leq n} \right] 
\end{align*}
which is then extended to all of $\alternating{k}{V}$ by linearity. 
We denote with $\lVert \cdot \rVert _\alternating{k}{V}$ the induced norm.
From this definition it follows directly that for a orthonormal basis 
$b_1$, ..., $b_n$ the corresponding basis 
$b^{i_1} \wedge b^{i_2} \wedge ... \wedge b^{i_k}$, 
$1\leq i_1 < ... < i_k \leq n$ is an orthonormal basis of $\alternating{k}{V}$.


$\alternating{n}{V}$ is one-dimensional and so can choose a basis by fixing 
a specific non-zero element.
We say that two orthonormal bases of $V$ 
have the same orientation if the change of basis has positive determinant. 
That divides the orthonormal bases into two classes with different orientation.
We choose one of these classes and call these orthonormal bases positively
oriented. In $\real^n$, the convention is to define the class as 
positively oriented which includes the standard orthonormal basis.
Take $\omega \in \alternating{n}{V}$. Then 
$\omega(b_1,...,b_n)$ is the same for any positively oriented orthornormal
basis. We now define the 
\textit{volume form} vol $\in \alternating{n}{V}$ by requiring
it to be $1$ on all positively oriented orthonormal bases. Using this volume 
form we can now define the \textit{Hodge star operator} 
$\star: \alternating{k}{V} \rightarrow \alternating{n-k}{V}$ via the property 
\begin{align*}
    \omega \wedge \mu = \langle \star\omega, \mu \rangle_{\alternating{n-k}{V}} 
    \text{vol}
    \quad \forall \omega \in \alternating{k}{V} , \, 
    \mu \in \alternating{n-k}{V}.
\end{align*}
The Hodge star is an isometry, we have $\star\star = (-1)^{k(n-k)}\text{Id}$ and 
\begin{align*}
    \omega \wedge \star\mu = \langle \omega, \mu \rangle_{\alternating{k}{V}} 
    \text{vol} \quad \forall \omega, \mu \in \alternating{k}{V}.
\end{align*}
In particular in $\real^3$, we have 
$\star\star = \text{Id}$ i.e. $\star$ is self-inverse.


$\alternating{0}{V}$ is just $\real$ by definition. So the Hodge star 
now gives us a natural isomorphism from $\alternating{n}{V} \rightarrow \real$.
We call the real number that is associated with an element of 
$\alternating{n}{V}$ \textit{scalar proxy}. Similarly we have that 
$\alternating{1}{V} = V'$, the usual dual space of $V$. Because $V$ is 
finite-dimensional we know that $V \cong V'$ using the Riesz representation.
Then with the help of the Hodge star we obtain a natural isomorphism 
from $\alternating{n-1}{V} \rightarrow V$. We call the vector corresponding
to a $1$- or $(n-1)$-alternating map \textit{vector proxy}. 

Note that for $n=2$ the situation 
is slightly ambiguous, see \cite[p.67]{arnold}. But this case will not be
relevant in this thesis.


Let us take a closer look at the case of $V= \real^3$ with the standard basis
vectors $e_1$, $e_2$ and $e_3$. Denote the resulting elements 
of the dual basis with $dx^i$. Then for some $v\in V$ we get the 
corresponding $1$-alternating map $v_1 \, dx^1 + v_2 \, dx^2 + v_3 \, dx^3 \in 
\text{Alt}^1$ and the $2$-alternating map
$v_1\, dx^2 \wedge dx^3 - v_2 \,dx^1 \wedge dx^3 + v_3 \,dx^1 \wedge dx^2 \in 
\text{Alt}^1$. For the scalar proxies, take $c \in \real$. Then 
$c \in \text{Alt}^0$ of course and because 
we have $\text{vol} = dx^1\wedge dx^2\wedge dx^3$ we get the corresponding 
$c \, dx^1\wedge dx^2\wedge dx^3 \in \text{Alt}^3$.


We can now use this relation to investigate the meaning of the wedge product
for vector proxies. Let us denote $\Phi^k$ as the isomorphism to the
$k$-alternating map with this scalar or vector as proxy. 
Let $v, w \in  \real^3$. Then 
\begin{align*}
    &\Phi^1 v \wedge \Phi^1 w = \Phi^2 (v\times w) \text{ and} \\
    &\Phi^1 v \wedge \Phi^2 w = \Phi^3 (v \cdot w).
\end{align*}
Note that for a $0$-alternating map (i.e. a scalar) 
the wedge product is nothing but the vector-scalar multiplication.

So in 3 dimensions, we can always relate the operations of differential forms
to the operations of the corresponding proxies. 


\subsection{Differential forms}

Before we define differential forms, let us start by revising some basics
from differential geometry. We follow the approach from 
\cite[Sec. II.5]{topology_and_geometry}. 
It should be noted that there are different definitions of tangent space, but
these lead to isomorphic notions 
(see e.g. \cite[Sec.\,1.B]{riemannian_geometry}).
Let $M$ be a smooth manifold without boundary.
For a point $p \in M$ and a neighorhood $U$ we call a function 
$f: U \rightarrow \real$ differentiable if for a local chart 
$\phi: U \rightarrow \real^k$ we have that $f \circ \phi^{-1}$ is differentiable
at $\phi(p)$. Note that we assume that our manifold is differentiable so 
all chart transitions are differentiable and thus this definition is independent
of the chosen chart.

Let $I \subseteq \real$ be an interval containing $0$ and 
$\gamma: I \rightarrow M$ be a differentiable curve with $\gamma(0) = p \in M$.
For a differentiable $f: U \rightarrow \real$ 
we define the the directional derivative 
$D_\gamma(f) \vcentcolon= \frac{d}{dt} f(\gamma(t)) |_{t=0}$.
We call the functional $D_\gamma: C^1(U) \rightarrow \real$ 
tangent vector.
Let us fix a chart $\phi: U \rightarrow \real^n$ and let us define 
\begin{align*}
    \frac{\partial}{\partial x_i}|_p: C^1(U) \rightarrow \real
    f \mapsto \frac{\partial (f \circ \phi^{-1})}{\partial x_i} (\phi(p)).
\end{align*}
The value depend in general on the chosen chart.
With the help of a local chart $\phi$ we can now 
express these derivations by 
\begin{align}
    D_\gamma(f) &=  \frac{d}{dt} f(\gamma(t)) |_{t=0}
    =  \frac{d}{dt} (f \circ \phi^{-1} \circ \phi)  (\gamma(t)) |_{t=0}
    \\ &= \sum\limits_{i=1}^k \frac{\partial (f \circ \phi^{-1})}{\partial x_i} 
        (\phi(p))
         \phi^{-1}\circ (\gamma)'_i (0)
    \\ &= \left( \sum\limits_{i=1}^k \frac{\partial}{\partial x_i}|_p 
          \phi^{-1}\circ (\gamma)'_i (0) \right) (f)
\end{align}
Thus we can express 
\begin{align*}
    D_\gamma = \sum\limits_{i=1}^k \frac{\partial}{\partial x_i}|_p 
    \phi^{-1}\circ (\gamma)'_i (0).
\end{align*}
So we have that 
$T_p M = \text{span}\, \left\{ \frac{\partial}{\partial x_i}|_p \right\}$.
We will show that this indeed a basis. 
Assume we have $\sum_{i=1}^n \lambda_i \frac{\partial}{\partial x_i}|_p = 0$
Then because $\phi_j \circ \phi^{-1} (x) = x_j$ for $x \in \phi(U)$ and 
$1 \leq j \leq n$. Then we have 
\begin{align*}
    0 = \left( \sum\limits_{i=1}^n \lambda_i \frac{\partial}{\partial x_i}|_p
        \right) (\phi_j)
    = \sum\limits_{i=1}^n \lambda_i \frac{\partial x_j}{\partial x_i}(\phi(p))
    = \lambda_j
\end{align*}
so $\left\{ \frac{\partial}{\partial x_i}|_p \right\}$ is 
linearly independent and thus a basis of $T_p M$. 

\begin{definition}[Differential forms]
    A differential $k$-form $\omega$ is a maps any point $p \in M$ to a 
    alternating $k$-linear mapping $\omega_p \in \alternating{k}{T_p M}$.
    We denote the space of differential $k$-forms on $M$ as $\Lambda^k M$.
\end{definition}

Let $T_p^* M$ be the dual space of $T_p M$ which is usually called 
\textit{cotangent space}.
As before let us choose a local chart $\phi: U \rightarrow \real^n$ with 
$p \in U$ and define $\frac{\partial}{\partial x_i}|_p$ as before. 
Because we know that this is a basis of $T_p M$ we denote the corresponding
dual basis as $dx^i$, $i = 1,...,n$. From the consideration about
alternating maps from section \ref{} we can now write any 
$\omega \in \Lambda^k M$ with 
\begin{align*}
    \omega_p = \sum\limits_{1\leq i_1 < ... < i_k \leq n} 
        a_{i_1,...,i_k}(p) dx^{i_1} \wedge dx^{i_2} \wedge ... \wedge dx^{i_k}
\end{align*}
with $a_{i_1,...,i_k}(p) \in \real$. The regularity of differential forms 
is then defined via the regularity of these coefficents i.e. we call 
a differential form smooth if all the $a_{i_1,...,i_k}$ are smooth 
and we call a differential form differentiable if all the $a_{i_1,...,i_k}$
are differentiable and so on. We denote the space $C^\infty \Lambda^k M$ the 
space of smooth differential $k$-forms and analogous for other regularity.

$\smoothcompforms{k}{M}$ are the smooth compactly supported differential forms
which will become very crucial later when we discuss Sobolev spaces 
of differential forms (see Sec. \ref{}). Note that the support is not 
defined via the coefficent functions as above, but directly on the manifold 
i.e. $\supp \omega = \overline{\{ p \in M \mid \omega_p \neq 0  \}} \subseteq M$
where the closure is w.r.t. the topology on $M$.

In order to define the Hodge star and an inner product on differential forms
we need that 
$T_p M$ is an inner product space.

A Riemannian metric gives us at every point $p \in M$ 
a symmetric, positive definite bilinear form 
$g_p: T_p M \times T_p M \rightarrow \real$. Additionally, a Riemannian metric 
is assumed to be smooth in the sense that for smooth vector fields 
$X$ and $Y$ we have $p \mapsto g_p(X,Y)$ is a smooth function. The degree 
of smoothness depends on the context. For example, for a differentiable 
manifold it is only required to be differentiable as well. 
{\color{red} More details... }
Manifolds on which a Riemannian metric is defined are called 
\textit{Riemannian manifolds}. The Riemannian metric provides us with the 
inner product on every tangent space $T_p M$. 

Now we will move on to differential forms on a smooth, oriented, Riemannian 
manifold $M$ of dimension $n$ 
with or without boundary. We denote the Riemannian metric by $g$.
%TBD: Lipschitz
Let $p \in M$ and $T_p M$ be the tangent space at the point $p$. 
Due to our assumptions on $M$, this is an inner product space of 
dimension $n$ and we can apply 
all of the constructions from the previous chapter
{\color{red} Are you sure?}. We have a volume form vol on $M$ so the 
previous constructions using the Hodge star operator are well-defined.  
\cite{riemannian_geometry}.



We define the Hodge star operator to differential forms 
$\star: \Lambda^k(\Omega) \rightarrow \Lambda^{n-k}(\Omega)$ simply by applying it 
pointwise. In order for 
the Hodge star to be well-defined the assumption of an orientation on our 
manifold is crucial. We do the same for the exterior product to get
$\wedge: \Lambda^k \times \Lambda^l \rightarrow \Lambda^{k+l}$. 


Next, we will define integration of an $n$-form over an $n$ dimensional 
manifold. At first, we do so for an open set $U \subseteq \real^n$.
This is the simplest example of an $n$-dimensional manifold where 
we only have one chart which is the identify and the local coordinates are 
just our standard coordinates. Let $\omega$ be an $n$-form on $U$ so we can 
write 
\begin{align*}
    \omega_x = f(x) dx_1 \wedge dx_2 \wedge ... \wedge dx_n
\end{align*}
for $x \in U$. We can now simply define 
\begin{align*}
    \int_U \omega = \int_U f(x) dx.
\end{align*}

With this definition at hand we can now extend this definition to 
any smooth oriented $n$-dimensional manifold $M$. As it is often done in 
differential geometry we will work locally first and then extend this 
construction globally by using a partition of unity.

Let $(U,\phi)$ be a chart on $M$ and assume $\supp \omega \subseteq U$. 
Then $(\phi^{-1})^* \omega$ is a $n$-form on $\phi(U) \subseteq \real^n$ and 
we can apply our prior definition. So now we just define 
\begin{align*}
    \int_M \omega \vcentcolon= \int_{\phi(U)} (\phi^{-1})^* \omega.
\end{align*}
It can be shown that this definition does not depend on the chart if 
we choose the atlas corresponding to the orientation of the manifold. 

Now let us move on to the global definition. Let $\{(U_i,\phi_i)\}_{i=1}^\infty$
be a oriented atlas and let $\{ \xi_i \}_{i=1}^\infty$ be a partition 
of unity subordinate to it. Then $\supp \xi_i \omega \subseteq U_i$ 
and we define 
\begin{align*}
    \int_M \omega \vcentcolon= \sum_{i=1}^\infty \int_M \xi_i \omega.
\end{align*} 
This definition is also independent of the chosen chart and partition of 
unity. We will omit the proof. 
{\color{red} In Bredon this is defined only for differentiable functions 
but Arnold uses this in a $L^2$ setting. So there must be a way to generalize 
the ideas of integrability etc. }

Now that we defined integration we will also introduce a natural way to 
differentiate differential forms. Let $\omega \in \Lambda^k (M)$ be given 
in local coordinates with some chart $(U,\phi)$ as above i.e.
\begin{align*}
    \omega_p = \sum\limits_{1\leq i_1 < ... < i_k \leq n} 
        a_{i_1,...,i_k}(p) dx^{i_1} \wedge dx^{i_2} \wedge ... \wedge dx^{i_k}
\end{align*}

Then we define the exterior derivative $d: Lambda^{k}(M) \rightarrow 
Lambda^{k+1}(M)$. By
\begin{align*}
    (d\omega)_p = \sum\limits_{1\leq i_1 < ... < i_k \leq n} \sum\limits_{i=1}^n
    \frac{\partial {a_{i_1,...,i_k}} }{\partial x_i}(\phi(p)) 
    dx^k \wedge dx^{i_1} \wedge dx^{i_2} \wedge ... \wedge dx^{i_k}
\end{align*}
As the other definitions above this is also independent of the chosen chart. 

If the manifold is oriented and we have thus a Hodge star operator.
Then we define the \textit{codifferential operator}
$\delta \vcentcolon= (-1)^{n(k-1)+1} \star d \star$ which is then 
an operator $\Lambda^{k}(M) \rightarrow 
Lambda^{k-1}(M)$.

The exterior derivative and the codifferential both require the differential 
form to be differentiable. Later we will extend this in weak sense so 
classical differentiabilty is no longer required (see \ref{}).

Then
we define the $L_p$-norm of a $k$-form $\omega$ for $1\leq p < \infty$
as (cf. \cite{goldshtein})
\begin{align*}
\lVert \omega \rVert _{L_p^k(\Omega)}\vcentcolon=
\left(\int_\Omega \lVert \omega(x) \rVert _{\text{Alt}^k}^p \,dx \right)^{1/p}
\end{align*}
%%% TBD: Is this independent of coordinates i.e. does this notation make sense?
and for $p=\infty$ as
\begin{align*}
\esssup_{x\in \Omega} \, \lVert \omega(x) \rVert _{\text{Alt}^k}.
\end{align*}
$L_p^k(\Omega)$ are the spaces of $k$-forms 
s.t. the corresponding $L_p$-norm is finite.
For $p=2$ we obtain a Hilbert space (cf. \cite[Sec. 6.2.6]{arnold}) 
with the $L_2$ inner product  
\begin{align}
    \langle \omega, \nu \rangle \vcentcolon= 
    \int_\Omega \langle \omega(x), \nu(x) \rangle _{\text{Alt}^k} \,dx
    = \int_\Omega \omega \wedge \star \nu
    \label{eq:def_inner_product} 
\end{align}
\begin{proposition}
    The Hodge star operator $\star:L^k_2(\Omega) \rightarrow L^{n-k}_2(\Omega)$ is a
    Hilbert space isometry.
\end{proposition}
\begin{proof}
    This follows directly from the definition of the inner product 
    (\refeq{eq:def_inner_product}) and the fact that $\star$ is an isometry 
    when applied to alternating forms $\text{Alt}^k$.
\end{proof}


Our next goal is to extend the exterior derivative $d$ 
of smooth differential forms in the weak sense (cf. \cite{goldshtein}). 
Let $\mathring{d}: L^k_2(\Omega) \rightarrow L^{k+1}_2(\Omega)$ be the exterior
derivative as an unbounded operator with domain 
$D(\mathring{d}) = \smoothcompforms{k}{\Omega}$ 
which are the smooth compactly supported differential forms $\varphi$ 
of degree $k$
with $\supp \varphi \subseteq \interior M$.

Note that when we talk about smoothness or regularity of differential forms we
always mean the regularity of the coefficents when the form is expressed 
via local charts. 

Analogous, let 
$\mathring{\delta}: L^k_2(\Omega) \rightarrow L^{k-1}_2(\Omega)$ be the 
codifferential operator $\mathring{\delta} \vcentcolon= 
(-1)^{n(k-1)+1}\star\mathring{d}\star$ 
also with domain $\smoothcompforms{k}{\Omega}$. 

Then the exterior derivative $ d\omega \in L^{k+1}_p(\Omega)$ is defined as
the unique $(k+1)$-form in $L^{k+1}_p(\Omega)$ s.t. 
\begin{align*}
\int_\Omega d\omega \wedge \star\phi = \int_\Omega \omega \wedge 
\star\mathring{\delta}\phi
\quad \forall \phi \in C_0^\infty \Lambda^{k}(\Omega).
\end{align*}
Just as in the usual Sobolev setting we define the following spaces:
\begin{align*}
W^k_p(\Omega) &= \left\{ \omega \in L^k_p(\Omega) \mid 
     d\omega \in L_p^{k+1}(\Omega) \right\}, \\ 
W^k_{p,loc}(\Omega) &= \left\{ \omega \ k \text{-form} \mid 
\omega|_A \in W^k_p(A) \text{ for every open } A \subseteq \Omega 
\text{ s.t. } \overline{A} \subseteq \Omega \text{ is compact} %TBD: Check that this is right
\right\}.
\end{align*}
For $\omega \in W^k_p(\Omega)$ for $p<\infty$ we define the norm 
\begin{align*}
\lVert \omega \rVert _{W^k_p(\Omega)} &\vcentcolon= 
\left( \norm{\omega}{L^k_p(\Omega)}^p 
    + \norm{d\omega}{L^k_p(\Omega)}^p \right)^{1/p}
\end{align*}
and for $p=\infty$
\begin{align*}
    \lVert \omega \rVert _{W^k_\infty} &\vcentcolon= 
    \max\left\{ \norm{\omega}{L^k_\infty},\, \norm{d\omega}{L^k_\infty}
    \right\}.
\end{align*}

\begin{remark} \label{rem:identification_sobolev_spaces}
    Throughout this thesis, we will mostly deal with open subdomains $\Omega 
    \subseteq \real^n$ with Lipschitz boundary. Then $\Omega$
    is a smooth submanifold of $\real^n$ and $\omegabar$ is a 
    Lipschitz manifold with boundary. If we now assume that
    $\Omega = \interior \omegabar$ then $W^k_p(\Omega)$ and $W^k_p(\omegabar)$
    are essentially the same. Take $\omega \in W^k_p(\Omega)$ and extend 
    it arbitrarily to $\overline{\omega} \in W^k_p(\omegabar)$. Then because 
    $\partial \Omega$ is a null set $\overline{\omega} \in L_2^k(\omegabar)$ and 
    because the definition of the exterior derivative uses only smooth 
    funtions with compact support contained in $\interior \omegabar = \Omega$ 
    we get that $d\overline{\omega} = d\omega \in L^k_2(\omegabar)$ (again 
    by choosing arbitrary values on the boundary). From now on we will in the
    assumed setting treat the spaces $W^k_p(\omegabar)$ and $W^k_p(\Omega)$ 
    as the same.
\end{remark}
    
\begin{definition}[$L^p$-cohomology]
    We define the following subspaces of $W^k_p(\Omega)$, $1\leq p \leq\infty$:
    \begin{align*}
        \mathfrak{B}_k &\vcentcolon= dW^{k-1}_p(\Omega) \text{ and} \\
        \mathfrak{Z}_k &\vcentcolon= \{ \omega \in W^k_p(\Omega)| 
        \, d\omega = 0\}.
    \end{align*}
    We call the $k$-forms in $\mathfrak{B}_k$ exact and the forms in 
    $\mathfrak{Z}_k$ closed. Because $d \circ d=0$ we always have  
    $\mathfrak{B}_k \subseteq \mathfrak{Z}_k$.
    Then we define the de Rham- or $L^p$-cohomology space $\lpcoho(\Omega)$ as 
    the quotient space
    \begin{align*}
        \lpcoho (\Omega) \vcentcolon= \faktor{\mathfrak{Z}_k}{\mathfrak{B}_k}.
    \end{align*}
\end{definition}
\vspace{0.5cm}
We want to examine the Hilbert space $L^k_2(\Omega)$ more closely
(see \cite[Sec. 6.2.6]{arnold} for more details).  
%%% TBD: This is only for bounded domains. What changes?
% Picard defines it as completion of smooth functions \cite[p.37]{picard}
We denote $H^k(d;\Omega) \vcentcolon= W^k_2(\Omega)$. If the domain is clear
we will leave it out. Note that the above definition of the exterior derivative
is in the Hilbert space setting equivalent to defining $d$ as the adjoint
of $\mathring{\delta}$. 



In order to extend $\mathring{\delta}$ as well, we will need the following

\begin{definition}[Codifferential operator]
    Analogous to the smooth case, 
    we define the \textit{codifferential operator} for any $k$ as an unbounded
    operator $\delta: L^k_2(\Omega) \rightarrow L^{k-1}_2$ as
    \begin{align*}
        \delta \vcentcolon= (-1)^{n(k-1)+1}\star d\,\star
    \end{align*}
    with domain
    \begin{align*}
        D(\delta) = \{ \omega \in L^k_2(\Omega) | \,
        \star\omega \in H^{n-k}(d) \} 
        =\vcentcolon H^k(\delta; \Omega).
    \end{align*}
\end{definition}

\begin{proposition}
    $\delta = \mathring{d}^*$ i.e. $\delta$ is the adjoint of $\mathring{d}$.
\end{proposition}
\begin{proof}
    Denote with
    $D(\mathring{d}^*) \subseteq L_2^{k-1}(\Omega)$ the domain of the adjoint.
    Now take $\omega \in H^k(\delta)$ and $\phi \in 
    \smoothcompforms{k}{\Omega}$.
    Then 
    \begin{align*}
        &\langle \delta \omega, \phi \rangle = 
        (-1)^{nk+1} \langle \star d\star \omega, \phi \rangle \\  
        &= (-1)^{nk+1} (-1)^{k(n-k)} \langle d\star \omega, \star\phi \rangle =
        (-1)^{nk+1} (-1)^{k(n-k)} 
            \langle \star \omega, \mathring{\delta}\star\phi \rangle \\
        &= (-1)^{nk+1} (-1)^{k(n-k)} (-1)^{n(n-k-1)+1}
            \langle \star \omega, \star\mathring{d}\star\star\phi \rangle \\
        &=(-1)^{n(n-1)+2} (-1)^{k(n-k)} \langle \omega, 
            \mathring{d}\star\star\phi \rangle\\
        &= \langle \omega, \mathring{d}\phi \rangle
    \end{align*}
    where we used repeatedly that $\star$ is an isometry and 
    $\star\star = (-1)^{k(n-k)}\text{Id}$. 
    This shows that $H^{k+1}(\delta) \subseteq 
    D(\mathring{d}^*)$ and that $\mathring{d}^* \omega = \delta \omega$. Now for 
    the other inclusion assume that $\omega \in D(\mathring{d}^*)$ and take 
    $\phi \in \smoothcompforms{n-k}{\Omega}$ arbitrary.
    \begin{align*}
        \langle \star\omega, \mathring{\delta}\phi \rangle = \pm 
        \langle \omega, \mathring{d}\star\phi \rangle = 
        \pm \langle \mathring{d}^* \omega, \star\phi \rangle = 
        \pm \langle \star\mathring{d}^* \omega, \phi \rangle.
    \end{align*} 
    Here we use $\pm$ to mean that we choose the sign correctly, s.t. all the
    operations are correct. Then by choosing the sign appropriately we find that
    $ \pm \star\mathring{d}^* \omega = d\star\omega$ and therefore 
    $\star\omega \in H^{n-k-1}(d)$ so we proved $D(\mathring{d}^*) \subseteq 
    H^{k+1}(\delta)$ and we are done.    
\end{proof}

In order to deal with the boundary of our domain we introduce Homogeneous
boundary conditions for these Sobolev spaces of differential forms.

\begin{definition}[Zero boundary condition] \label{def:zero_boundary_condition}
    We say that $\omega \in H^k(d;\Omega)$ has zero boundary condition if
    \begin{align*}
        \langle d\omega,\chi \rangle _{L^{k+1}_2(\Omega)}
        =  \langle \omega,\delta\chi \rangle _{L^k_2(\Omega)}
        \quad \forall \chi \in H^{k+1}(\delta;\Omega). 
    \end{align*}
    Denote $\mathring{H}^k(d;\Omega) \vcentcolon= \{ \omega \in H^k(d;\Omega)|\,
    \omega \text{ has zero boundary condition} \}$. 
\end{definition}
Of course we should justify why this is a reasonable definition. If $\Omega$ 
is lipschitz and bounded we have the integration by parts formula 
(cf. \cite[Thm.~6.3]{arnold})
\begin{align*}
    \int_\Omega d\omega \wedge \mu 
    = (-1)^k \int_\Omega \omega \wedge d \mu  + \int_{\partial\Omega} 
    \text{tr}\,\mu \wedge \text{tr}\,\omega \quad \text{for } \omega \in 
    H^1\Lambda^k(\Omega),\ \mu \in H^{n-k-1}(d;\Omega)  
\end{align*} 
where $H^1\Lambda^k(\Omega)$ are the differential forms with all coefficients
being in $H^1(\Omega)$ (here we mean just the standard Sobolev space). 
Let now $\omega \in \mathring{H}^k(d)$. Then if we use
the integration by parts formula and 
$\langle d\omega,\mu \rangle_{L^{k+1}(\Omega)} = 
\langle \omega, \delta \mu \rangle_{L^k(\Omega)}$ we get 
after some computation using the Hodge star
\begin{align*}
    \langle \text{tr}\,\omega, 
        \star\,\text{tr}\,\star\mu \rangle_{L^k(\Omega)}
    = 0 \quad \forall \mu \in H^1\Lambda^k(\Omega).
\end{align*}
The trace operator $\text{tr}: H^1\Lambda^k(\Omega) 
\rightarrow H^{1/2}\Lambda^k(\Omega)$ is surjective \cite[Thm.~6.1]{arnold}.
\begin{align*}
    \star\text{tr}\star H^1\Lambda^{k+1}(\Omega) 
    = \star \text{tr} H^1\Lambda^{n-k-1}(\Omega)
    = \star H^{1/2}\Lambda^{n-k-1}(\Omega)
    = H^{1/2}\Lambda^k(\Omega)
\end{align*}
is dense in $L^k_2(\Omega)$. Thus $\text{tr}\,\omega = 0$. So in the case 
of bounded Lipschitz domains this definition is reasonable. The reason why we
chose to define it as in Def.\,\ref{def:zero_boundary_condition} is that 
is easily extendible to unbounded domains and the regularity of the boundary
is not an issue. 


Then we define the spaces 
\begin{align*}
    H^k_0(d;\Omega) &\vcentcolon= \{ \omega \in H^k(d;\Omega) 
    | d\omega = 0 \} \\
    \mathring{H}^k_0(d;\Omega) &\vcentcolon= \{ \omega \in \mathring{H}^k(d;\Omega) 
    | d\omega = 0 \}
\end{align*}
i.e. the spaces of closed forms. We will use the analogous definition for 
$H^k_0(\delta;\Omega)$ and $\mathring{H}^k_0(\delta;\Omega)$ which we call 
coclosed forms. We then define the spaces of harmonic forms
\begin{align*}
    \mathring{H}^k_0(d,\delta;\Omega) \vcentcolon= 
    \{ \omega \in \mathring{H}^k(d;\Omega) 
    | \, d\omega = 0, \delta\omega = 0 \}.
\end{align*}
With this one can prove the Hodge decomposition (\cite[Lemma 1]{arnold})
\begin{align}
    L_2^k(\Omega) = \overline{d\mathring{H}^{k-1}(d)} \stackrel{\perp}{\oplus} 
    \mathring{H}^k_0(d,\delta) \stackrel{\perp}{\oplus} 
    \overline{\delta H^{k+1}(\delta)} \label{hodge_decomposition}
\end{align}
and furthermore for the closed and coclosed forms respectively,
\begin{align}
    \mathring{H}^k_0(d) &= \overline{d\mathring{H}^{k-1}(d)} 
    \stackrel{\perp}{\oplus}
    \mathring{H}^k_0(d,\delta) \label{decomposition_closed_forms} \\
    H^k_0(\delta) &= \overline{\delta H^{k+1}(\delta)} \stackrel{\perp}{\oplus}
    \mathring{H}^k_0(d,\delta). \label{decomposition_coclosed_forms}
\end{align}

\section{Simplicial topology}

Before we can reformulate the boundary value problem in the language of 
differential forms we have to 
introduce some things from simplicial topology. This material is taken from
\cite{topology_and_geometry} where a lot more details and results can be found.

\subsection{Simplicial complex}

\begin{definition}[Affine simplex]
    Let $x_0$, $x_1$, ..., $x_k \in \real^n$ be affine independent. Then 
    \begin{align*}
    [x_0,x_1,...,x_k] \vcentcolon= \text{conv}\{x_0,...,x_k\}
    \end{align*}
    is called an affine $k$-simplex. With $\text{conv}$ we mean the 
    convex combination.
\end{definition}
\noindent We will assume all simplices to be affine.
%%% TBD: Question: In the book this is defined in infinite dimensions. Does this
%%% lead to any problems?

\begin{definition}[Simplicial complex]
    A \textit{simplicial complex} $K$ is a collection of affine simplices s.t.
    \begin{enumerate}
        \item $\sigma \in K \Rightarrow$ any face of $\sigma$ is in $K$,
        \item $\sigma,\, \tau \in K \Rightarrow \sigma \cap \tau$  
                is either empty or a face of both $\sigma$ and $\tau$.
    \end{enumerate}
    We call $|K| \vcentcolon= \bigcup \{ \sigma | \sigma \in K\}$ the polyhedron of 
    $K$ and we denote the set of all $k$-simplices as $K^{(k)}$.
\end{definition}
For any topological space $X$ a homeomorphism 
$\tau: |K| \rightarrow X$ is called \textit{triangulation} of $X$.
Let $\{x_1,\, x_2,...\}$ be the vertices in the simplicial complex $K$.
We fix an ordering of the vertices for every simplex. 
That means for any $k$ every $k$ simplex $\sigma$ has
a designated representation in the form of
\begin{align*}
    \sigma = [x_{i_0},\, x_{i_1}, ...,x_{i_k}].
\end{align*} 

\begin{definition}[Star]
    The \textit{star} of a vertex $x_i$ is the collection of all simplices 
    $\sigma \in K$ s.t. $x_i$ is contained in $\sigma$. 
    
    We call a simplicial complex \textit{star-bounded} for some number $N$ 
    if the star of any vertex contains at most $N$ simplices.
\end{definition}
{\color{red} Does this imply quasi-regularity?}
Star-boundedness will be a crucial assumption later 
(see Assumption \ref{ass:GKS-assumption}). Now we will introduce
algebraic structures on the simplicial complex.

\begin{definition}[$k$-chain]
    Let $K$ be a simplicial complex. By $C_k(K)$ we will denote the 
    free abelian group on the $k$-simplices
    i.e. the abelian group of all formal finite sums
    \begin{align*}
        \sum_\sigma n_\sigma \sigma
    \end{align*}
    with $\sigma$ being $k$-simplices. The elements of $C_k(K)$
    (i.e. sums of the above form) are called \textit{$k$-chains}.
\end{definition}


\begin{definition}[Boundary]
    We have the 
    \textit{boundary operator} $\partial_k: C_k(K) \rightarrow C_{k-1}(K)$.
    We first define it
    for any simplex $[x_{i_0},\, x_{i_1},...,\, x_{i_k}]$ 
    \begin{align*}
        \partial[x_{i_0},\, x_{i_1},...,\, x_{i_k}] \vcentcolon=
        \sum\limits_{j=0}^k (-1)^j [x_{i_0},\, x_{i_1},...,\, 
        \hat{x}_{i_j},...,\, x_{i_k}]
    \end{align*}
    where $[x_{i_0},\, x_{i_1},...,\, \hat{x}_{i_j},...,\, x_{i_k}]$ is the
    simplex without vertex $x_{i_j}$. We then extend the definition
    of the boundary operator linearly to $k$-chains 
    $c = \sum_\sigma c_\sigma \sigma$ by
    \begin{align*}
        \partial c \vcentcolon= \sum\limits_\sigma c_\sigma \partial \sigma.
    \end{align*}
    By convention we define $\partial_0 = 0$. We will leave out the index $k$
    if it is clear from the context or irrelevant.
\end{definition}

A crucial property of the chain operator is the following.

\begin{proposition}
    $\partial \circ \partial = 0$.
\end{proposition}
\begin{proof}
    This can be proven by direct computation, analogous to 
    \cite[Chap.~4, Lemma~1.6]{topology_and_geometry}
\end{proof}

We can now generalize the notions of the chain groups by using factor groups.
Let $L \subseteq K$ be a simplicial complex which we will call simplicial
subcomplex. Then we define the factor group

\begin{align*}
    C_k(K,L) \vcentcolon= \faktor{C_k(K)}{C_k(L)}.
\end{align*}
We denote the elements of $C_k(K,L)$ i.e. the equivalence class 
by $\underline{c}$ for some $c \in C_k(K)$. 
Because $L$ is a simplicial complex we have $\partial_k c \in C_{k-1}(L)$ for
any $c \in C_k(L)$. Thus it induces a homomorphism on the factor groups denoted
by
\begin{align*}
    \underline{\partial}_k: C_k(K,L) \rightarrow C_{k-1}(K,L),\,
    \underline{\partial}_k \underline{c} \vcentcolon= \underline{\partial_k c}.
\end{align*}
Note that $C(K,L)$ generalizes the definition of $C_k(K,L)$ because we 
can choose $L = \emptyset$. Then the abelian group on $\emptyset$ is just 
$\{ 0 \}$ and thus $C_k(K,L) = C_k(K)$.

\begin{remark}
    The last equality is technically not precise of course. 
    However, for any group 
    $G$ with identity $\iota$ the resulting factor group
    $\faktor{G}{\{\iota\}}$ every equivalence class $\underline{g}$ would 
    only include $g$ itself so we can trivially identify $G$ with 
    $\faktor{G}{\{\iota\}}$ and we will do so from now on.
\end{remark}

We want to investigate how we can understand $\underline{c}$.

\begin{proposition}
    Let $c \in C_k(K)$ be arbitrary. Then
    \begin{align*}
        \underline{c} = \underline{0} 
        \leftrightarrow c = \sum\limits_{\tau \in L} n_\tau \tau.
    \end{align*}
    As usual the right hand side is a finite sum. 
    So for two $k$-chains $c = \sum_\sigma n_\sigma \sigma \in C_k(K)$ 
    and $c' =  \sum_\sigma n'_\sigma \sigma$ we have
    \begin{align}
        \underline{c} = \underbar{c}' 
        \leftrightarrow n_\sigma = n'_\sigma 
        \quad \forall \sigma \in K \setminus L. 
        \label{eq:equivalence_class_relative_chain_group}
    \end{align}
\end{proposition}
\begin{proof}
    $c \in \underline{0}$ is just equivalent to saying that 
    $c \in C_k(L)$ by definition of the factor group and thus we can write 
    $c = \sum\limits_{\tau \in L} n_\tau \tau$.
    (\ref{eq:equivalence_class_relative_chain_group}) follows from that 
    because $\underline{c} = \underbar{c}'$ is equivalent to
    $\underline{c - c'} = \underline{0}$.
\end{proof}

In this sense we can identify $C_k(K,L)$ with formal sums of 
simplices in $K^{(k)}\setminus L^{(k)}$. 

\subsection{Simplicial (Co-)Homology}

We call a $k$-chain $\underline{c} \in C_k(K,L)$ a \textit{$k$-cycle} if 
$\underline{\partial} \underline{c} = 0$ and we call $\underline{c}$ 
a \textit{$k$-boundary} if there exists $\underline{d}$ s.t. 
$\underline{c} = \underline{\partial} \underline{d}$. 
Let $Z_k(K,L) = \text{ker}\,\underline{\partial_k}$ 
and $B_k(K,L) = \text{im}\,\underline{\partial_{k-1}}$.
We can now define the simplicial homology groups of our simplicial complex.

\begin{definition}[Simplicial homology]
    The relative homology groups $H_k(K,L)$ are defined as
    \begin{align*}
        H_k(K)\vcentcolon= \faktor{Z_k(K,L)}{B_k(K,L)}.
    \end{align*}
\end{definition}

\begin{remark}
We call them relative homology group because define them on the 
for $C_k(K,L)$.
The standard homology groups $H_k(K)$ (i.e. taking $L = \emptyset$) 
are independent of the chosen triangulation of $|K|$
\cite[p.248]{topology_and_geometry} because they are isomorphic to the 
singular homology groups which only depend on $|K|$ 
\end{remark}

Let $G$ be any abelian group. Then we define the group 
of \textit{$k$-cochains} $C^k(K,L;G)$ by
\begin{align*}
    C^k(K,L;G) \vcentcolon= \text{Hom}(C_k(K,L),\,G)
\end{align*}
i.e. the group of all homomorphisms from $C_k(K,L)$ to $G$. 
We generally use the superindex $^k$ if something is related to 
cochains and the subindex $_k$ if it is related to chains. 
We now introduce an operator between the groups of cochains.

$C^k(K,L;G)$ can be identified with the cochains that vanish on $L$.

\begin{proposition}\label{prop:relative_cochains_vanish}
    Define
    \begin{align*}
        \tilde{C}^k(K,L;G) \vcentcolon= \{ F \in C^k(K;G) 
        \mid F(c) = 0 \quad \forall c \in C_k(L) \}.
    \end{align*}
    Then any $F \in \tilde{C}^k(K,L;G)$ induces and homomorphism
    $\underline{F} \in C^k(K,L;G)$ and $j:\tilde{C}^k(K,L;G) 
    \rightarrow C^k(K,L;G), \, F \mapsto \underline{F}$ is an isomorphism.
\end{proposition}
\begin{proof}
    Because $C_k(L) \subseteq \ker F$ is a subgroup we get the 
    induced homomorphism
    \begin{align*}
        \underline{F}: C_k(K,L) = \faktor{C_k(K)}{C_k(L)} \rightarrow
        \faktor{G}{{0}} = G.
    \end{align*}
    \textbf{$j$ is injective.} Let $\underline{F}(\underline{c})=0$ for every 
    $c \in C_k(K)$. Then $F(c) = \underline{F}(\underline{c}) = 0$ so $F = 0$.
    \textbf{$j$ is surjective.} Let $\varphi \in C^k(K,L;G)$ be arbitrary. 
    Then define $F \in C^k(K)$, $F(c) \vcentcolon= \varphi(\underline{c})$.
    We first have to show that $F \in \tilde{C}^k(K,L)$. For any $d \in C_k(L)$
    we get $\underline{d} = 0 \in C_k(K,L)$ and $F(d) = 0$.
    Then we have for any $\underline{c} \in C_k(K,L)$, 
    $\underline{F}(\underline{c}) = F(c) = \varphi(\underline{c})$ i.e. 
    $\underline{F} = \varphi$ which proves surjectivity.
\end{proof}

\begin{proposition}
    $j$ as defined in Prop.\,\ref{prop:relative_cochains_vanish} is a chain 
    map.
\end{proposition}
\begin{proof}
    Let $F \in \tilde{C}^k(K,L)$ be arbitrary. Then
    \begin{align*}
        \underline{\partial^k F}(\underline{c})
        = \partial^k F (c) = F(\partial_k F) 
        = \underline{F}(\underline{\partial_k c})
        = \underline{F}(\underline{\partial}_k \underline{c})
        = \underline{\partial}^k \underline{F} (\underline{c})
        %TBD: We need here that \tilde(C)^k is also a cochain complex
    \end{align*}
    i.e. $j \circ \partial^k = \underline{\partial}^k\circ j$ and $j$ is a 
    chain map.
\end{proof}

\begin{corollary}
    The induced map $[j]: H^k(K,L;G) \rightarrow \tilde{H}^k(K,L;G)$ is an 
    isomorphism.
\end{corollary}

\begin{definition}
    We define the operator $\delta: C^k(K;G) \rightarrow C^{k+1}(K;G)$ via
    \begin{align*}
        (\delta f) (c) \vcentcolon= f(\partial c).
    \end{align*}
    for a $(k+1)$-chain $c$.
    We call a cochain $f \in C^k(K;G)$ \textit{closed} if $\delta f = 0$ 
    and we call $f$
    \textit{exact} if there is a $g \in C^{k+1}(K;G)$ s.t. $f = \delta g$.
\end{definition}
\noindent We define the cohomology spaces analogous to homology spaces above.
\begin{definition}[Simplicial cohomology]
    Denote the closed $k$-cochains as $Z^k(K;G)$ and the 
    exact ones with $B^k(K;G)$. 
    We then define the \textit{simplicial cohomology groups}
    $H^k(K)$ as
    \begin{align*}
        H^k(K;G) \vcentcolon= \faktor{Z^k(K;G)}{B^k(K;G)}.
    \end{align*}
\end{definition}
Note that in the case of $G = \real$ this becomes a vector space.
We will later show that if we consider certain subspaces of cochains so called
\textit{p-summable} cochains (see Def.\,\ref{def:p_summable_cochains}) 
that the $L_p$-cohomology defined above and the
cohomology spaces of these $p$-summable cochains are isomorphic.


Now of course there is the question how the homology and cohomology groups 
are related to each other. This question is answered by the
\textit{universal coefficent theorem}. But before we can formulate it we have 
to introduce exact sequences.
\begin{definition}[Exact sequence]
    Let $(G_i)_{i\in \integers}$ be a sequence of groups and 
    $(f_i)_{i \in \integers}$ be a sequence of homomorphisms
    $f_i: G_i \rightarrow G_{i+1}$. Then this sequence of homomorphisms is
    called \textit{exact} if $\text{im}\,f_{i-1} = \text{ker}\,f_i$.
\end{definition}

The universal coefficent theorem in the case of simplicial homology states
that the sequence 
\begin{align}
    0 \rightarrow \text{Ext}(H_{k-1}(K),G) \rightarrow 
    H^k(K;G) \xrightarrow{\beta} \text{Hom}(H_k(K),G) 
    \rightarrow 0 \label{eq:univeral_coefficient_theorem}
\end{align}
is exact. 
$\beta$ is defined via $\beta([F])([c]) \vcentcolon= F(c)$.
The definition of Ext can be found in \cite{topology_and_geometry},
but it does not matter for our purpose because from now on we will assume
$G = \real$ and
$\text{Ext}(H_{k-1}(K),\real) = 0$. This follows from the fact that 
$\real$ is a divisible and hence injective abelian group. The definition 
these terms and the connections used can also be found in 
\cite[Sec.\,V.6]{topology_and_geometry}. However, we will not dwelve into the 
algebraic background further. We can conclude from the exactness of the 
above short sequence that $\text{ker}\,\beta = 0$ and 
$\text{im}\,\beta = \text{Hom}(H_k(K),\real)$. So $\beta$ is a isomorphism.

\subsection{Existence of uniqueness of cochain}



As an application, we will show the following proposition which will be used
later to show existence and uniqueness of a 
solution of the magnetostatic problem.

\begin{proposition}\label{prop:uniqueness_cochain}
    Assume that $H_1(K,L) = \integers [\gamma]$ i.e. the homology class of the 
    closed $1$-chain $\gamma$ is a generator of the first homology group.
    Then we have the following:
    \begin{enumerate}[(i)]
        \item For any $C_0 \in \real$ there exists a closed $1$-chain 
            $F \in Z^1(K,L)$ with $F(\gamma) = C_0$,
        \item any other $G \in Z^1(K,L)$ with $G(\gamma) = C_0$ 
            is in the same cohomology class i.e. $[F] = [G]$.
    \end{enumerate}
\end{proposition}
\begin{proof}
    \textbf{Proof of (i)} %TBD: This could be wrong
    Because $[\gamma]$ is a generator of the homology group we  obtain a 
    homomorphism $\hat{F} \in \text{Hom}(H_1(K,L),\real)$ by fixing
    $\hat{F}([\gamma]) = C_0$. This determines the other values.
    Then we know from (\ref{eq:univeral_coefficient_theorem}) that there exists
    a $[F] \in H^1(K,L)$ with $\beta([F]) = \hat{F}$ because $\beta$ is a 
    isomorphism. So we obtain
    \begin{align*}
        F(\gamma) = \beta([F])([\gamma]) = \hat{F}([\gamma]) = C_0.
    \end{align*}

    \textbf{Proof of (ii)} %TBD: This could be wrong
    Take $[c] \in H_1(K,L)$ arbitrary. Then there exists  $n \in \integers$ s.t.
    $[c] = n [\gamma]$.
    Using $\beta$ from (\ref{eq:univeral_coefficient_theorem})
    We have
    \begin{align*}
        \beta([F])([c]) = \beta([F])(n [\gamma]) 
        = n \beta([F])([\gamma]) = n \, F(\gamma) = n \, G(\gamma) = 
        \beta([G])([c])
    \end{align*}
    and thus $\beta([F]) = \beta([G])$. Because $\beta$ is an isomorphism
    we arrive at $[F] = [G]$.
\end{proof}

The following lemma will be needed later in Sec.\,\ref{} to prove existence 
and uniqueness of a solution of the magnetostatic problem. It is now 
useful to think of $K$ as a triangulation of an exterior domain and $L$ 
be the triangulation of the exterior of some ball $\real^3\setminus B_R$ with 
$R > 0$ large enough. Then we assume for the simplicial subcomplex $L$
that its geometric realization $|L|$ is homeomorphic to $\real^3\setminus B_R$.

First we need to understand how the zero-th homology group looks like. 
When we say path-connected we mean in the sense of simplicial topology i.e. 
we can connect all vertices with a $1$-chain.

\begin{lemma}\label{lem:zeroth_homology_group}
    Let $K$ be a path-connected simplicial complex. Then the homology class 
    of every vertex $[x_i] \in H_0(K)$ is a generator of the homology group.  
\end{lemma}
\begin{proof}
    The proof works exactly as the proof of Theorem IV.2.1 in 
    \cite{topology_and_geometry} where we use the notion of path-connected
    on simplicial complexes.
\end{proof}

\begin{lemma}\label{lem:inclusion_zeroth_homology}
    Let $K$, $L$ be simplicial complexes with $L \subseteq K$ and 
    assume they are path-connected  Then
    the inclusion $\iota: C_0(L) \hookrightarrow C_0(K)$ induces an isomorphism
    on homology.
\end{lemma}
\begin{proof}
    This follows immediately from Lemma \ref{lem:zeroth_homology_group}.
\end{proof}


\begin{lemma}\label{lem:isom_chains_and_relative_chains}
    Let $K$ and $L$ be as in the previous lemma. Assume additionally
    that $|L|$ is homeomorphic to $\real^3 \setminus B_R$.
    Then the natural surjection
    $j: C_1(K) \rightarrow C_1(K,L)$ induces an isomorphism on homology.
\end{lemma}
\begin{proof}
    We have the exact sequence 
    \begin{align*}
        &H_1(L) \xrightarrow{[\iota_1]} H_1(K) \xrightarrow{[j_1]} H_1(K,L) \\
        \xrightarrow{[\partial_1]} &H_0(L) \xrightarrow{[\iota_0]} H_0(K) 
            \xrightarrow{[j_0]} H_0(K,L).
    \end{align*}
    We know that $H_1(L) \cong H_1(|L|) \cong H_1(\mathbb{S}^2) = 0$. 
    Because $[\iota_0]$ is an isomorphism due to 
    Lemma\,\ref{lem:inclusion_zeroth_homology} $\Ima [\partial_1] = 0$. 
    Thus, we get the exact sequence 
    \begin{align*}
        0 \xrightarrow{[\iota_1]} H_1(K) \xrightarrow{[j_1]} H_1(K,L)
        \xrightarrow{[\partial_1]} 0
    \end{align*}
    and so $[j_1]$ is a isomorphism.
\end{proof}

Now using all these lemmas and results we can now prove the following
important lemma that will be needed later in Sec.\,\ref{}

\begin{lemma}\label{lem:existence_vanishing_cochain}
    Let $K$ and $L$ be as in the previous lemma and assume that 
    $H_1(K) = \integers [\gamma]$. Then there exists a cochain 
    $F \in \tilde{C}^1(K,L)$ s.t. $F(\gamma) = C_0$.
\end{lemma}
\begin{proof}
    We know from Lemma\,\ref{lem:isom_chains_and_relative_chains}
    that $[j]:H_1(K) \rightarrow H_1(K,L)$ is a isomorphism. Thus, we have
    $[j]([\gamma]) = [\underline{\gamma}] \in H_1(K,L)$ is a generator. 
    We can now apply Prop.\,\ref{prop:uniqueness_cochain} to get a 
    $\tilde{F} \in C^1(K,L)$ s.t. $\tilde{F}(\underline{\gamma}) = C_0$.
    Now we continue with Prop.\,\ref{prop:relative_cochains_vanish} to 
    get that there exists a unique $F \in \tilde{C}^1(K,L)$ s.t. 
    $jF = \underline{F} = \tilde{F}$ and we arrive at 
    \begin{align*}
        F(\gamma) = \underline{F}(\underline{\gamma}) 
        = \tilde{F}(\underline{\gamma}) = C_0.
    \end{align*}
\end{proof}



\section{Isomorphism of Cohomology}

In order to show existence and uniqueness of solutions of the magnetostatic 
problem we use a lot of results and tools from \cite{goldshtein}. 
In the diploma thesis of Nikolai
Nowaczyk \cite{nowaczyk}, which mostly is based on this paper, 
many additional details can be found. The results which are important for 
our progress will be presented in the
next section. It should be noted that even though the results in 
\cite{goldshtein} are
proved explicitely for smooth manifolds without boundary the results can be 
extended to Lipschitz manifolds with boundary (see the proof of Theorem 2 and 
the remark at the end in \cite{goldshtein}). Therefore, we can apply the result
to our case.

In the first section we will introduce S-forms which can be seen as a
intermediate object between differential forms and we will go over the 
basic results from \cite{goldshtein} how these S-forms are connected with 
cochains on our simplicial complex $K$. Then we will look at how S-forms 
and differential forms are connected. At last, we will introduce two  
operators and their properties which are introduced in \cite{goldshtein}
and will prove to be very useful later on.

Because $\omegabar$ from our problem is itself a polyhedron we can
assume that $\omegabar$ and $|K|$ are equal as subsets of $\real^n$ and we can
simply use the identity as triangulation.
However, we will use different metrics on $|K|$ and $\omegabar$. 
We use the Euclidian metric on 
$\omegabar$ and we use the standard simplicial metric on $|K|$ (cf. 
\cite[p.191]{goldshtein}). This metric is defined as follows:

Choose some numbering of the vertices $\{ x_1,\, x_2, ... \}$ and
take $f: |K| \rightarrow \ell^2$ where $\ell^2$ is the 
Hilbert space of real-valued square-summable sequences s.t. $f(x_i) = e_i$ 
with $e_i \in \ell^2$ being the standard unit vectors and $f$ is affine on 
every simplex. This mapping is unique.%%%TBD: Proof uniqueness%%%%

Then we define the metric on $|K|$ as the pullback $g_S = f^*g$ 
where $g$ is the standard metric in $\ell^2$. Let $\langle \cdot , 
\cdot \rangle$ be the standard scalar product on $\ell^2$. Then for $x \in |K|$ 
and $\sum_{i=1}^n v_i \frac{\partial}{\partial x_i}, \; 
\sum_{j=1}^n w_j \frac{\partial}{\partial x_j} \in T_x |K|$ we have 
\begin{align*}
g_S|_x\left(\sum_{i=1}^n v_i \frac{\partial}{\partial x_i}, 
\sum_{j=1}^n w_j \frac{\partial}{\partial x_j}\right) &= 
\left\langle \sum_{k=1}^\infty \sum_{i=1}^n v_i 
\frac{\partial f_k}{\partial x_i} (x)
\frac{\partial }{\partial y_k}, 
\sum_{l=1}^\infty \sum_{j=1}^n w_j \frac{\partial f_l}{\partial x_j} (x)
\frac{\partial }{\partial y_l} \right\rangle \\   
&= \sum_{i,j=1}^n \sum_{k,l=1}^\infty v_i \frac{\partial f_k}{\partial x_i} (x)
w_j \frac{\partial f_l}{\partial x_j} (x) 
\left\langle \frac{\partial }{\partial y_k}, \frac{\partial }{\partial y_l} 
\right\rangle\\
&= \sum_{i,j=1}^n \sum_{k=1}^\infty v_i \frac{\partial f_k}{\partial x_i} (x)
w_j \frac{\partial f_k}{\partial x_j} (x)\\
&= \sum_{i,j=1}^n \sum_{k=1}^\infty v_i w_j \left( Df(x)^T Df(x) \right)_{ij} \\
&= v^T Df(x)^T Df(x) w = \left\langle Df(x) v, Df(x) w \right\rangle,
\end{align*}
where $D$ denotes the Jacobian. 
{\color{red} (TBD: This Jacobian as written here would technically be in 
$\real^{\infty \times n}$. Only finitely many lines are non-zero though, 
but this is not quite rigorous yet. )}
%%% TBD: This is only well-defined if x is in the interior of a full simplex.

We have two crucial assumptions on the triangulation for the result to hold 
(cf. \cite[p.194]{goldshtein}). We summarize them under 
\textit{GKS-condition} named after the three authors of \cite{goldshtein}.

\begin{assumption}[GKS-condition] \label{ass:gks_condition}
    We will assume the following on the simplicial complex $K$ 
    and the triangulation $\tau$:
    \begin{enumerate}[(i)]
    \item The star of every vertex in $K$ contains at most $N$ simplices.
    \item For the differential of $\tau$ we have constants 
        $C_1, C_2 > 0$ s.t.
        \begin{align*}
        \lVert d\tau|_x \rVert < C_1, \; 
        \lVert d\tau^{-1}|_{\tau(x)} \rVert < C_2,
        \end{align*}
        where $d$ denotes the differential in the sense of differential 
        geometry and the norm is the operator norm w.r.t. the metrics on $|K|$ 
        and $\omegabar$.
    \end{enumerate}
\end{assumption}
The first assumption is equivalent to every vertex being contained in
at most $N$ simplices, which is fulfilled if we have a shape regular mesh.\par
{\color{red} This has to be shown.}
%%% TBD: Reference or proof or sth%%%

Because $\tau$ is just the identity in our case 
the second assumption says that for every 
$x \in |K|$
\begin{align*}
\sup\limits_{v \neq 0} \frac{\lVert v \rVert}{\sqrt{g_S|_x(v,v)}} =
\sup\limits_{v \neq 0} \frac{\lVert v \rVert}{\lVert Df(x)v\rVert} < C_1
\end{align*}
and analogously
\begin{align*}
    \sup\limits_{v \neq 0} \frac{\lVert Df(x)v\rVert}{\lVert v \rVert} < C_1.
\end{align*}
%%% TBD: Give more details and interpretation

\subsection{S-forms}
\begin{definition}[Induced map]
    Let $V$ and $W$ be real vector spaces, $X \subseteq V$, $Y \subseteq W$ be 
    subspaces. For a linear map $L: V \rightarrow W$ with $L(X) \subseteq Y$ 
    we define the induced map
    \begin{align*}
        [L]: \faktor{V}{X} \rightarrow \faktor{W}{Y},\,
        [v] \mapsto [Lv].
    \end{align*}     
\end{definition}
It is easy to check that the induced map is well-defined using the
definition of quotient space. %% Proven on the 27.11.2022


The first isomorphism is induced from a linear mapping
from the so called \textit{S-forms} 
$S_p^k(K)$ to 
\textit{p-summable $k$-cochains} $C_p^k(K)$ which will both 
be defined next.

\begin{definition}
    We define the following norm of a $k$-cochain $f$
    \begin{align*}
        \norm{f}{C_p^k(K)} \vcentcolon= 
        \left( \sum\limits_{\sigma \in K^{(k)}} |f(\sigma)|^p \right)^{1/p}.
    \end{align*} % TBD: Define K^(k), this is a convenient notation
    and the space of \textit{p-summable k-cochains}
    \begin{align*}
    C_p^k(K) \vcentcolon= \{f \in C^k(K) | \,  
    \norm{f}{C_p^k(K)} < \infty \}.
    \end{align*}
\end{definition}
Take $\tau, \sigma \in K$ s.t. $\tau$ is a face of $\sigma$ which we write as
$\tau < \sigma$. We need a restriction operator 
$j^*_{\tau,\sigma}:W_{\infty,loc}^k(\sigma) \rightarrow W_{\infty,loc}^k(\tau)$.
This is done by extending $\omega \in W_{\infty,loc}^k(\sigma)$ first 
to some $\tilde{\omega}\in W_{\infty,loc}^k(U)$ with 
$U$ an open neighborhood of $\sigma$ in the affine hull of $\sigma$. Then 
$\tau \subseteq U$ and we apply a restriction operator $j^*_{\tau,U}$
and define $j^*_{\tau,\sigma} \omega \vcentcolon= j^*_{\tau,U}\tilde{\omega}$. 
This restriction operator is then well defined, 
bounded and independent of the chosen 
extension $\tilde{\omega}$ (cf. \cite[p.191]{goldshtein}). It should be 
emphasized that this restriction only works for $W^k_\infty$ and fails for 
$W^k_p$, $p<\infty$. 

\begin{definition}[S-forms]
    Let 
    \begin{align*}
    \theta = \{ \theta(\sigma) \in W^k_\infty(\sigma) | \sigma \in K\}
    \end{align*}
    be a collection of differential $k$-forms. We call $\theta$ \textit{S-form 
    of degree
    $k$} if we have for all simplices
    $\mu <\sigma$ 
    \begin{align*}
    j^*_{\sigma,\mu}\theta(\sigma) = \theta(\mu).
    \end{align*}
    We denote with $S^k(K)$ the space of all S-forms of degree $k$ 
    over the chain complex $K$. 
    For $\theta \in S^k(K)$ we define $d\theta \vcentcolon= \{ d\theta(\sigma) | 
    \,\sigma \in K \} \in S^{k+1}(K)$. $S^*(K)$ is the resulting cochain complex.
    % TBD: Did i define what a cochain complex is?
\end{definition}
For $\theta \in S^k(K)$ we now define the norm
\begin{align*}
\lVert \theta \rVert _{S_p(K)}  \vcentcolon= \left( \sum_{\sigma \in K} 
\lVert \theta(\sigma) \rVert _{W^k_\infty(\sigma)}^p \right)^{1/p}.
\end{align*} 
$S^k_p(K)$ are the S-forms of degree $k$ s.t. this norm is finite.

Let $L \subseteq K$ be a simplicial subcomplex. Then we define the 
restriction operator 
\begin{align*}
    j^*_{K,L}: S^k(K) \rightarrow S^k(L),
    \{ \theta(\sigma) \mid \sigma \in K \} 
    \mapsto \{ \theta(\sigma) \mid \sigma \in L \}.
\end{align*}
Then we define $S^k(K,L) \vcentcolon= \ker j^*_{K,L}$ and analogously
$S^k_p(K,L)$. So these are the S-forms that vanish on a simplicial subcomplex.
As in the case of cochains (compare Sec.\,\ref{}) this generalizes 
the first definition because we get $S^k(K,\emptyset) = S^k(K)$.

The integration of an S-form $\theta \in S^k(K)$ over a chain
$c = \sum n_\sigma \sigma \in C_k(K)$ is then simply defined as 
\begin{align*}
    \int_c \theta 
    \vcentcolon= \sum\limits_\sigma n_\sigma \int_\sigma \theta(\sigma).
\end{align*}
We also write $I(\theta)(c) \vcentcolon= \int_c \theta$.
Then the integration mapping 
$I: S^k(K,L) \rightarrow C^k(K,L)$ is a homomorphism
(see \cite[p.191]{goldshtein}).
The reason why S-forms are a useful concept is because they provide an 
intermediate step between differential forms and cochains.

With the exterior derivative $d$ on S-forms as defined above we define 
\begin{align*}
    \mathscr{Z}^k(K,L) &\vcentcolon= \{ \theta \in S^k(K,L) 
        | \, d\theta = 0 \} \\
    \mathscr{B}^k(K,L) &\vcentcolon= dS^{k-1}(K,L)
\end{align*}
and then the resulting cohomology space
\begin{align*}
    \mathscr{H}^k(K,L) \vcentcolon= 
    \faktor{\mathscr{Z}^k(K,L)}{\mathscr{B}^k(K,L)}.
\end{align*}
We also define the in the case of $p$-summable S-forms
\begin{align*}
    \mathscr{Z}_p^k(K,L) &\vcentcolon= \{ \theta \in S_p^k(K,L) | \, d\theta = 0 \} \\
\end{align*}
as well $\mathscr{B}_p^k(K,L)$ and $\mathscr{H}_p^k(K,L)$ analogously.

Then we have that the integration mapping
$I: S_p^k(K,L) \rightarrow C_p^k(K,L)$ 
induces an isomorphism on the cohomologies i.e. 
$[I]: \mathscr{H}_p^k(K,L) \rightarrow 
\tilde{H}_p^k(K,L)$ is an isomorphism of vector 
spaces (see \cite[Thm.\,1]{goldshtein} and the proof thereof).



\subsection{Isomorphism between cohomologies of S-forms and $L^p$-cohomology}


The next step is to obtain an isomorphism between the cohomology 
of S-forms $\mathscr{H}_p^k(K)$ and the $L_p$ cohomology 
$\lpcoho(\omegabar)$. {\color{red} Do I really do that?}

In order to achieve that we need a some connection between differential forms
and S-forms. Take $\omega \in W_{\infty,loc}^k(\omegabar)$. Using the analogous
reasoning as above, we use the restriction operators $j^*_{\sigma,\tau}$ to 
restrict $\omega$ to the simplices. Denote the resulting 
forms as $\omega(\sigma) \in W^k_{\infty,loc}(\sigma)$. So we obtain an 
S-form $\{ \omega(\sigma) | \, \sigma \in K \}$. This way we constructed 
an operator 
\begin{align}
    \varphi: W_{\infty,loc}^k(\omegabar) \rightarrow S^k(K),\ \omega \mapsto 
    \{ \omega(\sigma) | \, \sigma \in K \} \label{eq:forms_Sforms_isom}
\end{align}
This 
operator $\varphi$ is an isomorphism \cite[Lemma~1]{goldshtein}. 
To also obtain corresponding forms to the $p$-summable S-forms we simply 
define $S^k_p(\omegabar) \vcentcolon= \varphi^{-1}S^k_p(K)$.
It can be shown that $S^k_p(\omegabar) \subseteq W^k_p(\omegabar)$.
Now if we have some $\omega \in S^k_p(\omegabar)$ then we can look at 
two possible norms. From the inclusion we know that 
$\norm{\omega}{W^k_p(\omegabar)} < \infty$. But we can also use the norm 
induced by $\varphi$ denoted by 
\begin{align}
    \norm{\omega}{S^k_p(\omegabar)} 
    \vcentcolon= \norm{\varphi \omega}{S^k_p(K)}. \label{eq:induced_norm}
\end{align}
Let us briefly show that this is indeed a norm.
\begin{proposition}
    Let $K$ be a simplicial complex and $\omegabar = |K|$. 
    Then $\norm{\cdot}{S^k_p(\omegabar)}$ 
    is a norm and $\varphi$ as defined at (\ref{eq:forms_Sforms_isom})  
    is an isometry.
\end{proposition}
\begin{proof}
    Subadditivity and absolute homogeneity follow directly from the linearity 
    of $\varphi$ and the corresponding property of $\norm{\cdot}{S^k_p(K)}$.
    For the positive definiteness 
    \begin{align*}
        \norm{\omega}{S^k_p(\omegabar)} = 0 
        \Leftrightarrow \norm{\varphi \omega}{S^k_p(K)}
        \Leftrightarrow \varphi \omega = 0
        \stackrel{\varphi \text{ isom.}}{\Leftrightarrow} \omega = 0.
    \end{align*}
    $\varphi$ is an isometry by construction of the norm.
\end{proof}
We then also have that the inclusion operator $\iota: S^k_p(\omegabar) 
\hookrightarrow W^k_p(\omegabar)$
is bounded (\cite[Lemma\,4]{goldshtein}). 

\begin{remark} \label{rem:local_construction_regularization}
    In the paper \cite{goldshtein}, no difference is made between the spaces 
    $S^k_p(\omegabar)$ and $S^k_p(K)$ because they can be identified with each 
    other via $\varphi$. However, we will follow the approach from Nowaczyk's 
    thesis \cite{nowaczyk} to treat them separately because it is more precise 
    and less confusing.
\end{remark}

We can now extend our definition of integration on a simplex to any differential
form $\omega \in W_{\infty,loc}^k(\omegabar)$ thanks to the 
well-definedness of the restriction operators. We can do this by first 
applying $\varphi$ which means nothing but applying these restrictions to obtain
an S-form $\varphi \omega \in S^k(K)$ for which we have defined the integration 
above. Note that this is exactly how integration is defined for continuous 
differential forms. The only difference is that we had to give an additional 
argument because the restriction can not be applied directly as in the 
continuous case. We will sometimes just write $I(\omega) \vcentcolon= 
I(\varphi\omega)$ or $\int_\sigma \omega = \int_\sigma (\varphi\omega)(\sigma)$
for any simplex $\sigma \in K$.

Now we also want to introduce a regularization operator in order to 
obtain 
a form in $S^k_p(\omegabar)$ from form $W^k_p(\omegabar)$. 
This is done 
using two operators $\rop: L^k_{1,loc}(\omegabar) \rightarrow 
L^k_{1,loc}(\omegabar)$ and $\aop: L^k_{1,loc}(\omegabar) \rightarrow 
L^{k-1}_{1,loc}(\omegabar)$.
We will not go over their construction but instead we will only use
some properties that we collect in the following theorem
(cf. \cite[Thm.2]{goldshtein}).

\begin{theorem}\label{thm:operators}
    Assume that the triangulation $\tau$ fulfills the GKS-condition 
    defined at Def.~\ref{def:GKS_condition}.
    Then there exist linear mappings 
    $\mathscr{R}: L^k_{1,loc}(\omegabar) \rightarrow 
    L^k_{1,loc}(\omegabar)$, $\mathscr{A}: L^k_{1,loc}(\omegabar) 
    \rightarrow L^{k-1}_{1,loc}(\omegabar)$ 
    such that
    \begin{enumerate}[(i)]
        \item $\rop(W^k_{1,loc}(\omegabar)) \subseteq W^k_{1,loc}(\omegabar)$,
            $\aop(W^k_{1,loc}(\omegabar)) \subseteq W^{k-1}_{1,loc}(\omegabar)$
            and $\mathscr{R}\omega - \omega = 
            d\mathscr{A}\omega + \mathscr{A}d\omega$ and 
            $d\rop \omega  = \rop d \omega$ for 
            $\omega \in W^k_{1,loc}(\omegabar)$
        \item for any $1 \leq p \leq \infty$, 
            $\rop(W^k_p(\omegabar)) \subseteq S^k_p(\omegabar)$ and
            $\aop(S^k_p(\omegabar)) \subseteq S^{k-1}_p(\omegabar)$
        \item $\rop: W^k_p(\omegabar) \rightarrow 
            (S^k_p(\omegabar),\norm{\cdot}{S^k_p(\omegabar)})$ is bounded (here
            we mean with $(S^k_p(\omegabar),\norm{\cdot}{S^k_p(\omegabar)})$ 
            the space $S^k_p(\omegabar)$ endowed with the norm 
            $\norm{\cdot}{S^k_p(\omegabar)}$))
    \end{enumerate}
\end{theorem}

Let 
$\iota: S^k_p(\omegabar) \hookrightarrow W^k_p(\omegabar)$ be the inclusion operator. 
The inclusion induces an
isomorphism on cohomology \cite[Lemma 4, Corollary]{goldshtein} i.e. 
$[\iota \circ \varphi^{-1}]: \mathscr{H}_p^k(K) \rightarrow \lpcoho(\omegabar)$ is an isomorphism. 

\begin{remark}
    \color{red}
    Let $L \subseteq K$ be a full-dimensional %TBD: Define that
    simplicial subcomplex. Let $\rop_L$ be the regularization operator 
    constructed for this simplicial subcomplex. The construction of the
    regularization operator is done locally on the stars of the vertices. %TBD: Define stars
    Therefore if we now look at a vertex $x_i \in L$ s.t. its star 
    $\Sigma_i$ does not include any boundary vertices of $L$ then we have 
    $\rop \omega = \rop_L(\omega|_{|L|})$ for any $\omega \in W^k_{1,loc}(|K|)$
    if we choose the ordering of the vertices appropriately.
    We will not prove this because it would require to go over the construction
    of the operator which we want to avoid as it is quite technical. Thus, 
    we refer to \cite[Sec.\,2]{goldshtein} for further details.
\end{remark}


% However, the question remains how we can go to the other direction i.e.

% In conclusion, we get the following isomorphisms of cohomologies:
% \begin{align*}
%     \lpcoho(\omegabar) \xrightarrow{[\iota]^{-1}} \mathscr{H}_p^k(K) 
%     \xrightarrow{[I]} H^k_p(K).
% \end{align*}
% This result will be crucial in the next section to obtain a unique solution
% to our problem.

% The idea is that for a $k$-form $\theta$ we have the cochain 
% $I(\theta) \in C^k(K)$, $I(\theta)(\sigma) = \int_\sigma \theta$ for any
% $k$-simplex $\sigma$. We want to show that this map induces a isomorphism on
% cohomology. 

% In the next step, we will use two operators $\rop$ and $\aop$ from 
% \cite{goldshtein}.
% The precise definition and details of their construction
% are not relevant for our purposes because we will only use
% the some properties that we collect in the following theorem
% (cf. \cite[Thm.2]{goldshtein}). Let 
% \begin{align*}
%     S^k_p(\omegabar) = \{ \eta \in W^k_{\infty,loc} | \, 
%     I\eta \in C^k_p(K)\}.
% \end{align*}

% Now define $\bar{I} \vcentcolon= I \circ \rop$. This gives a well-defined 
% mapping from $W^k_p(\omegabar)$ to $C^k_p(K)$. Following the arguments of 
% \cite{goldshtein} and using the properties of $\rop$ and $\aop$ one can show 
% that the induced homomorphism of cohomologies 
% $[\bar{I}]: H^k_{p,dR}(\omegabar) \rightarrow H^k_p(K)$ is an isomorphism.
% This isomorphism will play a crucial part in the proof of uniqueness and 
% existence of solutions in the next section.


\section{Existence and uniqueness of solutions}

We now return to the main problem that this thesis is about, the magnetostatic 
problem. We exterior domain we mean that our domain $\Omega \subseteq \real^3$
is the complement of a compact set. The main motivation for this problem is 
the when $\Omega$ complement of a torus. This is also the motivation behind the
topological assumption that we will give. 

Of course in order for the curve integral constraint to be well-defined 
we need to check the regularity of solutions. Then using the tools we developed
in the previous sections we will proof existence and uniqueness.

\subsection{Regularity of solutions}

We will rely heavily on standard regularity results about elliptic systems 
of the following form. Take $A_{ij}^{\alpha \beta} \in \real$
for $i$, $j$, $\alpha$, $\beta = 1,2,3$ and 
$A_{ij}^{\alpha \beta} = A_{ji}^{\beta \alpha}$. Then we have systems for the 
form 
\begin{align*}
    -\sum\limits_{\alpha \beta j} \partial_\alpha 
        (A_{ij}^{\alpha \beta} \partial_\beta B_j)
    = f_i - \sum\limits_\alpha \partial_\alpha F_i^\alpha
\end{align*}
with data $f_i, F_i^\alpha \in L^2(\Omega)$. We call this system 
\textit{elliptic} if $A$ satisfies the Legendre condition i.e.
\begin{align}
    \sum_{\alpha,\beta,i,j}A_{ij}^{\alpha \beta} \xi_\alpha^i \xi_\beta^j
    \geq c |\xi|^2, \quad \forall \xi \in \real^{3 \times 3} 
    \label{eq:legendre_condition}
\end{align}
with $c > 0$.
We then call $B$ a weak solution
of the problem if 
\begin{align}
    \int_\Omega \sum\limits_{\alpha,\beta,i,j} 
        A_{ij}^{\alpha \beta} \partial_\beta B_j \partial_\alpha \varphi_i dx
    = \int_\Omega \left\{ \sum\limits_i f_i \varphi_i + 
        \sum\limits_{\alpha,i} F_i^\alpha \partial_\alpha \varphi_i \right\} dx
    \label{eq:weak_elliptic_system}
\end{align}
for all $\varphi \in C^1_0(\Omega;\real^3)$. This formulation is taken from 
\cite[Sec. 1.3]{Lectures on PDE}. At first we will slightly modify
the notion of weak solution. 

\begin{proposition}
    Assume that we have an elliptic system with constant coefficients i.e. 
    $A$ is constant. Then
    (\ref{eq:weak_elliptic_system}) is fulfilled for all 
    $\varphi \in C^1_0(\Omega;\real^3)$ if and only if it is fulfilled 
    for $\varphi \in C^\infty_0(\Omega;\real^3)$.
\end{proposition}
\begin{proof}
    This follows by a simple density argument. Assume that 
    (\ref{eq:weak_elliptic_system}) is fulfilled for all test functions in 
    $C^\infty_0(\Omega;\real^3)$. Now take $\varphi \in C^1_0(\Omega;\real^3)$
    arbitrary. Because $\varphi \in H^1(\Omega)^3$ and 
    $C^\infty_0(\Omega;\real^3)$ is dense in $H^1(\Omega)^3$ we can find 
    a sequence $(\varphi^{(l)})_{l \in \naturalnum} \subseteq 
    C^\infty_0(\Omega;\real^3)$ s.t. $\varphi^{(l)} \rightarrow \varphi$
    in $H^1(\Omega)^3$. Thus the partial derivatives converge in $L^2(\Omega)$
    and we get
    \begin{align*}
        &\int_\Omega \sum\limits_{i,j,\alpha,\beta} 
            A_{ij}^{\alpha, \beta} \partial_\beta B_j \partial_\alpha \varphi_i
            dx
        = \sum\limits_{i,j,\alpha,\beta} A_{ij}^{\alpha, \beta}
            \int_\Omega \partial_\beta B_j \lim\limits_{l\rightarrow \infty} 
            \partial_\alpha \varphi^{(l)} dx
        \\ &\stackrel{\text{$L^2$ limit}}{=} 
            \lim\limits_{l\rightarrow \infty} 
            \int_\Omega \left\{ \sum\limits_i f_i \varphi^{(l)}_i + 
            \sum\limits_{\alpha,i} F_i^\alpha \partial_\alpha \varphi^{(l)}_i 
            \right\} dx
        = \int_\Omega \left\{ \sum\limits_i f_i \varphi_i + 
            \sum\limits_{\alpha,i} F_i^\alpha \partial_\alpha \varphi_i 
            \right\} dx.
    \end{align*}
    Since $\varphi \in C_0^1(\Omega;\real^3)$ was arbitrary the first 
    direction of the equivalence is proved. The other direction is trivial.
\end{proof}
So we see that in the case of constant coefficents we can consider 
just smooth compactly supported functions as test functions.

Next, we will state the crucial result about the regularity of elliptic systems
which will give us the desired regularity. This is Theorem 2.13 and Remark 2.16 
in \cite{Lectures on ellitic PDEs} in slightly less generality.
\begin{theorem}
    Let $\Omega$ be an open domain in $\real^n$. Let $A$ be constant and 
    satisfy the Legendre condition \ref{eq:legendre_condition}. Then for every 
    $B \in H^1_{loc}(\Omega)^3$ weak solution in the sense of
    (\ref{eq:weak_elliptic_system}) with $f \in H^k_{loc}(\Omega)^3$ and 
    $F \in H^{k+1}_{loc}(\Omega;\real^{m\times n})$ 
    we have $B \in H^{k+2}_{loc}(\Omega)^3$. 
\end{theorem}

\begin{corollary}\label{cor:smooth_solution}
    If under the assumptions of the previous theorem we consider the 
    homogeneous problem, i.e.
    \begin{align*}
        \int_\Omega \sum\limits_{\alpha,\beta,i,j} 
        A_{ij}^{\alpha \beta} \partial_\beta B_j \partial_\alpha \varphi_i dx =0
    \end{align*}
    for all $\phi \in C^\infty_0(\Omega;\real^3)$, then $B$ is smooth.   
\end{corollary}
It should be noted that this does not guarantee us any regularity on the 
boundary. 

Before we can apply this result, we have to check whether a solution of our 
problem $B$ is actually in $H^1_{loc}(\Omega)^3$. 

\begin{theorem}\label{thm:solution_in_H1loc}
    Assume $B \in H(\diver;\Omega) \cap H(\curl;\Omega)$. Then 
    $B \in H^1_{loc}(\Omega)^3$.
\end{theorem}
Note that we did not assume $B$ to be a solution.
\begin{proof}
    We know that for a function $u \in H_0(\curl;U) \cap H(\diver;U)$ 
    for some smooth domain $U$ we have $u \in H^1(U)^3$ 
    (cf. \cite[Remark 3.48]{monk}). Our $\Omega$ is 
    not assumed to be smooth so we can not apply this result directly.

    Take $Q \subset\subset \Omega$ open. Then we can find an open 
    cover of $\overline{Q}$ with a finite set of open balls $\{K_i\}_{i=1}^N$
    s.t. $K_i \subseteq \Omega$ and 
    \begin{align*}
        \overline{Q} \subseteq \bigcup\limits_{i=1}^N K_i.
    \end{align*}
    As a open cover of a compact set we can find a smooth partition of unity 
    $\{\chi_i\}_{i=1}^N$ subordinate to $\{K_i\}_{i=1}^N$. 
    $(B \chi_i)|_{K_i} \in H_0(\curl;K_i) \cap H(\diver;K_i)$ and thus 
    $(B \chi_i)|_{K_i} \in H^1(K_i)^3$ by the above mentioned result. 
    Also because $B \chi_i$ has compact support in $K_i$ we can extend it by 
    zero to obtain 
    $B \chi_i \in H^1(\real^3)^3$. Whence,
    \begin{align*}
        B|_Q = \large( \sum\limits_{i=1}^N \chi_i |_Q \large) B|_Q = 
        \sum\limits_{i=1}^N (\chi_i B)|_Q \in H^1(Q)
    \end{align*}
    i.e. $B \in H^1_{loc}(\Omega)^3$.
\end{proof}

The following lemma is a reformulation of the differential operator 
$\grad \diver - \curl \curl$ which will be needed when we write our 
magnetostatic problem in the above standard elliptic form.

\begin{lemma}
    Let $F \in H^2_{loc}(\Omega)^3$. Then 
    \begin{align*}
        \grad \diver F - \curl \curl F 
        = \begin{pmatrix} \Delta F_1 \\ \Delta F_2 
            \\ \Delta F_3  \end{pmatrix}.
    \end{align*}
\end{lemma}
\begin{proof}
    By a simple calculation and changing the order of differentiation
    \begin{align*}
        \grad \diver F = \begin{pmatrix} \partial_1^2 F_1 \\ \partial_1\partial_2 
            F_2 \\ \partial_1 \partial_3 F_3  \end{pmatrix}    
    \end{align*}
    and 
    \begin{align*}
        &\curl \curl F = \curl \begin{pmatrix} 
            \partial_2 F_3 - \partial_3 F_2 \\ \partial_3 F_1 - \partial_1 F_3 
            \\ \partial_1 F_2 - \partial_2 F_1  \end{pmatrix}
        = \begin{pmatrix} 
            \partial_2 (\partial_1 F_2 - \partial_2 F_1)
                - \partial_3 (\partial_3 F_1 - \partial_1 F_3)
            \\ \partial_3 (\partial_2 F_3 - \partial_3 F_2)
                - \partial_1 (\partial_1 F_2 - \partial_2 F_1)
            \\ \partial_1 (\partial_3 F_1 - \partial_1 F_3)
                - \partial_2 (\partial_2 F_3 - \partial_3 F_2)  
            \end{pmatrix}
        \\ &= \begin{pmatrix}
            \partial_1 \partial_2 F_2 - \partial^2_2 F_1 - \partial_3^2 F_1
            + \partial_1 \partial_3 F_3 
            \\ \partial_2 \partial_3 F_3 - \partial^2_3 F_2 - \partial_1^2 F_2
            + \partial_1 \partial_2 F_3
            \\ \partial_1 \partial_3 F_3 - \partial^2_3 F_2 - \partial_2^2 F_3
            + \partial_2 \partial_3 F_2  
            \end{pmatrix}
    \end{align*}
    and so by subtracting the two expressions
    \begin{align*}
        \grad \diver F - \curl \curl F  
        = \begin{pmatrix}
            \partial_1^2 B_1 + \partial_2^2 B_1 + \partial_3^2 B_1
            \\ \partial_1^2 B_2 + \partial_2^2 B_2 + \partial_3^2 B_3
            \\ \partial_1^2 B_3 + \partial_2^2 B_3 + \partial_3^2 B_3
        \end{pmatrix}
        = \begin{pmatrix}
            \Delta B_1 \\ \Delta B_2 \\ \Delta B_3
        \end{pmatrix}.
    \end{align*}
\end{proof}
We want to rewrite this system in the expression of the elliptic system 
\ref{}. We have the differential operator for $i = 1,2,3$
\begin{align*}
    - \Delta F_i = - \sum\limits_{\alpha = 1}^3 
        \partial_\alpha \partial_\alpha F_i
    = - \sum\limits_{\alpha,\beta = 1}^3 
    \partial_\alpha \delta_{\alpha,\beta} \partial_\beta F_i
    = - \sum\limits_{\alpha,\beta,j = 1}^3 
    \partial_\alpha \delta_{\alpha,\beta} \delta_{ij} \partial_\beta F_j
\end{align*}
so we get $A_{ij}^{\alpha\beta} = \delta_{ij} \delta_{\alpha \beta}$.
We have to check that the resulting differential operator is indeed
elliptic, but this trivial because for any 
$(\xi_\alpha^i)_{1\leq i,\alpha \leq 3}$
we get 
\begin{align*}
    \sum\limits_{\alpha,\beta,i,j} A_{ij}^{\alpha\beta} \xi_\alpha^i \xi_\beta^j 
    = \sum\limits_{\alpha,\beta,i,j} \delta_{ij} \delta_{\alpha \beta} 
        \xi_\alpha^i \xi_\beta^j 
    = \sum\limits_{\alpha,i} (\xi_\alpha^i)^2 = |\xi|^2
\end{align*}
so  the Legendre condition (\ref{eq:legendre_condition}) 
is fulfilled and the resulting system is elliptic. The weak formulation 
is 
\begin{align*}
    \int_\Omega \sum\limits_{\alpha,\beta,i,j} \delta_{ij} \delta_{\alpha\beta}
        \partial_\beta B_j \partial_\alpha \phi_i dx 
    = \sum\limits_{i=1}^3 \nabla B_i \cdot \nabla \varphi_i dx.
\end{align*}
Here we can assume $\varphi \in C_0^\infty (\Omega)^3$ because all coefficients
are constant.

\begin{theorem}[Smoothness of solutions]
    Let $\Omega \subseteq \real^3$ open and 
    $B \in H(\diver;\Omega) \cup H(\curl;\Omega)$ and 
    \begin{align*}
        \curl B &= 0,
        \\ \diver B &= 0.
    \end{align*}
    Then $B$ is smooth.
\end{theorem}
\begin{proof}
    Take $\varphi \in C_0^\infty(\Omega)^3$. Then 
    \begin{align*}
        0 &= \int_\Omega \diver B \diver \varphi + \curl B \cdot \curl \varphi dx
        = - \int_\Omega B \cdot (\grad \diver \varphi - \curl \curl \varphi) dx
        \\ &\stackrel{Lemma \ref{lem:}}{=} - \int_\Omega B \cdot 
            \begin{pmatrix}
                \Delta \varphi_1 \\ \Delta \varphi_2 \\ \Delta \varphi_3
            \end{pmatrix}
        = \sum\limits_{i=1}^3 \int_\Omega \nabla B_i \cdot \nabla \varphi_i dx.
    \end{align*}
    Note that the last integration by parts is well defined because 
    $B \in H^1_{loc}(\Omega)$ according to \ref{thm:solution_in_H1loc}. 
    So $B$ is a weak solution 
    of the elliptic system given by 
    $A_{ij}^{\alpha \beta} = \delta_{ij} \delta_{\alpha\beta}$. Because 
    we look at the homogenous problem our right hand side is obviously smooth 
    and thus $B$ is smooth as well due to Cor.\,\ref{cor:smooth_solution}.
\end{proof}

\begin{remark}
    Obviously, the above arguments can be generalized by using a 
    non-zero right hand side of our problem. Then we will in general 
    not obtain a smooth solution, but for a sufficiently regular right hand 
    side the curve integral would still be well-defined.
    {\color{red} How much regularity? Source?}
\end{remark}


\subsection{Reformulation of the problem} 

We will return now to the magnetostatic problem. In order to use the results
above we will reformulate the problem in the notation of differential forms.
From now on we assume $n=3$ i.e. we are in three dimensional space.
There are two ways to identify a vector field with a differential form 
(cf. \cite[Table 6.1 and p.70]{arnold}) either as a 1-form or a 2-form. 
For a vector field $B$ we define
\begin{align*}
    F^1\, B &\vcentcolon= B_1 \, dx_1 + B_2 \, dx_2 + B_3\, dx_3 \text{ and}\\
    F^2\, B &\vcentcolon= B_2 \, dx_2 \wedge dx_3 - B_2 \, dx_1 \wedge dx_3
        + B_3 \, dx_1 \wedge dx_2
\end{align*} 
as the corresponding 1-form and 2-form. 
Then the exterior derivative is $dF^2\,\omega$ corresponds to the divergence,
the codifferential $\delta F^2\,\omega$ 
corresponds to the curl and the normal component
being zero on the boundary corresponds 
to $\omega \in \mathring{H}^2(d)$.\cite{}. 

If we then use the association of
3-forms with scalars we have the corresponding boundary value problem without
the integral condition for 
2-forms: Find $\omega \in \mathring{H}^2(d)$ s.t.
\begin{align}
    \delta \omega &= 0, \\ 
    d\omega  &= 0 \text{ in } \Omega.
\end{align}
Next, we have to add the integral condition. 
We remind the reader that we are in three dimensions so
$**= \text{Id}$ 
and observe
\begin{align*}
    *F^2 \, B  &= B_1 \, **dx_1 + B_2 \, **dx_2 + B_3\, **dx_3 
        = B_1 \, dx_1 + B_2 \, dx_2 + B_3\, dx_3\\ 
    &= F^1 \, B.
\end{align*}
{\color{red}: Actually, we already use the Hodge star to define the vector 
proxies. So of course the vector proxy of the Hodge star will be the same.}
Then we have 
\begin{align*}
    \int_\gamma * F^2\, B = \int_\gamma F   ^1\, B = \int_\gamma B \cdot \text{d}l.
\end{align*}
In the last step we used the fact that the integration of a 1-form over a
curve is equivalent to the curve integral of the associated vector field
(cf. \cite[Sec. 6.2.3]{arnold}). Hence, we can add the integral condition 
\begin{align}
    \int_\gamma *\omega = C_0 \label{integral_condition}.
\end{align}
However, we have  only $\omega \in \mathring{H}_0^2(d,\delta)$ so 
$*\omega \in H^1(d)$ so this integral might not be well defined. In order 
to deal with this, we will again use the operator $\rop$ from 
Sec.\,\ref{sec:isomorphism_cohomology}. 

We know from Thm.~\ref{thm:operators} that $\rop *\omega \in S^1_2(\omegabar)$.
Using the operator $\varphi$ from \ref{eq:} we obtain 
$\varphi \rop *\omega \in S^1_2(K)$. Now we know from Sec.~\ref{sec:} that 
the integration mapping $I: S^1_2(K) \rightarrow C^1_2(K)$ is well-defined.
Denote $\bar{I} \vcentcolon= I\circ \varphi \circ \rop$ and 
let us replace the integral condition (\ref{eq:integral_condition}) with 
\begin{align*}
    \bar{I}(*\omega)(\gamma) = C_0.
\end{align*}

Of course, we have to justify why this is reasonable. 
So let us take $\eta \in S^1_2(\omegabar)$. That means we can integrate it 
directly using the definition from \ref{}. 
Let us also assume that $\eta$ is closed and 
$\int_\gamma \eta = C_0$. %TBD: Define the integral on S^k_p(\omegabar)
Then we know from Thm.~\ref{thm:operators}
\begin{align*}
    \rop \eta = \eta + d\aop \eta + \aop d\eta 
    \stackrel{\eta \text{ closed}}{=} \eta + d\aop \eta.
\end{align*}
We know further that $\aop \eta \in S^0_2(\omegabar)$. 
Apply $\varphi$ on both sides and use that fact that it commutes 
\ref{} with the 
exterior derivative to get
\begin{align*}
    \varphi \rop \eta = \varphi\eta + d \varphi\aop \eta.
\end{align*}
$[I]$ is an isomorphism of cohomology and thus sends exact S-forms to exact 
cochains. $\gamma$ is closed so $I(d \varphi\aop \eta)(\gamma) = 0$ and 
we conclude 
\begin{align*}
    \bar{I}(\eta)(\gamma) = I(\varphi \rop \eta)(\gamma) = I(\varphi \eta)
    \stackrel{\text{by def.}}{=} I(\eta),
\end{align*}
i.e. the integral remains unchanged if we integrate closed forms over 
closed chains. Because $*\omega$ is closed we thus do not change the integral 
if the curve integral $*\omega$ would already have been well defined before.

To summarize we obtain the following problem.
\begin{problem} \label{prob:magnetostatic_reformulated}
    Find $\omega \in \mathring{H}^2(d;\Omega)$ s.t.
    \begin{align*}
        d \omega &= 0, \\
        \delta \omega &= 0 \text{ in $\Omega$}, \\
        \bar{I}(*\omega)(\gamma)  &= C_0.
    \end{align*}
\end{problem}
\noindent We will examine existence and 
uniqueness of this problem in the next section.


\subsection{Existence and uniqueness}

% We start with the following
% \begin{proposition}\label{uniqueness_cochain}
%     Let $\gamma$ be a closed $k$-chain s.t. the homology class $[\gamma]$ 
%     spans the homology space $H^k_c$. Then for any $C_0 \in \real 
%     \setminus \{0\}$ 
%     if we have closed cochains $F,G$ s.t.
%     \begin{align*}
%     F(\gamma) = G(\gamma)= C_0
%     \end{align*}
%     then $[F] = [G]$ i.e. their cohomology classes are equal.
% \end{proposition} 
% \begin{proof}
%     From \cite[Sec. 2.5]{arnold} we know that 
%     $\text{dim}\,H^k(K) = \text{dim}\,H^k_c$. Because $F$ and $G$ are closed we
%     therefore have $\lambda_F, \lambda_G \in \real$ and a cohomology class 
%     $[b]$ s.t.
%     $[F] = \lambda_F [b]$ and $[G] = \lambda_G [b]$. This is 
%     equivalent to the existence of $(k-1)$-cochains $J_F,J_G$ s.t. 
%     \begin{align*}
%         F = \lambda_F b + \delta J_F \text{ and } G = \lambda_G b + \delta J_G.
%     \end{align*}
%     so 
%     \begin{align*}
%         0 \neq \lambda_F b(\gamma) + \delta J_F(\gamma) = F(\gamma) = G(\gamma)
%         = \lambda_G b(\gamma) + \delta J_F(\gamma).
%     \end{align*}
%     Because $\gamma$ is closed we have for any $(k-1)$-chain $J$, 
%     $\delta J(\gamma) = J(\partial \gamma) = 0$ and so we arrive at
%     $\lambda_F = \lambda_G$ i.e. $[F] = [G]$.
% \end{proof}


We start with the following

\begin{proposition}\label{prop:integral_exact_form_zero}
    Let $(\phi_i)_{i\in\mathbb{N}} \subseteq H^0(d)$ s.t. 
    $(d\phi_i)_{i\in\mathbb{N}}$ 
    is convergent and let $\gamma \in C_1(K)$ be a bounded closed $1$-chain.
    Then
    \begin{align*}
        \bar{I}(\lim\limits_{i \rightarrow \infty}d\phi_i)(\gamma) = 0.
    \end{align*}
\end{proposition}
\begin{proof}
    % Let $L \subseteq K$ be a bounded simplicial subcomplex 
    % s.t. it is full dimensional and $\gamma \in C_1(L)$
    % {\color{red} Make this more rigorous. We need to be able to restrict 
    % $L^2$-functions without problems.}  That means there is
    % $\phi_L \in H^0(d;|L|)$ s.t.
    % $\lim\limits_{i\rightarrow \infty} d\phi_i = d\phi_L$. 
    % We can now use the existence and properties of the operators from 
    % Thm\,\ref{operators}, but on $L$ instead of $K$. We call the operators 
    % $\rop_L$ and $\aop_L$. Then we have 
    
    Because 
    $[\bar{I}]$ is an isomorphism of cohomology $\bar{I}$ has to send 
    exact forms to exact cochains. Because $\gamma$ is a closed $1$-chain 
    we obtain $\bar{I}(d\psi)(\gamma) = 0$ for every $\psi \in H^0(d)$. Because
    $\rop$, $\phi$ and $I$ are all bounded
    $\bar{I}$ is a continuous operator \ref{} and so we have 
    $\bar{I}(\lim\limits_{i \rightarrow \infty}d\phi_i) = 
    \lim\limits_{i \rightarrow \infty} \bar{I}(d\phi_i)$ with the limit of 
    cochains taken w.r.t. the norm 
    in $C^1_2(K)$. Now let $J_\gamma$ be the set of indices of simplices 
    contained in $\gamma$ i.e.
    \begin{align*}
        \gamma = \sum\limits_{i \in J_\gamma} \sigma_i.
    \end{align*}
    Here we remind that we fixed the ordering and indices for our simplicial 
    complex \ref{}. Let $N_\gamma \vcentcolon= |J_\gamma|$ which is finite. Then 

    \begin{align*}
        &\left| \left( \lim\limits_{i \rightarrow \infty} \bar{I}(d\phi_j) 
        \right) (\gamma) \right| 
        \leq \left| \left( \lim\limits_{i \rightarrow \infty} \bar{I}(d\phi_i) 
        - \bar{I}(d\phi_j) 
        \right) (\gamma) \right| \\
        &\leq \sum\limits_{k \in J_\gamma} 
        \left| \left( 
            \lim\limits_{i \rightarrow \infty} \bar{I}(d\phi_i) 
            - \bar{I}(d\phi_i) 
        \right) (\sigma_k) \right| \\
        &\leq N^{1/2}_\gamma \left(
            \sum\limits_{k \in J_\gamma} 
            \left| \left( 
                    \lim\limits_{i \rightarrow \infty} \bar{I}(d\phi_i) 
                    - \bar{I}(d\phi_j) 
                \right) (\sigma_k) 
            \right|^2
        \right)^{1/2} \\
        &\leq N^{1/2}_\gamma \lVert  
            \lim\limits_{i \rightarrow \infty} \bar{I}(d\phi_i)
            - \bar{I}(d\phi_j)
        \rVert _{C^{1}_2(K)} 
        \leq N^{1/2}_\gamma \epsilon
    \end{align*}
    and thus 
    \begin{align*}
        \left| \left( \lim\limits_{i \rightarrow \infty} \bar{I}(d\phi_i) 
        \right) (\gamma) \right| = 0
    \end{align*}
    because $\epsilon$ was arbitrarily small. {\color{red} This can be done 
    more elegantly by identifying $C^1_2(K)$ with $\ell^2$. 
    See notes from 28.12}
    % For every closed $1$-form $\nu \in H^1(d)$ we have 
    % \begin{align*}
    %     \bar{I}(\rop \nu)(\gamma) = \int_\gamma \rop \rop \nu 
    %     = \int_\gamma \rop \nu + d\aop\rop\nu + \aop d\rop \nu 
    %     = \bar{I}(\nu)(\gamma)
    % \end{align*}
    % where we used the properties of $\rop$ from Thm\,\ref{operators}. 
    % Let $\tilde{\Omega}$ be a bounded
    % subdomain of $\Omega$.
    % Because the image of $d$ is closed for every bounded domain 
    % (cf. \cite[Lemma 7]{picard}) we have that for any convergent sequence 
    % $(d\psi_i)_{i\in \mathbb{N}} \subseteq L^1_2(\Omega)$ 
    % there exists a $\psi \in H^0(d;\tilde{\Omega})$ s.t.
    % \begin{align*}
    %     \lim\limits_{i \rightarrow \infty}d\psi_i| _{\tilde{\Omega}} 
    %     = d\tilde{\psi}.
    % \end{align*}
    % Then we use the continuity of $\rop$
    % \begin{align*}
    %     \bar{I}(\lim\limits_{i \rightarrow \infty}d\phi_i) =
    %     \bar{I}(\rop \lim\limits_{i \rightarrow \infty}d\phi_i) = 
    %     \bar{I}(\lim\limits_{i \rightarrow \infty}d\rop \phi_i) =
    %     \int_\gamma \rop \lim\limits_{i \rightarrow \infty}d\rop \phi_i =
    %     \int_\gamma \rop d\tilde{\psi} = 0.
    % \end{align*}
\end{proof}

\begin{lemma}\label{lem:extension_by_zero}
    Let $\Omega$, $U$ be open and $U \subseteq \Omega$. 
    Let $\nu \in \mathring{H}^k(d;U)$. Define $\bar{\nu}$ as the 
    zero extension of $\nu$ in $\Omega \setminus U$. 
    Then $\bar{\nu} \in \mathring{H}^k(d;\Omega)$ and 
    $d\bar{\nu} = d\nu$ in $U$ and $d\bar{\nu} = 0$ in $\Omega \setminus U$.
\end{lemma}
\begin{proof}
    The proof works exactly as in the standard Sobolev case. 
    Take $\varphi \in \smoothcompforms{k}{\Omega}$ arbitrary. 
    Define $\bar{\nu}' = d\nu$ in $U$ and $\bar{\nu}' = 0$ in 
    $\Omega \setminus U$. We want to show that 
    \begin{align*}
        \int_\Omega \bar{\nu} \wedge *\delta \varphi 
        = \int_\Omega \bar{\nu}' \wedge *\varphi
    \end{align*}
    which implies that $\bar{\nu}' = d\bar{\nu}$. If
    $\text{supp}\, \varphi \subseteq U$ this follows by applying the definition 
    of $d\nu$. If $\text{supp}\, \varphi \subseteq 
    \text{int}(\Omega \setminus U)$ then it is also trivial since both 
    $\bar{\nu}$ and $\bar{\nu}'$ are zero in $\Omega \setminus U$. 

    Otherwise, we recognize $\varphi |_U \in C_c^\infty \Lambda^{k+1}(U)$. 
    %Did I define that space?
    Thus we get from the definition of the zero boundary condition 
    (Def.~\ref{def:zero_boundary_condition})
    \begin{align*}
        \langle \bar{\nu}, \delta \varphi \rangle_{L^k_2(\Omega)}
        = \langle \nu, \delta \varphi \rangle_{L^k_2(U)}
        = \langle d\nu, \varphi \rangle_{L^{k+1}_2(U)}
        = \langle \bar{\nu}', \varphi \rangle_{L^{k+1}_2(\Omega)}.
    \end{align*}
    Thus, $\bar{\nu}' = d\bar{\nu}$. Because $d\nu \in L^{k+1}_2(U)$,
    $d\bar{\nu} \in L^{k+1}_2(\Omega)$ and the lemma is proven.
\end{proof}

\begin{theorem}[Existence of solution]\label{thm:existence}
    Let $\Omega \subseteq \real^3$ be such that $\real^3 \setminus \Omega$
    is compact and $\interior \omegabar = \interior \Omega$. Assume there 
    exists a simplicial complex $K$ s.t. $\omegabar = |K|$ and a simplicial
    subcomplex $L \subseteq K$ s.t. $|L|$ is homeomorphic to the exterior 
    of a ball with radius $R > 0$ and $K \setminus L$ is finite. Both 
    $K$ and $L$ are assumed path-connected in the sense of simplicial complexes
    (cf. \ref{}). 
    For $K$ we require that $H_1(K) = \integers [\gamma]$ for a $1$-chain 
    $\gamma$.
    Assume that 
    the GKS-condition \ref{ass:gks_condition} holds for $K$. 
    Then there exists a solution 
    to Problem \ref{prob:magnetostatic_reformulated}.
\end{theorem}
\begin{proof}
    Under the given assumptions we know that we can find a 
    $F \in \tilde{C}^1(K,L)$ (see Lemma \ref{lem:existence_vanishing_cochain})
    with $F(\gamma) = C_0$. 
    Because $K \setminus L$ is finite $F$ is obviously 2-summable 
    i.e. $F \in  \tilde{C}_2^1(K,L)$
    Then we use the fact that the integration mapping 
    $I: S^1_2(K,L) \rightarrow \tilde{C}^1_2(K,L)$ induces an isomorphism 
    on cohomology
    $[I]: \mathscr{H}^1_2(K,L) \rightarrow \tilde{H}^1_2(K,L)$.
    So there exists a 
    S-form $\tilde{\theta} \in S^1_2(K,L)$
    s.t. $[I(\tilde{\theta})] = [F] \in \tilde{H}_2^1(K,L)$. Thus 
    we have some $J \in \tilde{C}^0_2(K,L)$ s.t. 
    $I(\tilde{\theta}) = F + \partial J$ and thus 
    \begin{align*}
        I(\tilde{\theta})(\gamma) = F(\gamma) + J(\partial \gamma)
        \stackrel{\gamma \text{ closed}}{=} F(\gamma) = C_0.      
    \end{align*}
    Now define $\theta \vcentcolon= \varphi^{-1} \tilde{\theta}$.
    So we found already a $1$-form that has the desired curve integral.

    $\tilde{\theta} \in S^1_2(K,L) \subseteq S^1_2(K)$ so 
    $\theta \in S^1_2(\omegabar) \subseteq W^1_2(\omegabar)$.
    We now use the fact that $\text{int}\,\omegabar = \text{int}\,\Omega$ and
    remember the resulting identification of the Sobolev spaces 
    $W^1_2(\omegabar)$ and $W^1_2(\Omega)$ 
    (cf. Remark \ref{rem:identification_sobolev_spaces}) to obtain
    $\theta \in H^1(d;\Omega)$.
    Next, we will use the Hodge decomposition (\ref{hodge_decomposition}).
    Now we project $\star \theta$ onto harmonic forms to get 
    $\omega \in \mathring{H}^2_0(d,\delta;\Omega)$. 
    So because 
    $\star\theta$ is co-closed we 
    obtain a sequence $(\phi_i)_{i \in \naturalnum} \subseteq H^3(\delta)$ s.t.
    \begin{align*}
        \star\theta = \omega + \lim\limits_{i \rightarrow \infty}\delta \phi_i
    \end{align*}
    where the limit is in the $L^2$-sense.
    Apply the Hodge star operator on 
    both sides, use the fact that it is an isometry and thus continuous and then 
    remember the definition of $\delta$ to get a sequence 
    $(\psi_i)_{i \in \naturalnum} \subseteq H^0(d)$ s.t.
    \begin{align*}
        \theta = \star\omega + \lim\limits_{i \rightarrow \infty}d \psi_i.
    \end{align*}    
    Then we get
    \begin{align*}
        \int_\gamma \rop \star \omega 
        =\int_\gamma \rop (\theta - 
        \lim\limits_{i \rightarrow \infty}d \psi_i) 
        \stackrel{  \text{Prop.\ref{prop:integral_exact_form_zero}}  }{=} 
            \int_\gamma \rop \theta 
        \stackrel{  \text{Prop.\ref{}}  }{=}  
            \int_\gamma \theta = C_0.
    \end{align*}
    In the penultimate step equality we used the fact that 
    $\rop$ does not change the integral because $\theta \in S^1(K)$ is closed.
    Thus $\omega$ fulfills the integral condition. 
    Because $\omega \in \mathring{H}^2_0(d,\delta;\Omega)$ all other conditions 
    are satisfied as well and $\omega$ is a solution.
\end{proof}


In the proof of uniqueness we will use the following lemma. 
We are now back in the realm of standard vector analysis so all the notation
is to be seen in this light (e.g. $H^1$ is here the standard Sobolov space 
and not a Sobolev space of differential forms).

\begin{lemma}\label{lem:gradient_sequence}
    Let $\phi \in L^2_{loc}(\Omega)$ with $\nabla \phi \in L^2(\Omega)$. Then 
    there exists a sequence $(\phi_i)_{i \in \naturalnum} \subseteq H^1(\Omega)$
    s.t. $\nabla \phi_i \rightarrow \nabla \phi$ in $L^2(\Omega)^3$.
\end{lemma}
\begin{proof}
    Take $B_R$ with $R$ large enough s.t. $B_R^c \subseteq \Omega$. 
    Define $\Omega_R \vcentcolon= B_R \cap \Omega$. Then 
    $\overline{\Omega}_R \subseteq B_{R+1}$ and $\Omega_R$ is a Lipschitz 
    domain and $B_{R+1}$ is pre-compact and
    $\phi|_{\Omega_R} \in W^{1,2}(\Omega_R)$. Therefore we can find an extension
    $E\phi \in W_0^{1,2}(\Omega_{R+1}) \hookrightarrow W^{1,2}(\mathbb{R}^3)$
    (cf. \cite[Sec.\,1.5.1]{mazya}). So we can now define
    \begin{align*}
    \bar{\phi} \vcentcolon=
    \begin{cases}
        \phi & \mbox{in $\Omega$}\\
        E\phi & \mbox{in $\Omega^c$.}\\
    \end{cases}
    \end{align*}
    Then $\bar{\phi} \in L^2_{loc}(\real^3)$ and 
    $\nabla \bar{\phi} \in L^2(\real^3)^3$. 
    Then there exists a sequence 
    $(\phi_l) _{l \in \naturalnum} \subseteq C^\infty_0(\real^3)$ s.t.
    $\nabla \phi_l \rightarrow \nabla \bar{\phi}$ in $L^2(\real^3)^3$ 
    (cf. \cite[Lemma 1.1]{simader}). By restricting $\phi_l$ to $\Omega$ 
    we obtain the result.
\end{proof}


\begin{theorem}
    Let the same assumptions hold as in Thm.\,\ref{thm:existence}.
    Then solution of the problem is unique.
\end{theorem}
\begin{proof}
    Let $\omega$ and $\tilde{\omega}$ both be solutions. Then $\star\omega$ is 
    closed thus $\rop \star\omega \in S_2^1(\omegabar)$ is also closed because 
    \begin{align*}
        d \rop \star \omega = \rop d\star\omega = 0.
    \end{align*}
    The same 
    holds for $\tilde{\omega}$. 
    $I(\varphi \rop \star \omega)
    , I(\varphi \rop \star \tilde{\omega}) \in C^1_2(K)$ 
    are closed $1$-cochains with
    $I(\varphi\rop \star \omega)(\gamma) 
    = I(\varphi\rop \star \tilde{\omega})(\gamma)$. 
    Thus Prop.\,\ref{prop:uniqueness_cochain} implies 
    $[I(\varphi\rop \star \omega)] 
    = [I(\varphi\rop \star \tilde{\omega})] \in H^1(K)$.
    Because $[I]: \mathscr{H}^1(K) \rightarrow H^1(K)$ is an isomorphism
    (see \ref{}) we have 
    \begin{align*}
        [\varphi\rop \star \omega] = [\varphi\rop \star \tilde{\omega}] 
    \end{align*}
    Here the cohomology classes are in the cohomology spaces of S-forms 
    $\mathscr{H}^1(K)$. That means there is a S-form $\theta \in S^0(K)$ s.t.
    \begin{align*}
        &\varphi \rop \star\omega - \varphi \rop \star\omega = d\theta \\ 
        &\Rightarrow \rop \star\omega -  \rop \star\omega = \varphi^{-1} d\theta 
        = d\varphi^{-1}\theta =\vcentcolon d\tilde{\theta}.
    \end{align*}
    We have $\tilde{\theta} \in S^0(\omegabar) \subseteq 
    W_{\infty,loc}^0(\omegabar)$ and $d \tilde{\theta} \in S_2^1(\omegabar)
    \subseteq W_2^1(\omegabar)$.

    Using the 
    properties of $\rop$ there exists a $\eta \in L_{2,loc}^0(\Omega)$
    with $d\eta \in L^1_2(\Omega)$ s.t. 
    \begin{align*}
        \star\omega - \star\tilde{\omega} 
        = -d\aop(\star\omega - \star\tilde{\omega})
        + d \theta = d\eta.
    \end{align*}
    By applying the Hodge star operator on both sides we find 
    $\mu \in L_{2,loc}^3(\Omega)$ with $\delta \eta \in L^2_2(\Omega)$ s.t.
    \begin{align}
        \omega - \tilde{\omega} = \delta \mu. \label{difference_solutions}
    \end{align}
    But because $\mu$ is only in $L_{2,loc}^3(\Omega)$ 
    we can not immediately conclude that $\delta \mu = 0$.

    Let us briefly return to vector proxies. Let $B$ and $\tilde{B}$ be 
    vector proxies of $\omega$ and $\tilde{\omega}$ respectively. 
    (\refeq{difference_solutions}) then translates to 
    $B, \tilde{B} \in \mathring{H}(\diver)$ %TBD: Not defined
    and $B-\tilde{B} = \nabla \phi$ with $\phi \in L^2_{loc}(\Omega)$ and
    $\nabla \phi \in L^2(\Omega)^3$.
    Because $B$ and $\tilde{B}$ are both harmonic we have 
    \begin{align*}
        0 = \int_\Omega (B-\tilde{B}) \cdot \nabla f\, dx 
        = \int_\Omega \nabla \phi \cdot \nabla f \,dx
        \quad \forall 
        f \in H^1(\Omega).
    \end{align*}
    % Let $\rho_\epsilon$ be a standard mollifier with compact support. 
    % Then we have 
    % \begin{align*}
    %     \partial_{x_i} (\rho_\epsilon * \phi) = 
    %     (\partial_{x_i} \rho_\epsilon) * \phi = 
    %     \rho_\epsilon * (\partial_{x_i} \phi) 
    %     \xrightarrow{\epsilon \rightarrow 0} 
    %     \partial_{x_i} \phi \text{ in $L^2(\Omega)$}.
    % \end{align*}
    % In the last step it is crucial that $\partial_{x_i} \phi \in L^2(\Omega)$.
    Thank to Lemma~\ref{lem:gradient_sequence} we know that 
    $\nabla \phi \in \overline{\nabla H^1(\Omega)}$. 
    Because $B$ and $\tilde{B}$ are harmonic we have 
    $B - \tilde{B} \in 
    \overline{\nabla H^1(\Omega)}^\perp$ and hence $\nabla \phi = 0$
    and $B = \tilde{B}$. Because the corresponding vector proxies are equal we 
    we obtain $\omega = \tilde{\omega}$.
\end{proof}



% \begin{proof}
%     Let $\omega, \tilde{\omega}$ both be solutions. 
%     Because  $*\omega$ and
%     $*\tilde{\omega}$ are closed the cochains 
%     $c\mapsto \int_c \rop*\omega$ and 
%     $c\mapsto \int_c \rop*\tilde{\omega}$ are closed.
    
%     Due to $\int_\gamma \rop *\omega = \int_\gamma \rop *\tilde{\omega}$ and the 
%     assumption that $[\gamma]$ spans the homology space we have with 
%     Prop.\,\ref{uniqueness_cochain} 
%     $[I(\rop *\omega)] = [I(\rop *\tilde{\omega})]$
%     and because $[I]$ is an isomorphism 
%     $[\rop *\omega] = [\rop *\tilde{\omega}]$. Hence,
%     \begin{align*}
%     [*\tilde{\omega}] = [\rop *\tilde{\omega}] = 
%     [\rop *\omega] = [*\omega].
%     \end{align*}
%     That is equivalent to the
%     existence of some $0$-form $\phi \in H^0(d)$ s.t.
%     $*\omega = *\tilde{\omega} + d\phi$. We continue by applying the Hodge
%     star operator to both sides and use the definition of the codifferential 
%     $\delta$:
%     \begin{align*}
%         \omega = \tilde{\omega} + *d\phi = \tilde{\omega} + *d**\phi 
%         = \tilde{\omega} + (-1)^{(n-k)(k-1)+1}\delta * \phi.
%     \end{align*}
%     Then because $\omega$ and 
%     $\tilde{\omega}$ are harmonic we have 
%     $\omega, \tilde{\omega} \perp \delta H^{3}(\delta)$ and therefore 
%     \begin{align*}
%     \omega = \tilde{\omega}.    
%     \end{align*}
% \end{proof}
% If we now translate this back to standard vector calculus terms we have found 
% the unique solution of the homogeneous magnetostatic on our domain $\Omega$.

% \section{Application: Homogeneous magnetostatic problem on the 
% exterior domain of a torus}

% As an application of this general boundary value problem we will have a look
% at the following magnetostatic problem. Let $\Omega$ be the exterior domain of
% a triangulated torus i.e. $\real^3 \setminus \Omega$ is a torus with 
% triangulated surface. Let $B$ be the magnetic field. We then have the following
% boundary value problem:
% %%%TBD: Include a picture
% It is natural to identify the magnetic field $B$ with a 2-form $\omega$.
% Then the exterior derivative is $d\omega$ corresponds to the divergence,
% the codifferential $\delta$ corresponds to the curl and the normal component
% being zero is corresponds to $\omega \in \mathring{W}^2_2(\Omega)$.\cite{}. 
% The curve integral is an integration over a one-dimensional manifold and
% corresponds therefore to the integration of a one-form. Therefore the
% condition \ref{curve_integral_contition} is can be expressed with the hodge 
% star operator $*$ as
% \begin{align*}
%     \int_\gamma *\omega = C_0.
% \end{align*}


% Now we want to apply our previous results.
% $\Omega$ fulfills all required assumptions for the domain. Because $\gamma$
% goes around the torus once and the homology space $H^1_c$ is one-dimensional
% {\color{red}(TBD: This has to be referenced or proven)}. Therefore because 
% $\gamma$ is not a boundary $[\gamma]$ spans $H^1_c$. Now all assumptions are
% fulfilled and we can apply our result. We will do so on 1-forms and transfer the
% result to 2-forms using the Hodge star operator.

% \subsection*{Existence}
% Let 
% $\tilde{\omega} \in \mathring{W}^1_2(\Omega) $ be the unique solution of our 
% general problem \ref{} and define $\omega \vcentcolon= *\tilde{\omega}$. Then
% we use $**=(-1)^{k(n-k)}\tilde{\omega} = \tilde{\omega}$ \cite[p.66]{arnold} 
% to get 
% \begin{align*}
%     d\omega = ** d*\tilde{\omega} = * (-1)^{n(k-1)+1} \delta \tilde{\omega}
%     = 0
% \end{align*}
% and
% \begin{align*}
%     \delta \omega = (-1)^{n(k-1)+1} *d*\omega = (-1)^{n(k-1)+1}* d\tilde{omega}
%     = 0.
% \end{align*}



%%% TBD: Nowaczyk's thesis is not a master's thesis
\printbibliography
\end{document}