\documentclass[12pt,a4paper]{article}
\usepackage[utf8]{inputenc}
\usepackage[english]{babel}
\usepackage{amsmath}
\usepackage{amsfonts}
\usepackage{amssymb}
\usepackage{amsthm}
\usepackage{xcolor}
\usepackage{csquotes}
\usepackage{faktor}
\usepackage{mathrsfs}
\usepackage{mathtools}

\usepackage{biblatex}
\addbibresource{bibliography.bib}
% \bibliography{bibliography}
% \bibliographystyle{ieeetr}


\newtheorem{proposition}{Proposition}
\newtheorem{theorem}{Theorem}

\theoremstyle{definition}
\newtheorem{assumption}{Assumption}
\newtheorem{definition}{Definition}

\DeclareMathOperator*{\esssup}{ess\,sup}

\newcommand{\aop}{\mathscr{A}}
\newcommand{\lpcoho}{H^k_{p,dR}}
\newcommand{\norm}[1]{\lVert #1 \rVert}
\newcommand{\omegabar}{\overline{\Omega}}
\newcommand{\real}{\mathbb{R}}
\newcommand{\rop}{\mathscr{R}} % short for R operator


\begin{document}

Let $\Omega \subseteq \real^n$ be a Lipschitz domain and $\text{Alt}^k$ 
the space of alternating $k$-linear maps on $\real^n$. 
Let $\langle \cdot,\cdot\rangle_{\text{Alt}^k}$ be the standard inner
product of alterating $k$-linear maps and 
$\lVert \cdot \rVert _{\text{Alt}^k}$ be the induced norm. Then
we define the $L^p$-norm of a $k$-form $\omega$ for $1\leq p < \infty$
as (cf. \cite{goldshtein})
\begin{align*}
\lVert \omega \rVert _{L^p\Lambda^k(\Omega)}\vcentcolon=
\left(\int_\Omega \lVert \omega \rVert _{\text{Alt}^k}^p \right)^{1/p}
\end{align*}
%%% TBD: Is this independent of coordinates i.e. does this notation make sense?
and for for $p=\infty$ as
\begin{align*}
\esssup_{x\in \Omega} \, \lVert \omega(x) \rVert _{{\text{Alt}^k}}.
\end{align*}
For $p=2$ we obtain a Hilbert space (cf. \cite[Sec. 6.2.6]{arnold}) 
with the $L^2$ inner product  
\begin{align*}
\langle \omega, \nu \rangle \vcentcolon= \int_\Omega \omega \wedge *\nu.
\end{align*}
Then $L^p\Lambda^k(\Omega)$ are the spaces of $k$-forms 
s.t. the corresponding $L^p$-norm is finite. \par

Our next goal is to extend the exterior derivative $d$ 
of smooth differential forms in the weak sense (cf. \cite{goldshtein}). 
This is done analogous to the definition of the usual weak derivative.
The exterior derivative $d\omega \in L^p\Lambda^{k+1}(\Omega)$ is defined as
the unique $(k+1)$-form in $L^p\Lambda^{k+1}(\Omega)$ s.t. 
\begin{align*}
\int_\Omega d\omega \wedge \phi = (-1)^{k+1}\int_\Omega \omega \wedge d\phi
\quad \forall \phi \in C_0^\infty \Lambda^{n-k-1}(\Omega).
\end{align*}
Just as in the usual Sobolev setting we define the following spaces:
\begin{align*}
W^k_p(\Omega) &= \left\{ \omega \in L^p\Lambda^k(\Omega)\, | 
\, d\omega \in L^p\Lambda^{k+1}(\Omega) \right\}, \\ %%%%%%%%%%%%%%%%%%%%%%
W^k_{p,loc}(\Omega) &= \left\{ \omega \, k \text{-form} \, | \,
\omega|_A \in W^k_\infty(A) \text{ for every } A \subseteq M \text{ compact} 
\right\}.
\end{align*}
For $\omega \in W^k_p(\Omega)$ for $p<\infty$ we define the norm 
\begin{align*}
\lVert \omega \rVert _{W^k_p} &\vcentcolon= 
\left( \norm{\omega}^p_{L^k_p} + \norm{d\omega}^p_{L^k_p} \right)^{1/p}
\end{align*}
and for $p=\infty$
\begin{align*}
    \lVert \omega \rVert _{W^k_p} &\vcentcolon= 
    \max\left\{ \norm{\omega}_{L^k_\infty},\, \norm{d\omega}_{L^k_\infty}
    \right\}.
\end{align*}
    


We want to examine the Hilbert space $L^2\Lambda^k(\Omega)$ more closely
(see \cite[Sec. 6.2.6]{arnold} for more details).  
%%% TBD: This is only for bounded domains. What changes?
Because $d:L^2\Lambda^k(\Omega) \rightarrow L^2\Lambda^{k+1}(\Omega)$
is an unbounded, densely defined operator with domain 
$C_0^\infty \Lambda^{k}(\Omega)$ the adjoint $\delta$ is well-defined and is 
equal to the codifferential operator of differential forms. We then define 
additionally the space $\mathring{W}^k_2(\Omega)$ as the closure of 
$C_0^\infty(\Omega) \subseteq W^k_2(\Omega)$ w.r.t. the $W^k_2(\Omega)$-norm.









Now we can formulate our boundary value problem. Let $\Omega \subseteq \real^n$ 
from now on be a an exterior polyhedral domain of a compact set i.e. 
$\real^n \setminus \Omega$ 
is a compact polyhedron. We then have the following boundary value
problem: For a fixed $C_0 \in \real$ and closed bounded $k$-chain $\gamma$ 
find $\omega \in \mathring{W}^k_2(\Omega)$ s.t.
\begin{align*}
    d\omega &= 0, \\
    \delta\omega &= 0 \text{ and} \\
    \int_\gamma \omega &= C_0.
\end{align*}
Because we consider polyhedral domains we assume that $\gamma$ consists of 
finitely many $k$-simplices and that the cohomology class $[\gamma]$ is a 
generator of the simplicial homology group. The well-definedness of the 
integral over $\gamma$ will be discussed later.
Our goal will be to show existence and uniqueness of solutions. In order to 
achieve this, we rely on a result about the isomorphism of a simplicial 
cohomology space $H^k_p(K)$ which will be defined below
and the $L^p$-cohomology space $H^k_{p,dR}(\omegabar)$ ($dR$ short for de Rham).
This result was proven in \cite{goldshtein}. In the diploma thesis of Nikolai
Nowaczyk \cite{nowaczyk}, which mostly is based on this paper, 
many additional details can be found. The result will be presented in the
next section. It should be noted that even though the results in 
\cite{goldshtein} are
proved explicitely for smooth manifolds without boundary the results can be 
extended to Lipschitz manifolds with boundary (see the proof of Theorem 2 and 
the remark at the end in \cite{goldshtein}). Therefore, we can apply the result
to our case.










\section{Isomorphism of Cohomology}

\subsection{Assumptions}
In order to formulate the assumptions necessary for the result from 
\cite{goldshtein} to work we will define some basic things from 
simplicial topology theory. More details and references can be found in 
\cite[Chapter 4.21]{topology_and_geometry}.

\begin{definition}[Affine simplex]
    Let $x_1$, $x_2$, ..., $x_n$ be affine independent. Then 
    \begin{align*}
    [x_1,x_2,...,x_n] \vcentcolon= \text{conv}\,\{x_1,...,x_n\}
    \end{align*}
    is called an affine simplex.
\end{definition}

%%% TBD: Question: In the book this is defined in infinite dimensions. Does this
%%% lead to any problems?

\begin{definition}[Simplicial complex]
    A \textit{simplicial complex} $K$ is a collection of affine simplices s.t.
    \begin{enumerate}
        \item $\sigma \in K \Rightarrow$ any face of $\sigma$ is in $K$,
        \item $\sigma,\, \tau \in K \Rightarrow \sigma \cap \tau$ is in $K$.
    \end{enumerate}
    We call $|K| \vcentcolon= \bigcup \{ \sigma | \sigma \in K\}$ the polyhedron of 
    $K$.
\end{definition}
For any topological space $X$ a homeomorphism 
$\tau: |K| \rightarrow X$ is called \textit{triangulation} of $X$.


Because $\omegabar$ is itself a polyhedron we can
assume that $\omegabar$ and $|K|$ are equal as subsets of $\real^n$ and we can
simply use the identity as triangulation.
However, we will use different metrics on $|K|$ and $\omegabar$. 
We use the Euclidian metric on 
$\omegabar$ and we use the standard simplicial metric on $|K|$ (cf. 
\cite[p.191]{goldshtein}). This metric is defined as follows:

Choose some numbering of the vertices $\{ x_1,\, x_2, ... \}$ and
take $f: |K| \rightarrow \ell^2$ where $\ell^2$ is the 
Hilbert space of real-valued square-summable sequences s.t. $f(x_i) = e_i$ 
with $e_i \in \ell^2$ being the standard unit vectors and $f$ is affine on 
every simplex. This mapping is unique.%%%TBD: Proof uniqueness%%%%

Then we define the metric on $|K|$ as the pullback $g_S = f^*g$ 
where $g$ is the standard metric in $\ell^2$. Let $\langle \cdot , 
\cdot \rangle$ be the standard scalar product on $\ell^2$. Then for $x \in |K|$ 
and $\sum_{i=1}^n v_i \frac{\partial}{\partial x_i}, \; 
\sum_{j=1}^n w_j \frac{\partial}{\partial x_j} \in T_x |K|$ we have 
\begin{align*}
g_S|_x\left(\sum_{i=1}^n v_i \frac{\partial}{\partial x_i}, 
\sum_{j=1}^n w_j \frac{\partial}{\partial x_j}\right) &= 
\left\langle \sum_{k=1}^\infty \sum_{i=1}^n v_i 
\frac{\partial f_k}{\partial x_i} (x)
\frac{\partial }{\partial y_k}, 
\sum_{l=1}^\infty \sum_{j,l=1}^n w_j \frac{\partial f_l}{\partial x_j} (x)
\frac{\partial }{\partial y_l} \right\rangle \\   
&= \sum_{i,j=1}^n \sum_{k,l=1}^\infty v_i \frac{\partial f_k}{\partial x_i} (x)
w_j \frac{\partial f_l}{\partial x_j} (x) 
\left\langle \frac{\partial }{\partial y_k}, \frac{\partial }{\partial y_l} 
\right\rangle\\
&= \sum_{i,j=1}^n \sum_{k=1}^\infty v_i \frac{\partial f_k}{\partial x_i} (x)
w_j \frac{\partial f_l}{\partial x_j} (x)\\
&= \sum_{i,j=1}^n \sum_{k=1}^\infty v_i w_j \left( Df(x)^T Df(x) \right)_{ij} \\
&= v^T Df(x)^T Df(x) w = \left\langle Df(x) v, Df(x) w \right\rangle,
\end{align*}
where $D$ denotes the Jacobian.
%%% TBD: This is only well-defined if x is in the interior of a full simplex.
%%% Also these matrices is not precise because we are not in finite dimensions

We have two crucial assumptions on the triangulation for the result to hold 
(cf. \cite[p.194]{goldshtein}). We summarize them under 
\textit{GKS-condition} named after the three authors of the \cite{goldshtein}.

\begin{assumption}[GKS-condition]
    We will assume the following on the simplicial complex $K$ 
    and the triangulation $\tau$:
    \begin{enumerate}
    \item The star of every vertex in $K$ contains at most $N$ simplices.
    \item For the differential of $\tau$ we have constants 
        $C_1, C_2 > 0$ s.t.
        \begin{align*}
        \lVert d\tau|_x \rVert < C_1, \; 
        \lVert d\tau^{-1}|_{\tau(x)} \rVert < C_2,
        \end{align*}
        where $d$ denotes the differential in the sense of differential 
        geometry and the norm is the operator norm w.r.t. the metrics on $|K|$ 
        and $\omegabar$.
    \end{enumerate}
\end{assumption}
The first assumption is equivalent to every vertex being contained in
at most $N$ simplices, which is fulfilled if we have a shape regular mesh.\par
%%% TBD: Reference or proof or sth%%%

Because $\tau$ is just the identity in our case 
the second assumption says that for every 
$x \in |K|$
\begin{align*}
\sup\limits_{v \neq 0} \frac{\lVert v \rVert}{\sqrt{g_S|_x(v,v)}} =
\sup\limits_{v \neq 0} \frac{\lVert v \rVert}{\lVert Df(x)v\rVert} < C_1
\end{align*}
and analogously
\begin{align*}
    \sup\limits_{v \neq 0} \frac{\lVert Df(x)v\rVert}{\lVert v \rVert} < C_1.
\end{align*}
%%% TBD: Give more details and interpretation

% In order to formulate the second assumption we have to note that for our
% triangulation $\tau$ we have that $M = |K|$. Therefore $\tau$ is just the 
% identity in our case. However, we use a different metric on $M$ and $|K|$. 
% On our domain $M$ we use the Euclidian metric. On $|K|$ however, we have to use
% the standard simplicial metric which is defined as follows (cf. 
% \cite[p.191]{goldshtein}). 
% We enumerate the vertices of the triangulation as ${ x_1,\, x_2,\, ... }$.
% Let $f: |K| \rightarrow \ell^2$ be an embedding 
% of the triangulation $|K|$ into $\real^\infty$ s.t. $f(x_i) = e_i$ where $e_i$ 
% are the standard unit vectors of $\real^\infty$ and $f$ is affine on every 
% simplex. Keeping this in mind, we have the following assumption 


\subsection{Statement of the Isomorphism}

The isomorphism of the cohomology spaces from \cite{goldshtein} uses several
mappings between different cohomology spaces. 
The first isomorphism is induced from a linear mapping
between from the so called \textit{S-forms} 
$S_p^k(K)$ to \textit{p-summable $k$-cochains} $C_p^k(K)$ which will both 
be defined next.

\begin{definition}
    We define the following norm of a $k$-cochain $f$
    \begin{align*}
    \norm{f}_{C_p^k(K)} \vcentcolon= 
    \left( \sum\limits_{c \, k\text{-chain}} |f(c)|^p \right)^{1/p}.
    \end{align*}
    and the space of \textit{p-summable k-cochains}
    \begin{align*}
    C_p^k(K) \vcentcolon= \{f \, k\text{-cochain} | \,  
    \norm{f}_{C_p^k(K)} < \infty \}.
    \end{align*}
\end{definition}
Take $\tau, \sigma \in K$ s.t. $\tau$ is a face of $\sigma$ which we write as
$\tau < \sigma$. It can be shown that the standard embedding 
$j: \tau \hookrightarrow \sigma$ induces an
restriction operator 
$j^*_{\sigma, \tau}:W^*_\infty(\sigma) \rightarrow W^*_\infty(\tau) $ which is 
bounded (cf \cite[p.191]{goldshtein}). 
\begin{definition}[S-forms]
    Let 
    \begin{align*}
    \theta = \{ \theta(\sigma) \in W^k_\infty(\sigma) | \sigma \in K\}
    \end{align*}
    be a collection of differential $k$-forms. We call $\theta$ S-form of degree
    $k$ if we have for all for simplices
    $\mu <\sigma$ 
    \begin{align*}
    j^*_{\sigma,\mu}\theta(\sigma) = \theta(\mu).
    \end{align*}
    We denote with $S^k(K)$ the space of all S-forms of degree $k$ over the chain
    complex $K$. 
    For $\theta \in S^k(K)$ we define $d\theta \vcentcolon= \{ d\theta(\sigma) | 
    \sigma \in K \} \in S^{k+1}(K)$. $S^*(K)$ is the resulting cochain complex.
\end{definition}
For $\theta \in S^k(K)$ we now define the norm
\begin{align*}
\lVert \theta \rVert _{S_p(K)}  \vcentcolon= \left( \sum_{\sigma \in K} 
\lVert \theta(\sigma) \rVert _{W^k_\infty(\sigma)}^p \right)^{1/p}.
\end{align*} 
$S^k_p(K)$ are the S-forms of degree $k$ s.t. this norm is finite.

Using integration we can define define the homomorphism 
(see \cite[p.191]{goldshtein})
\begin{align*}
I: S_p^k(K) \rightarrow C_p^k(K), \; I(\theta)(\sigma) = 
\int_\sigma \theta(\sigma) \text{ for } \sigma \in K.
\end{align*}
With the exterior derivative $d$ on S-forms as defined above we define 
\begin{align*}
    \mathcal{Z}_p^k &\vcentcolon= \{ \theta \in S^k_p(K) | \, d\theta = 0 \} \\
    \mathcal{B}_p^k &\vcentcolon= dS^k_p(K)
\end{align*}
and then the resulting cohomology space

\begin{align*}
    \mathscr{H}_p^k(K) \vcentcolon= 
    \faktor{\mathcal{Z}_p^k}{\mathcal{B}_p^k}.
\end{align*}

We denote the standard cochain cohomology as $H^k_p(K)$. Then we have that the 
integration mapping
$I: S_p^k(K) \rightarrow C_p^k(K)$ 
induces an isomorphism on the cohomologies i.e.
$[I]: \mathscr{H}_p^k(K) \rightarrow H^k_p(K)$ is an isomorphism of vector 
spaces (see Theorem\,1 in \cite{goldshtein}
and the proof thereof).







The next step is to obtain an isomorphism between the cohomology 
of S-forms $\mathscr{H}_p^k(K)$ and the $L^p$ cohomology 
$\lpcoho(\omegabar)$. At first, we define 
\begin{align*}
\varphi: W^k_{\infty,loc}(M) \rightarrow S^k(K), \;
\omega \mapsto \{ \omega|_\sigma \, |\, \sigma \in K \}.
\end{align*}
This is a well-defined vector space isomorphism (\cite[p.191]{goldshtein}). This
way we can identify $W^k_{\infty,loc}(M)$ with $S^k(K)$.  Using the
isomorphism $\varphi$ we now define 
$S^k_p(M) \vcentcolon= \varphi^{-1} S^k_p(K)$.
It can be shown that $S^k_p(M) \subseteq W^k_p(M)$. Let 
$\iota: S^k_p(M) \hookrightarrow W^k_p(M)$ be the inclusion operator. 
The inclusion induces an
isomorphism on cohomology \cite[Lemma 4, Corollary]{goldshtein} i.e. 
$[\iota]: \mathscr{H}_p^k(K) \rightarrow \lpcoho(\omegabar)$ is an isomorphism. 


In conclusion, we get the following isomorphisms of cohomologies:
\begin{align*}
    \lpcoho(\omegabar) \xrightarrow{[\iota]^{-1}} \mathscr{H}_p^k(K) 
    \xrightarrow{[I]} H^k_p(K).
\end{align*}







\section{Existence and uniqueness of solutions}
\subsection{Well-definedness of the integral constraint}


In the problem, we have the additional constraint 
\begin{align*}
    \int_\gamma \omega = C_0.
\end{align*}
for some $C_0 \in \real$ and a closed bounded $k$-chain $\gamma$. 
A priori however, we only assume 
$\omega \in \mathring{W}^k_2(\Omega)$. We have to check if and how this can
be well defined.

Above, we introduced the integral operator $I$ for $S^k_p(K)$ which can be 
therefore be applied on $\omega \in S^k_p(M)$ as
\begin{align*}
    I(\omega) \vcentcolon= I (\varphi(\omega)).
\end{align*}
\noindent If we fix now the closed 
$k$-chain $\gamma$ then $I(\cdot)(\gamma) = \int_\gamma$ becomes a functional on
$S^k_p(M)$, but it is a-priori not clear how to extend this to closed forms in
$W_p^k(M)$. 


We know that $\int_\gamma d\eta = 0$ for $\eta \in S^{k-1}_p(M)$ because 
otherwise $I$ would not induce an isomorphism on cohomology. We extend this now
by setting $\int_\gamma d\nu = 0$ for all $\nu \in W^{k-1}_p(M)$. 
We have to check whether this is consistent with the definition above. 
Let be $\nu \in W_p^{k-1}k(M)$ s.t. $d\nu \in S^k_p(M)$. Let $A \subseteq M$ be a 
bounded neighborhood of $\gamma$. We can then find 
$\tilde{\nu}$ s.t.  $\tilde{\nu} \in W^{k-1}_q(A)$ for any $q > 1$ and 
$d\tilde{\nu} = d\nu$ \cite[Thm 3.1.1]{schwarz}. Now it is possible to apply 
Stoke's theorem \cite[Thm. 9]{goldshtein_integration} to get  
$\int_\gamma d\nu = 0$. This shows consistency.


In order to extend the functional $\int_\gamma$ we will use two operators
$\mathscr{R}$ and $\mathscr{A}$ which are constructed in the second part 
of \cite{goldshtein}. The precise definition and details of their construction
are not relevant for our purposes because we will only use
the following properties (cf. \cite[Thm.2]{goldshtein}).

\begin{theorem}\label{operators}
    Assume that the triangulation $\tau$ fulfills the GKS-condition.
    Then there exist linear mappings $\mathscr{R}: L^k_{1,loc} \rightarrow 
    L^k_{1,loc}$, $\mathscr{A}: L^k_{1,loc} \rightarrow L^{k-1}_{1,loc}$ 
    such that
    \begin{enumerate}
        \item $\mathscr{R}\omega - \omega = 
            d\mathscr{A}\omega + \mathscr{A}d\omega$ for 
            $\omega \in W^k_{1,loc}(M)$
        \item for any $1 \leq p \leq \infty$, 
            $\rop(W^k_p(M)) \subseteq S^k_p(M)$.
    \end{enumerate}
\end{theorem}

\noindent We can now use this operator $\rop$ to define $\int_\gamma \omega$ for closed
$\omega \in W^k_p(M)$ as
\begin{align*}
\int_\gamma \omega \vcentcolon= \int_\gamma \rop\omega.
\end{align*}
This is consistent because if $\omega \in S^k_p(M)$ closed then due to 
Thm.\,\ref{operators}
\begin{align*}
\int_\gamma \rop\omega = 
\int_\gamma \omega + d\mathscr{A}\omega + \mathscr{A}d\omega = 
\int_\gamma \omega.
\end{align*}

\subsection{Existence}

Returning now back to the problem, we are now able to proof existence of a 
solution. Take a closed cochain $F \in C^k_p(K)$ s.t. $F(\gamma) = C_0$ and 
$F(\partial d) = 0$ for (k-1)-chains $d$. Then we know from ??? that 
there exists a unique $[\theta] \in \mathscr{H}_p^k(K)$ s.t. 
$[I]([\theta]) = [F]$. Let us take $\eta \vcentcolon= 
\varphi_\tau^{-1} \theta$. Then $\int_\gamma \eta = C_0$ holds. If we now take
the Hodge decomposition
$L^k_2(M) = \bar{\mathfrak{B^k}} \bigoplus \mathcal{H}^k \bigoplus
\bar{\mathfrak{B^*_k}}$ and define $\omega$ as the projection of $\eta$ onto the
harmonic forms $\mathcal{H}^k$. Then we know that $d\omega = 0$, 
$\delta\omega = 0$ and $\text{tr}\,\omega = 0$. So we only have to show that
\begin{align*}
\int_\gamma \omega = C_0.
\end{align*}

We know from the Hodge decomposition that there exists a sequence
$(\phi_i)_{i\in \mathbb{N}} \subseteq L^{k-1}_2(M)$ s.t. 
$\omega = \eta - \lim_{i \rightarrow \infty} d\phi_i$.
Let now be $R>0$ large enough s.t. $\gamma \subseteq B_R$. Then we know that
$d W^{k-1}_2(B_R)$ is closed in $L_2^k(B_R)$. Therefore there exists 
$\phi_R \in  W^{k-1}_2(B_R)$ s.t. 
$\lim_{i \rightarrow \infty} d\phi_i|_{B_R} = d\phi_R$. So we have
$\omega|_{B_R} = \eta|_{B_R} - d\phi_R$ and 
\begin{align*}
\int_\gamma \omega = \int_\gamma \omega|_{B_R} = \int_\gamma \eta|_{B_R} = C_0.
\end{align*}
This proves existence.

\subsection{Uniqueness}

The first step is to show that the cochain chosen in the proof of existence
is in fact unique if restricted to closed chains.

\begin{proposition}
    Let $\gamma$ be a $k$-chain s.t. the homology class $[\gamma]$ 
    is a generator of the homology group. Assume for some $C_0 \in \real$ 
    there exist cochains $F,G \in C^k_p(K)$ s.t.
    \begin{align*}
    F(\gamma) = C_0 \text{ and } F(\partial d) = 0 
    \text{ for all } (k-1) \text{-chains } d
    \end{align*}
    and the same for $G$. Then the restriction of $F$ and $G$ to closed 
    chains is the same.
\end{proposition} \label{uniqueness_cochain}
\begin{proof}
    Take any closed $k$-chain $c$. Because $\gamma$ is the generator of the 
    homology group we have $n \in \mathbb{Z}$ s.t. $[c] = [n \, \gamma]$
    where $[\cdot]$ is the corresponding homology class. That means that we have
    some $(k-1)$-chain $d$ s.t. $c = n \, \gamma + \partial d$. Using the 
    properties of $F$ and $G$,
    \begin{align*}
        F(c) = F(n \, \gamma + \partial d) = n F(\gamma) = n \, C_0.
    \end{align*}
    Because the same computation is valid for $G$ $F(c) = G(c)$ follows.
\end{proof}

\begin{theorem}
    Assume that a co-chain as in Prop. \ref{uniqueness_cochain}  exists. 
    %%% TBD: reference 
    Then the solution of the problem is unique.
\end{theorem}
\begin{proof}
    Let $\omega, \tilde{\omega}$ both be solutions. 
    Because $\int_\gamma \omega = \int_\gamma \tilde{\omega}$ and $\omega$ and
    $\tilde{\omega}$ are closed we have due to Prop. \ref{uniqueness_cochain} %%%TBD
    that $\int_c \omega = \int_c \tilde{\omega}$ for any closed $k$-chain $c$.
    So we have for the induced homomorphism $[I]([\rop \omega]) = 
    [I]([\rop \tilde{\omega}])$ and therefore due to the isomorphism of 
    cohomology $[\rop \omega] = [\rop \tilde{\omega}]$. Hence,
    \begin{align*}
    [\tilde{\omega}] = [\rop \tilde{\omega}] = 
    [\rop \omega] = [\omega]
    \end{align*}
    we get the equality of the cohomology classes. \par
    
    That is equivalent to the
    existence of some $(k-1)$-form $\phi \in W^{k-1}_2(\omegabar)$ s.t.
    $\omega = \tilde{\omega} + d\phi$. Then because $\omega$ and 
    $\tilde{\omega}$ are harmonic we have 
    $\omega, \tilde{\omega} \perp dW^{k-1}_2(\omegabar)$ and therefore 
    \begin{align*}
    \omega = \tilde{\omega}.
    \end{align*}
\end{proof}
\printbibliography
\end{document}