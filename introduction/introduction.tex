\documentclass[../master_thesis.tex]{subfiles}
\begin{document}
Mathematics is the art of abstraction, of putting basic theories into more 
general frameworks to gain insights that would otherwise be hard or 
impossible to see. James Clerk Maxwell recognized how many years ago that some
vector fields are to be understood in relation to curves 
i.e. the natural relation is the curve integral and some are to be understood 
as flux through a surface. This thought paves the way to a new paradigm 
which presents itself in differential forms which are a classic 
subject of differential geometry. Differential forms were recognized later by many to be the more suitable objects 
to study in fields like electromagnetism \cite{} and 
fluid dynamics. 

This thesis is about the problem of finding a static solution to Maxwell's equations.
This would be the magnetic field $\mathbf{B}$ s.t. for a given current source 
$\mathbf{J}$ 
\begin{align*}
    \curl \mathbf{B} &= \mathbf{J} 
    \\ \diver \mathbf{B} &= 0.
\end{align*}
A fascinating realization is that the well-posedness of this equation actually 
depends on fundamental topological quantities of the domain. If it is simply 
connected then there exists a unique solution.

This problem arises as a part of the ??? code where it is posed on the 
exterior of a toroidal domain with Neumann boundary conditions 
$\mathbf{B} \cdot \mathbf{n} = 0$. This is obviously not a simpliy connected domain 
and as an additional constraint we are 
given the curve integral 
\begin{align*}
    \int_\Gamma \mathbf{B}\cdot d\ell = C_0
\end{align*}
with $C_0 \in \real$ and $\Gamma$ being the curve that goes around the torus 
in poloidal direction (see \ref{}). 
It is a fascinating PDE for the reason that it combines the differential 
equations with this curve integral constraint which gives this problem 
a strong topological flavour. For a student familiar with standard PDEs, it 
begs the question how these notions can be combined to investigate this problem.

To come back to the statement from the beginning, the answer is provided 
by changing the point to a more general one, namely to differential geometry which
gives a much more general framework in which basic vector calculus can 
be embedded. One standard example of this is the beautiful Stokes theorem 
which generalizes the Gauss divergence theorem and Stokes' from vector calculus 
into the language of differential forms.

Differential forms are especially useful when we start integrating them. 
Whitney derives in his classic book \cite that differential forms are the actual objects 
suited to be integrated over manifolds like curves and surfaces which comes back 
to what Maxwell said about the natural integration 
of vector fields either over a curve or a surface. 
Another recommended exposition about this topic is Terence Tao's paper \cite{}. 

This integration of differential forms over manifolds turns out to provide 
a beautiful synergy between the three fields, 
calculus, differential geometry and topology. De Rham's famous theorem which 
relies on Stokes theorem for 
gives a very simple and yet deep relationship between the singular homology -- 
a popular tool in algebraic topology to study topological quanities -- 
and the integration of differential forms.

For a mathematician interested in analysis and motivated by the beauty of abstraction 
these reasons should already be sufficient to start learning and investigating 
differential forms and as a necessary fundament differential geometry. We will 
give a short exposition of these topics in the first sections of this thesis 
and then apply these ideas to the magnetostatic problem above.

This has been done before on bounded domains,
but we want to do it on unbounded domains. 
Traditionally, the theory of PDEs on unbounded domains 
is sparse in comparison to bounded ones. But unbounded domains can change the 
situation drastically. Even one of the simplest most basic problems in the topic 
of PDEs, the Laplace equation with Dirichlet boundary conditions, does not 
have a unique solution anymore when posed on an unbounded domains. 

Combining the tools from singular homology, differential geometry and 
functional analysis we will the existence and uniqueness of the magnetostatic problem 
under suitable assumptions.

The second part of this thesis will then be concerned with the numerical approximation. 
As is often done in numerical analysis -- especially in a master's thesis -- we 
will simplify the problem. to bounded domains in 2D, but we keep the curve integral 
constraint and the non-trivial topology of the domain. As explained above, the natural 
way to study an equation like this is in the language of differential forms. 
In the last decade, there has been a lot of research done about the combination of that 
which has been formulated in abstract form in the Acta Numerica paper from Arnold, Falk and Winther 
\cite{} who coined the term Finite Element Exterior Calculus (FEEC). This is the perfect 
tool for our purposes. We will study a variational formulation of the problem 
and will strongy rely on the tools from FEEC.

% We will replace the curve integral using the integration 
% by parts in 2D to obtain a new variational formulation to study. We will prove 
% an inf-sup condition and a a-priori error estimate greatly relying on the 
% general framework provided be the FEEC theory. 

This work will be split between the study of well-posedness of the magnetostatic problem 
in chapter \ref{} and then the numerics in chapter \,

In chapter \, will first spend a lot of time 
introducing the necessary tools for the proof of well-posedness of the magnetostatic problelm 
in 3D. We start with differential forms in Sec. \ref{} from the starting point 
of multilinear algebra. Afterwards we will give a short introduction on 
singular homology in Sec.\, which concludes in the de Rham isomorphism and then 
we will talk about unbounded operators and Hilbert 
complexes. After applying some regularity results in \ref{}
we will put all this together in Sec.\, to prove the existence and uniqueness 
of the magnetostatic problem und suitable suitable assumptions.

In Chapter\,\ref{}, we derive a variational formulation of a 2D version of the 
magnetostatic problem in Sec.\,. Then we give a short overview of the discrete Hilbert complexes 
which are fundamental to FEEC theory before applying it to our new variational formulation 
to obtain the well-posedness and a-priori estimate in Sec.\,. In the end, we will explain how this 
problem is actually implemented (Sec.\,\ref{}) and provide some numerical examples in Sec\,\ref{}.

\end{document}