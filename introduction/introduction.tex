\documentclass[../master_thesis.tex]{subfiles}
\begin{document}
Mathematics is the art of abstraction, of putting basic theories into more 
general frameworks to gain insights that would otherwise be hard or 
impossible to see.

This thesis is about the magnetostatic problem which reads.
It is a fascinating PDE for the reason that it combines the differential 
equations with this curve integral constraint which gives this problem 
a strong topological flavour. For a student familiar with PDEs it 
begs the question how these notions can be combined to investigate this problem.

James Clerk Maxwell recognized how many years ago that some
vector fields are to be understood in relation to curves 
i.e. the natural relation is the curve integral and some are to be understood 
as flux through a surface. This thought paves the way to a new paradigm 
which presents itself in differential forms which are a classic 
subject of differential geometry. Differential geometry especially on Riemannian 
manifolds gives a much more general framework in which basic vector calculus can 
be embedded. Differential forms turn out to be the more suited objects 
to study in fields like electromagnetism \cite{} and 
fluid dynamics. 

Differential forms become especially useful when we start integrating. 
Whitney derives in his classic book \cite that differential forms are the actual objects 
suited to be integrated over manifolds like curves and surfaces which comes back 
to Maxwell. Another nice exposition about this topic is Terence Tao's paper \cite{}. 

This integration of differential forms over manifolds turns out to provide 
a beautiful synergy between three fields, 
calculus, differential geometry and topology. De Rham's famous theorem which 
relies on Stokes theorem for 
gives a very simple and yet deep relationship between the singular homology -- 
a popular tool in algebraic topology to study topological quanities -- 
and the integration of differential forms.

For a mathematician interested in analysis and motivated by the beauty of abstraction 
these reasons should already be sufficient to start learning and investigating 
differential forms and as a necessary fundament differential geometry. We will 
give a short exposition of these topics in the first sections of this book. 

Applying these ideas to the magnetostatic problem above has been done before,
but here another challenge arises. Usually this problem is either posed 
on bounded domains. Traditionally, the theory of PDEs on unbounded domains 
is sparse in comparison to bounded ones. But unbounded domains can change the 
situation drastically. Even one of the simplest most basic problems in the topic 
of PDEs, the Laplace equation with Dirichlet boundary conditions, does not 
have a unique solution anymore when posed on an unbounded domains. 

Combining the tools from singular homology, differential geometry and 
analysis we will the existsnce and uniqueness of the magnetostatic problem 
under suitable assumptions.

The second part of this thesis will then be concerned with the numerical approximation. 
As usual for the numerical analysis -- especially in a master's thesis -- we 
will simplify the problem. to bounded domains in 2D, but we keep the curve integral 
constraint and the non-trivial topology of the domain. As explained above, the natural 
way to study an equation like this is in the language of differential forms. 
In the last decade, there has been a lot of research done about the comination of that 
which has been put in abstraction in the Acta Numerica paper from Arnold, Falk and Winther. 
\cite{} who coined the term Finite Element Exterior Calculus (FEEC). This is the perfect 
tool for our purposes. We will replace the curve integral using the integration 
by parts in 2D to obtain a new variational formulation to study. We will prove 
an inf-sup condition and a a-priori error estimate greatly relying on the 
general framework provided be the FEEC theory. 

The outline is as follows. We will first spend a lot of time 
introducing the necessary tools for the proof of well-posedness of the magnetostatic problelm 
in 3D. We start with differential forms in Sec. \ref{} from the direction 
of multilinear algebra. Afterwards we will give a short introduction on 
singular homology in Sec.\, which concludes in the de Rham isomorphism and then 
we will talk about unbounded operators and the special situation of Hilbert 
complexes which provide us with the good tools to deal with taht. 
We will put all this together in Sec.\, to prove the existence and uniqueness 
of the magnetostatic problem. 

In Chapter\,\ref{} we will focus on numerics. We derive a variational formulation of the 2D problem of the 
magnetostatic problem in Sec.\,. Then we give a short overview of the discrete Hilbert complexes 
which are fundamental to FEEC theory before applying it to our new variational formulation 
to obtain the well-posedness and a-priori estimate. In the end we will explain how this 
problem is actually implemented and provide some numerical examples.

\end{document}