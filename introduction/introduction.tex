\documentclass[../master_thesis.tex]{subfiles}
\begin{document}
Mathematics is the art of abstraction, of putting basic theories into more 
general frameworks to gain insights that would otherwise be hard or 
impossible to see. James Clerk Maxwell recognized that some
vector fields are to be understood in relation to curves 
and some to be understood in reference to a surface \cite[p.69]{arnold}. 
This thought paves the way to a new level of abstraction
which presents itself in differential forms which are a classic 
subject of differential geometry. Differential forms have since been sucessfully
used for electromagnetic theory (see \cite{differential_forms_electromagnetism}).
This connection to differential forms and their deep relation to geometry and topology
will be one major subject we investigate. These are indeed relations between different 
fields of mathematics which would be hard to reach without this overarching abstract theory.

To become more specific, this thesis is about the problem of finding a static solution to Maxwell's equations,
i.e.~we look for the magnetic field $\mathbf{B}$ s.t.~for a given current source 
$\mathbf{J}$ 
\begin{align*}
    \curl \mathbf{B} &= \mathbf{J} 
    \\ \diver \mathbf{B} &= 0.
\end{align*}
A fascinating realization is that the well-posedness of this equation actually 
depends on fundamental topological quantities of the domain. If it is simply 
connected, then there exists a unique solution, but otherwise additional constraints 
might be necessary.

Our main motivation behind studying this problem arises as part of the search for plasma equilibria where 
the magnetic field outside of the plasma has to be computed (see \cite{merkel1986}).
In this situation, we know that the magnetic field lines are orthogonal to the plasma surface
which translates to imposing Neumann boundary conditions 
$\mathbf{B} \cdot \mathbf{n} = 0$. The complement of a toroidal domain is obviously not simply connected and so 
we require an additional constraint. This is given by prescribing the total 
current  $I_{tot}$ flowing through the cross section of the torus.
Let $\Gamma$ be the curve that goes around the torus 
in poloidal direction (see Fig.\,\ref{fig:exterior_domain_with_curve_integral}). 
We assume that $\Gamma$ has positive distance from the torus and 
$\mathbf{J}$ is zero between $\Gamma$ and the torus.
Denote the surface enclosed by $\Gamma$ as $\Sigma$.
We assume that $\mathbf{J}$ is zero between $\Gamma$ and the surface of the toroidal domain.
After using Stokes theorem 
for surfaces, this gives us 
\begin{align*}
    \int_\Gamma \mathbf{B}\cdot d\ell= \int_\Sigma \curl \mathbf{B} \cdot d\mathbf{S} 
    = \mu_0 I_{tot} =\vcentcolon C_0
\end{align*}
which is added as an additional constraint to the system.

\begin{figure}
    \centering
    \tikzset{invclip/.style={clip,insert path={{[reset cm]
        (-\maxdimen,-\maxdimen) rectangle (\maxdimen,\maxdimen)
        }}}}
    \begin{tikzpicture}[line cap=round,line join=round,>=triangle 45,x=1.0cm,y=1.0cm]
    
    \draw [rotate around={0:(0,0)}] (0,0) ellipse (2.2249670036cm and 0.9749246981cm);
    
    \draw [shift={(0.0002330134,2.5017998378)}] plot[domain=4.1540732725:5.2805268013,variable=\t]({1*2.7976952264*cos(\t r)+0*2.7976952264*sin(\t r)},{0*2.7976952264*cos(\t r)+1*2.7976952264*sin(\t r)});
    \draw [shift={(0.0526305801,-2.6930446389)}] plot[domain=1.1350452078:2.0252255165,variable=\t]({1*3.0113988641*cos(\t r)+0*3.0113988641*sin(\t r)},{0*3.0113988641*cos(\t r)+1*3.0113988641*sin(\t r)});
    
    \begin{scope}
        \clip (0.5,-0.15263) rectangle (1.097116,-0.8481614);
        \draw [rotate around={88.3634229584:(1.4475483891,-0.3522365936)}, dashed] 
        (1.4475483891,-0.3522365936) ellipse (0.6122305788cm and 0.5963878315cm);
    \end{scope}
        
    \begin{scope}
        \begin{pgfinterruptboundingbox}
        \path[invclip] (0.5,-0.15263) rectangle (1.097116,-0.8481614);
        \end{pgfinterruptboundingbox}
    
        \draw [rotate around={88.3634229584:(1.4475483891,-0.3522365936)}] 
        (1.4475483891,-0.3522365936) ellipse (0.6122305788cm and 0.5963878315cm);
    \end{scope}
    \node at (2,-1) {$\Gamma$};
    
    \end{tikzpicture}
    \caption{Sketch of a feasible domain with curve $\Gamma$. 
        The domain of interest would be the complement of the torus.}
    \label{fig:exterior_domain_with_curve_integral}
\end{figure}

The resulting partial differential equation is fascinating for the reason that it combines the differential 
equations with this curve integral constraint which gives the problem 
a strong topological flavour. For a student familiar with standard PDEs, it 
begs the question how these notions can be combined to investigate this problem.

To come back to the statement from the beginning, the answer is provided 
by changing the context to a more general one, namely to differential geometry. 
This field
gives a much more general framework in which basic vector calculus can 
be embedded. One standard example of this is Stokes' theorem 
which generalizes the Gauss divergence theorem and Stokes' theorem from vector calculus 
in the language of differential forms.

Differential forms prove especially useful for integration. 
Whitney derives in his classic book \cite{whitney} why differential forms are
the correct objects to be integrated over manifolds like curves and surfaces which comes back 
to Maxwell's statement about the relation 
of vector fields to either a curve or a surface. 
Another recommended exposition about this topic is Terence Tao's short paper \cite{terence_tao}. 

The integration of differential forms over manifolds turns out to provide 
a beautiful synergy between the three fields 
calculus, differential geometry and topology. De Rham's famous theorem, which 
relies on Stokes theorem, 
gives a very simple and yet deep relationship between the singular homology -- 
a popular tool in algebraic topology to study topological quanities -- 
and the integration of differential forms. We recall it in Sec.\,\ref{sec:de_rhams_theorem}.

For a mathematician interested in analysis and motivated by the beauty of abstraction, 
these reasons should already be sufficient to start learning and investigating 
differential forms and -- as a necessary fundament -- differential geometry. We will 
give a short exposition of these topics in the first sections of this thesis 
and then apply these ideas to the magnetostatic problem described above.

We focus on the homogeneous problem with $\mathbf{J} = 0$.
Existence and uniqueness have been proven before on bounded domains 
(see \cite[Thm.\,5.4]{mitrea_layer_potentials}),
yet an exterior domain is always unbounded. 
Traditionally, the theory of PDEs on unbounded domains 
is sparse in comparison to bounded ones. But unbounded domains can change the 
situation drastically. Even one of the most basic problems in the topic 
of PDEs, the Laplace equation with homogeneous Dirichlet boundary conditions, does not 
have a unique classical solution (i.e. a solution using strong derivatives) 
anymore when posed on unbounded domains. As a counterexample,
pose the problem on the domain of positive real numbers. Then, any linear function 
is a solution.

By combining the tools from singular homology, differential geometry and 
functional analysis, we will prove the existence and uniqueness of the magnetostatic problem 
under suitable assumptions.

The second part of this thesis will then be concerned with the numerical approximation. 
As is often done in numerical analysis -- especially in a master's thesis -- we 
will simplify the problem. We will investigate the 2D magnetostatic problem on bounded domains, 
but we keep the curve integral 
constraint and the non-trivial topology of the domain. The ideas from differential geometry and 
differential forms can be applied in finite elements
which has been formulated in the Acta Numerica paper from Arnold, Falk and Winther 
\cite{arnold_falk_winther} who coined the term Finite Element Exterior Calculus (FEEC). This will prove to 
be the right tool for our purposes. We will study a variational formulation of the problem 
and will strongy rely on the tools from FEEC.

% We will replace the curve integral using the integration 
% by parts in 2D to obtain a new variational formulation to study. We will prove 
% an inf-sup condition and a a-priori error estimate greatly relying on the 
% general framework provided be the FEEC theory. 

This work will be split between the study of well-posedness of the magnetostatic problem 
in Chapter\,\ref{chap:existence_and_uniqueness} and then the numerics in Chapter\,\ref{chap:approximation_in_2D}.

In Chapter\,\ref{chap:existence_and_uniqueness}, we will spend the first sections on 
introducing the necessary tools for the proof of well-posedness of the magnetostatic problem 
in 3D. We start with differential forms in Sec.\,\ref{sec:differential_forms} from the starting point 
of multilinear algebra. Afterwards we will give a short introduction on 
singular homology in Sec.\,\ref{sec:singular_homology} which concludes in the de Rham isomorphism and then 
we will talk about unbounded operators and Hilbert 
complexes in Sec.\,\ref{sec:hilbert_complexes}. After applying some regularity results in Sec.\,\ref{sec:regularity_of_solutions},
we will put all this together in Sec.\,\ref{sec:existence_and_uniqueness} to prove the existence and uniqueness 
of the magnetostatic problem under suitable assumptions.

In Chapter\,\ref{chap:approximation_in_2D}, we derive a variational formulation of a 2D version of the 
magnetostatic problem in Sec.\,\ref{sec:variational_formulation_of_the_magnetostatic_problem_in_2D}. Then we give a short overview of the discrete Hilbert complexes 
which are fundamental to FEEC theory before applying it to our new variational formulation 
to obtain the well-posedness and a-priori estimate in Sec.~\ref{sec:discretized_magnetostatic_problem}. 
In the end, we will explain how this 
problem is actually implemented in Sec.\,\ref{sec:implementation} 
and provide some numerical examples in Sec\,\ref{sec:numerical_examples}.

\end{document}