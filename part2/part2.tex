\documentclass[../master_thesis.tex]{subfiles}
\begin{document}
\section{The Hodge Laplacian problem}

A Hilbert complex $(W^k, d^k)$ provides us with a natural operator which 
we call the \textit{abstract Hodge Lpalacian} defined as 
\begin{align*}
    L^k: D(L^k) \rightarrow W^k, u \mapsto d^{k+1}_* d^k u + d^{k-1} d^*_k u.
\end{align*}
with 
\begin{align*}
    D(L^k) \vcentcolon= \{ u \in V^k \cap V^*_k \mid d^k u \in V^*_{k+1}, d^*_k u \in V^{k-1} \}   
\end{align*}
The name stems from the standard Hodge Laplacian on Riemannian manifolds where the differential operator 
is the exterior derivative (see \cite{}). We will usually just leave out the term abstract.
This section will be dealing with the 
operator equation
\begin{align}
    L^k u = f.
\end{align}
We will follow the corresponding section 4.4 in Arnold's book \cite{arnold}.
Let us start with the question of well-posedness for which the following 
Proposition is crucial. Also we will stick to the notation introduced in Sec.\,\ref{sec:hilbert_complexes}

\subsection{Strong formulation}

\begin{proposition}[Kernel of the Hodge Laplacian]
    \begin{align}
        \ker L^k = \mathfrak{H}^k \label{eq:hodge_laplacian_operator_eq}
    \end{align}
\end{proposition}
\begin{proof}
    The direction "$\supseteq$" is clear from the definition of harmonic forms.
    We will leave away the index now. Let $u \in \ker L^k$,
    \begin{align*}
        0 = \langle L^k u, u \rangle 
        = \langle d^{k+1}_* d^k u, u \rangle + \langle d^{k-1} d^*_k u, u \rangle
        = \lVert d^k u, d^k u \rVert + \lVert d^*_k u, d^*_k u \rVert
    \end{align*}
    the norms and inner products are in the spaces $W^{k-1}$, $W^{k}$ and $W^{k+1}$.
    Thus we see that $d^*_k u = 0$ and $d^k u = 0$ and hence $u \in \mathfrak{H}^k$.
\end{proof}

This has immediate implications. 
Let us denote in general the orthogonal projection operator onto a subset 
$Z$ as $P_Z$.
This means that (\ref{eq:hodge_laplacian_operator_eq})
can only have a solution if $P_\mathfrak{H} = 0$ due to the Hodge decomposition 
and also that solutions are only unique up to addition of harmonic forms and 
thus (\ref{eq:hodge_laplacian_operator_eq}) is in general not well-posed. 
We require additional constraints. The simplest option is to require 
$u \perp \mathfrak{H}^k$. This leads us to the standard 
\textit{strong formulation} of the Hodge Laplacian problem with 
\textit{orthogonality constraint}: For any 
$f \in W^k$ find $u \in V^k \cap V^*_k$ s.t.
\begin{align*}
    L^k u = f - P_\mathfrak{H} f \text{ and } u \perp \mathfrak{H}^k.
\end{align*}
Because we know from $u \perp \ker L^k$ that the solution must be unique if it 
exists. We will later also consider different constraints on the harmonic part.

Let us investigate the Hodge-Laplacian in 3D which we are well familiar with.
We have $V^0 = H^1$, $V^1 = H(\curl)$, $V^2 = H(\diver)$ and $V^3 = L^2$. As before, 
we leave out the reference to the domain $\Omega$ which is assumed to have 
a compact Lipschitz boundary as in Sec.\,\ref{sec:l2_de_rham_complex_in_3d}.
Then we get $V^*_0 = L^2$, $V^*_1 = H_0(\diver)$, $V^*_2 = H_0(\diver)$
and $V^*_3 = H^1_0$. Since $d^0 = \grad$ and $d^*_1 = -\diver$ and we set 
$d^*_0 = 0$ to get 
\begin{align*}
    D(L^0) &= \{ u \in H^1 \mid \grad u \in H_0(\diver) \} \text{ and }
    \\ L^0 &= -\diver \grad = - \Delta.
\end{align*}
Assume that our domain is path connected. Recall that then 
$\mathfrak{H}^0$ are the constant functions and thus the 
condition $f \perp \mathfrak{H}^0$ means that 
\begin{align}
    \int_\Omega f \, dx = 0 \label{eq:average_equals_zero}
\end{align}
Then we see that the resulting Hodge Laplacian problem: Take $f \in L^2$ with (\ref{eq:average_equals_zero}) 
being fulfilled. 
Find $u \in D(L^0)$ 
s.t.
\begin{align*}
    -\diver \grad u &= f \text{ and }
    \\ \int_\Omega u &= 0
\end{align*}
Note that $u \in D(L^0)$ requires $\grad u \cdot \mathbf{n} = 0$ on $\partial \Omega$ and 
thus we recognize the strong formulation of the well-known Poisson problem with 
Neumann boundary conditions. (\ref{eq:average_equals_zero}) is the well-known 
compatibility condition for the Neumann boundary conditions.

Let us also look at situation for $k=1$. Then 
\begin{align*}
    D(L^1) &= \{ \mathbf{u} \in H(\curl) \cap H_0(\diver) 
        \mid \curl \mathbf{u} \in H_0(\curl), \diver \mathbf{u} \in H^1 \} \text{ and }
    \\ L^1 &= \curl \curl - \grad \diver.
\end{align*}
This operator we already know form \ref{}. The connection to the magnetostatic problem 
will be made clear later (?). The full problem then reads: Take $\mathbf{f} \in L^2(\Omega;\real^3)$. 
Find $\mathbf{u} \in D(L^1)$ s.t. 
\begin{align*}
    \curl \curl \mathbf{u} - \grad \diver \mathbf{u} &= \mathbf{f} - P_{\mathfrak{H}^1}\mathbf{f}.
    \\ u \perp \mathfrak{H}^1
\end{align*}

Analogously, we obtain the same problem with changed boundary conditions for the operator $L^2$ 
and $\mathfrak{H}^2$ instead of $\mathfrak{H}^1$. For $k=3$ we obtain the Poisson problem, 
but with Dirichlet boundary conditions.

In this second part of the thesis, the situation in 2D will also become relevant. So we will

\subsection{Mixed formulation}
We want to deal with Hodge Laplacian problem using a finite element approach. 
So we woul like to write this problem in variational form. The most principled way to do so 
is using the \textit{mixed weak formulation}: For any $f \in W^k$ find $\sigma \in V^{k-1}$, 
$u\in V^k$ and $p\in \mathfrak{H}^k$ s.t.
\begin{align}
    \langle \sigma, \tau \rangle - \langle u, d^{k-1}\tau \rangle 
        &= 0 \quad \forall \tau \in V^{k-1}, \label{eq:first_eq_mixed_formulation}
    \\ \langle d\sigma, v \rangle + \langle du, dv \rangle + \langle p, v \rangle 
        &= \langle f, v \rangle \quad \forall v \in V^k \text{ and} \label{eq:second_eq_mixed_formulation}
    \\ \langle u, q \rangle \quad \forall q \in \mathfrak{H}^k.
\end{align} 
This formulation is of course much more complicated than the first one \ref{}, 
but this formulation will turns out to be well-suited for finite element approximations.
But it begs the question if the two formulations are equivalent. We will answer it 
in two propositions. We will treat the harmonic constraint separately.

\begin{proposition}
    For any $f \in W^k$, (\ref{eq:first_eq_mixed_formulation}) and 
    (\ref{eq:second_eq_mixed_formulation}) hold i.i.f. 
    $\sigma = d^*_k u$, $p=P_\mathfrak{H}f$ and 
    \begin{align*}
        L^k u = f - P_\mathfrak{H}f
    \end{align*}
    i.e. the strong formulation is fulfilled without the additional 
    constraint $u \perp \mathfrak{H}^k$.
\end{proposition}
\begin{proof}
    We recognize that the first equation (\ref{eq:first_eq_mixed_formulation})
    is actually just the defining property of the adjoint hence 
    $\sigma = d^*_k u$. It also implies that $u$ is in fact in $V^*_k$.

    Now assume additionally, that  
    (\ref{eq:second_eq_mixed_formulation}) holds. Then by choosing $v \in \mathfrak{H}^k$
    we get from the definition of the harmonic forms 
    and $\mathfrak{H}^k \perp \mathfrak{B}^k$ from the Hodge decomposition and thus
    \begin{align*}
        \langle d\sigma, v \rangle + \langle du, dv \rangle + \langle p, v \rangle
        = \langle p, v \rangle = \langle f, v \rangle \quad \forall v \in \mathfrak{H}^k
    \end{align*}
    which is equivalent to $p = P_\mathfrak{H}f$. Next, we would like to show that 
    $\langle du, dv \rangle = \langle d^* du, v \rangle$ which is fulfilled if 
    $du \in V^*_{k+1}$. This follows from (\ref{eq:second_eq_mixed_formulation})
    since then 
    \begin{align*}
        \langle du, dv \rangle &= - \langle d\sigma, v \rangle + \langle f - P_\mathfrak{H}f, v \rangle
        \\ &\leq \lVert f - P_\mathfrak{H}f - d\sigma \rVert \lVert v \rVert
    \end{align*}
    and so $du \in V^*_{k+1}$.
    Thus we have 
    \begin{align*}
        &\langle d^{k-1}\sigma, v \rangle + \langle du, dv \rangle + \langle p, v \rangle
        \\ &= \langle d^{k-1} d^*_k u, v \rangle + \langle d^* du, v \rangle + \langle p, v \rangle
        \\ &= \langle d^{k-1} d^*_k u + d^* du + P_\mathfrak{H}f, v \rangle = \langle f, v \rangle
    \end{align*}
    and so since by assumption $V^k$ is dense in $W^k$,
    \begin{align*}
        d^*_k u + d^* du + P_\mathfrak{H}f = f.
    \end{align*}

    For the other direction assume $u \in D(L^k)$, $d^*_k u = \sigma$ and $p= P_\mathfrak{H} f$
    and 
    \begin{align}
        d^{k-1} d^*_k u + d^* du  = f - P_\mathfrak{H}f.\label{eq:proof_equivalence_of_formulations:strong_form}
    \end{align}    
    The first equation 
    holds as explained above. For the second one, we take the inner product of 
    (\ref{eq:proof_equivalence_of_formulations:strong_form}) with an arbitrary 
    $v \in V^k$ and immediately obtain (\ref{eq:second_eq_mixed_formulation}).
\end{proof}
So we immediately obtain 

\begin{corollary}
    The strong formulation and mixed weak formulation of the Hodge Laplacian problem 
    with orthogonality constraint $u \perp \mathfrak{H}^k$ are equivalent.
\end{corollary}
The reason why did not proof the equivalence directly is that we might add another 
harmonic constraint and would still get equivalence which we will need later. 
The problem with orthogonality constraint is then well-posed see \cite{}.


\section{Discrete Hilbert complex}

In order to approximate the Hodge Laplacian problem we want to use finite elements.
We want to use them in a way that we can rebuild the structure of the Hilbert complex 
in our discretization. This section follows Sec. 5.2 in Arnold's book \cite{arnold}.

Let us assume the we have finite dimensional subspaces $V_h^k \subseteq V^k$. 
Then we define completely analogous to the continuous case,
\begin{align*}
    \mathfrak{Z}_h^k &\vcentcolon= \{ v \in V_h^k \mid d v = 0 \} = \ker d \cap V^k_h
    \\ \mathfrak{B}^k_h &\vcentcolon= \{ dv \mid v \in V_h^{k-1} \}.
\end{align*}
We can now also define the discrete harmonic forms. Now the situation is slightly 
different however. We will not use the continuous adjoint $d^*_k$ to define it.
Instead,
\begin{align*}
    \mathfrak{H}_h^k \vcentcolon= \{ v \in \mathfrak{Z}_h^k \mid v \perp \mathfrak{B}^k_h \}
        = \mathfrak{Z}_h^k \cap \mathfrak{B}_h^{k,\perp}.
\end{align*}
Notice that we have $\mathfrak{Z}_h^k \subseteq \mathfrak{Z}^k$ and 
$\mathfrak{B}_h^k \subseteq \mathfrak{b}^k$, but due to 
$\mathfrak{B}_h^{k,\perp} \supseteq \mathfrak{B}^{k,\perp}$ we have in general
\begin{align*}
    \mathfrak{H}^k = \mathfrak{Z}_h^k \cap \mathfrak{B}_h^{k,\perp} 
    \not\subseteq    \mathfrak{Z}^k \cap \mathfrak{B}^{k,\perp} = \mathfrak{H}^k.
\end{align*}

This is now already enough to define apply the mixed formulation to the discrete setting: 
For a given $f \in W^k$, find $\sigma_h \in V_h^{k-1}$, $u_h \in V_h^k$ and 
$p_h \in \mathfrak{H}^k$ s.t.

\begin{align}
    \langle \sigma_h, \tau_h \rangle - \langle u_h, d\tau_h \rangle &= 0 \quad \tau_h \in V_h^{k-1},
    \\ \langle d\sigma_h, v_h \rangle + \langle du_h, dv_h \rangle + \langle p_h, v_h \rangle 
        &= \langle f, v_h \rangle \quad v_h \in V_h^k,
    \\ \langle u_h, q_h \rangle = 0, \quad \forall q_h \in \mathfrak{H}^k_h \label{eq:third_eq_discrete_mixed}.
\end{align}
When $\mathfrak{H}_h^k \not\subseteq \mathfrak{H}^k_h$ this is a nonconforming
method. The harmonic constraint (\ref{eq:third_eq_discrete_mixed}) can 
be replaced with a different one.

There are three crucial properties that are necessary for stability and convergence 
of the method. The first one is the common and reasonable assumption that 
-- as usual in finite element theory -- we want that the discrete spaces $V_h^k$
approximate the continuous ones $V^j$. This can be generally summarized that 
\begin{align*}
    \lim_{h \rightarrow 0} \inf_{v_h \in V_h^j} \lVert w - v_h \rVert = 0, \quad \forall w \in V^j.
\end{align*}
This is usually satisfied if we use established finite elements for a given space 
e.g. if we take Lagragian FE if $V = H^1$ or Raviart-Thomas if $V=H(\diver)$ \cite{}.

The next property is more restrictive. We require that $dV_h^{k-1} \subseteq V_h^k$ 
and $dV_h^j \subseteq V_h^{j+1}$. This shows that the we cannot simply use arbitrary 
discrete subspaces independent from one another. This property has a very nice
consequence. 
It shows that 
\begin{align*}
    V_h^{k-1} \xrightarrow{d^{k-1}} V_h^k \xrightarrow{d^k} V_h^{k+1}
\end{align*}
is itself a Hilbert complex and we can apply the general theory from 
Sec. \ref{sec:hilbert_complexes} directly to it. Let us do that.

Denote the restriction of $d^j$ to $V_h^j$ as $d_h^j$. Then as a linear map 
between finite spaces the adjoint -- denoted as $d_{j,h}^*: V_h^j \rightarrow V_h^{j-1}$ -- 
is everywhere defined. It is important to notice that in contrast to $d_h$ 
the adjoint $d^*_jh$ is not the restriction of the adjoint the continuous adjoint $d^*_j$.
In general, $V_h \not\subseteq V^*$ and so the continuous adjoint might not be 
well-defined for a given $v_h \in V_h$. 

So we obtain the Hilbert complex
\begin{align*}
    V_h^{k-1} \xrightarrow{d^{k-1}} V_h^{k} \xrightarrow{d^{k}} V_h^{k+1}
\end{align*}
and its dual complex
\begin{align*}
    V_h^{k-1} \xleftarrow{d^*_k} V_h^{k} \xleftarrow{d^*_{k+1}} V_h^{k+1}
\end{align*}
From the general Hilbert complex theory (Thm.\,\ref{thm:hodge_decomposition})
we thus obtain the \textit{discrete Hodge decomposition}
\begin{align*}
    V_h^j = \mathfrak{B}^j_h \stackrel{\perp}{\oplus} \mathfrak{H}^j_h \stackrel{\perp}{\oplus}
        \mathfrak{B}^*_{jh}.
\end{align*}
So we achieved our goal of getting a structure like in the continuous case 
for our discrete approximation. Especially the question how well the discrete harmonic 
forms approximate the contiinous one will be looked at more closely.


The third crucial assumption is the existence of \textit{bounded cochain projections} $\pi_h$. 
This is a projection that is a cochain map in the sense of cochain complexes \ref{} i.e. 
the following diagram commutes:
% \begin{tikzcd}
%     A \arrow[r, "\phi"] 
%     & B \arrow[d, "\psi" red] 
% \end{tikzcd}    
$\pi_h$ are either bounded in the $V$ or in the $W$-norm where  
$W$-boundedness implies $V$ boundedness. 
The cochain projection will play an important 
role in the stablity of the discrete system.

Let us now answer the question about the difference between discrete and continous 
harmonic forms. In order to do that we need some way to measure the "difference" 
between two subspaces.

\begin{definition}[Gap between subspaces]
    For a Banach space $W$ with subspaces $Z_1$ and $Z_2$. 
    Let $S_1$ and $S_2$ be the unit spheres in $Z_1$ and $Z_2$ respectively i.e.
    $S_1 = \{ z\in Z_1 \mid \lVert z \rVert _W = 1 \}$.
    Then we define 
    the gap between these subspaces as 
    \begin{align*}
        \gap(Z_1, Z_2) = \max\{ \sup_{z_1 \in S_1} \text{dist}{z_1, Z_2}, \sup_{z_2 \in S_2} \text{dist}{z_2, Z_1} \}
    \end{align*}
\end{definition}
This definition is from \cite[p.198]{kato perturbation theory} and defines a metric on the set of closed subspaces
of $W$ (see \ref[Remark p.198]{})
If $W$ is a Hilbert space -- as it is throughout this section -- and $Z_1$ and $Z_2$ are closed then
the $\gap(Z_1, Z_2) = \lVert P_{Z_1} - P_{Z_2} \rVert$ i.e. the difference in operator norm of the 
orthogonal projections onto $Z_1$ and $Z_2$. This gives us a measure of distance between 
spaces which we can now apply to the question of the difference of the difference between 
discrete and continous harmonic forms.

\begin{proposition}[Gap between harmonic forms]
    Assume that the discrete complex \ref{} admits a $V$-bounded cochain projection
    $\pi_h$. Then
    \begin{align*}
        \lVert (I - P_{\mathfrak{H}^k_h}) q \rVert _V \leq \lVert (I - \pi_h^k) q \rVert _V, 
            \forall q \in \mathfrak{H}^k 
        \\ \lVert (I - P_{\mathfrak{H}^k}) q_h \rVert _V 
        \leq \lVert (I - \pi_h^k)P_{\mathfrak{H}^k} q \rVert _V, \forall q \in \mathfrak{H}^k, 
                \forall q_h \in \mathfrak{H}^k_h 
    \end{align*}
    and then 
    \begin{align*}
        \gap (\mathfrak{H}, \mathfrak{H}_h) 
        \leq \sup_{q \in \mathfrak{H}, \lVert q \rVert = 1} \lVert (I - \pi_h^k) q \rVert_V
    \end{align*}
\end{proposition}
\begin{proof}
    See \cite[Thm.\,5.2]{arnold}. 
\end{proof}
{\color{red} Do not forget the continuous poincare inequality}
\begin{proposition}[Discrete Poincare inequality]
    Assume that we have a $V$-bounded cochain projection $\pi_h$ for 
    the discrete Hilbert complex \ref{}. Then 
    \begin{align*}
        \lVert v \rVert _V \leq c_P \lVert \pi_h \rVert _V \lVert dv \rVert, 
            \quad \forall v \in \mathfrak{Z}_h^{k\perp}\cap V_h
    \end{align*}
    with $c_P$ being the Poincare constant from \ref{}.
\end{proposition}
\begin{proof}
    This indeed is a direct consequence of the existence of bounded cochain projections.
    Take $v_h \in \mathfrak{Z}_h^{k\perp}\cap V_h$ arbitrary. 
    Since $d (\mathfrak{Z}^{k,\perp} \cap V^k) = \mathfrak{B} \supseteq \mathfrak{B}_h$ we find 
    $z\in \mathfrak{Z}^{k\perp}\cap V_h$ s.t. $dz = dv$. We can apply now the continuous 
    Poincare inequality \ref{} to get $\lVert z \rVert _V \leq c_P \lVert dz \rVert _V = c_P \lVert dv_h \rVert _V$.
    Now we can combine the different assumptions about the discrete Hilbert complex teo get 
    $v_h - \pi_h z \in V_h^k$. Now we can use the fact that $\pi_h$ is a cochain map 
    and the fact that $\pi_h$ is a projection:
    \begin{align*}
        d\pi^k_h z = \pi^{k+1}_h dz = \pi^{k+1}_h dv_h = dv_h
    \end{align*}
    For the last equality we used also the fact that we have a discrete complex i.e. $d^k V^k_h \subseteq V^{k+1}_h$.
    That shows that $d(v_h - \pi_h z) = 0$ i.e. $(v_h - \pi_h z) \in \mathfrak{Z}_h^k$.
    Because $v_h \in \mathfrak{Z}_h^{k,\perp}$ by assumption we have 
    \begin{align*}
        0 = \langle v, v_h - \pi_h z \rangle = \langle v, v_h - \pi_h z \rangle + \langle dv, d(v_h - \pi_h z) \rangle
            = \langle v, v_h - \pi_h z \rangle _V
    \end{align*}
    so $v_h - \pi_h z$ is $V$ orthogonal to $v_h$. So 
    \begin{align*}
        \lVert v_h \rVert _V^2 = \langle v_h, \pi_h^k z \rangle _V + \langle v_h, v_h - \pi_h^k z\rangle _V 
        = \langle v_h, \pi_h^k z \rangle _V \leq \lVert \pi_h \rVert _V \lVert dv \rVert
        \stackrel{Poincare ineq.}{\leq} c_P \lVert \pi_h \rVert _V \lVert dv \rVert _V
    \end{align*}
\end{proof}

So we get the inf sup condition with $c_{P,h} = c_P \lVert \pi_h \rVert _V$ instead of $c_P$ 
and obtain well-posedness.

\section{Magnetostatic problem in 2D}

We are interested in solving the 2D version of the magnetostatic problem. 
Find $\mathbf{B} \in H(\curl) \cap H_0(\diver)$ s.t.
\begin{align*}
    \diver \mathbf{B} &= 0
    \\ \curl \mathbf{B} &= J.
\end{align*}
This problem is generally not-well posed since we need to add a harmonic constraint. 
One option is to take the orthogonality constraint again. But now we want to use the 
analogous problem that we dealt with in the first part of the thesis and add the 
constraint 
\begin{align*}
    
\end{align*}
\end{document}