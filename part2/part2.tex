\documentclass[../master_thesis.tex]{subfiles}
\begin{document}
\section{Variational formulation of the magnetostatic in 2D}

In 2D we have the Hilbert complex 
\begin{align*}
    H^1_0 \xrightarrow{\veccurl} H_0(\diver) \xrightarrow{\diver} L^2
\end{align*}
where we have the $\curl$ which is the scalar curl and 
$\mathbf{\curl}$ which is the vector valued curl defined as.

We use the notation introduced in \ref{} for general Hilbert complexes.

It is easily derived that we have the integration by parts formula if $\mathbf{v} \in C_b^1(\overline{\Omega})$.
Here $\Omega$ is from now on assumed bounded and Lipschitz.

The 2D magnetostatic problem is then
Let $J \in B^*_0$ be given. Then
\begin{problem}[2D magnetostatic problem]
    Find $\mathbf{B} \in H_0(\diver) \cap H(\curl)$ s.t.
    \begin{align*}
        \curl \mathbf{B} &= J,
        \\ \diver \mathbf{B} = 0,
        \text{+ additional constraints}
    \end{align*}
\end{problem}
The additional constraints are necessary to give a unique solution. 
We will focus, just as in the first part, on 
a curve integral as additional constraint. 
Another option would be an orthogonality constraint as in \ref{}.

\subsection{Mixed formulation}
In order to solve this problem numerically using finite elements we have to 
choose a suitable variational formulation of the problem. We will use the following
For any $J \in \curl H(\curl)$ find $\sigma \in H^1_0$, 
$B \in H_0(\diver)$ and $p \in \mathfrak{H}^1$ s.t.
\begin{align}
    \langle \sigma, \tau \rangle - \langle u, \mathbf{\curl}\tau \rangle 
        &=  -\langle J, \tau \rangle \quad \forall \tau \in H^1_0, \label{eq:first_eq_mixed_formulation}
    \\ \langle \mathbf{\curl}\sigma, \mathbf{v} \rangle + \langle \diver \mathbf{B}, \diver \mathbf{v} \rangle 
        + \langle \mathbf{p}, \mathbf{v} \rangle 
        &= 0 \quad \forall \mathbf{v} \in V^k \text{ and} \label{eq:second_eq_mixed_formulation}
        \text{+ additional constraints}
\end{align} 
This formulation is of course much more complicated than the first one \ref{}, 
but this formulation will turns out to be well-suited for finite element approximations.
But it begs the question if the two formulations are equivalent. We will answer it 
in two propositions. We will treat the harmonic constraint separately.

\begin{proposition}
    For any $J \in L^2$, (\ref{eq:first_eq_mixed_formulation}) and 
    (\ref{eq:second_eq_mixed_formulation}) hold i.i.f. 
    $\sigma = 0$, $\mathbf{p}=0$ and $\curl \mathbf{B} = J$ and 
    $\diver \mathbf{B} = 0$
    i.e. the magnetostatic \ref{}  is fulfilled without the additional 
    constraint.
\end{proposition}
\begin{proof}
    Assume $(\sigma,\mathbf{B},\mathbf{p})$ is a solution of (\ref{eq:first_eq_mixed_formulation}) and 
    (\ref{eq:second_eq_mixed_formulation}). Then the first equation is
    \begin{align*}
        \langle \sigma + J, \tau \rangle  
        &=  \langle \mathbf{B}, \mathbf{\curl}\tau \rangle  \quad \forall \tau \in H^1_0
    \end{align*}
    which is equivalent to $\mathbf{B} \in H(\curl)$ and $J + \sigma = \curl \mathbf{B}$.

    Now assume additionally, that  
    (\ref{eq:second_eq_mixed_formulation}) holds. Then by choosing $\mathbf{v} \mathfrak{p} \in \mathfrak{H}^1$
    we get from the definition of the harmonic forms 
    and $\mathfrak{H}^1 \perp \mathbf{\curl} H^1_0$ from the Hodge decomposition and thus
    \begin{align*}
        \langle \mathbf{\curl} \sigma, \mathfrak{p} \rangle + \langle du, d\mathfrak{p} \rangle + \langle p, \mathfrak{p} \rangle
        = \langle p, \mathfrak{p} \rangle = 0
    \end{align*}
    and so $\mathfrak{p} = 0$. Then we can choose $\mathbf{v} = \mathbf{\curl} \sigma$ to get 
    \begin{align*}
        \langle \mathbf{\curl} \sigma, \mathbf{\curl} \sigma \rangle + \langle \diver \mathbf{B}, \diver \mathbf{\curl} \sigma \rangle 
            + \langle \mathbf{p}, \mathbf{\curl} \sigma \rangle
        = \lVert \mathbf{\curl} \sigma \rVert^2.
    \end{align*}
    Because $\sigma \in H^1_0$ this gives us $\sigma = 0$. Also we have then 
    $J = \curl \mathbf{B}$. At last we choose $\mathbf{v} = \mathbf{B}$ which gives us 
    $\diver \mathbf{B} = 0$ and thus we proved the first direction. 

    The other implication is clear i.e. if $\mathbf{B} \in H(\curl) \cap H_0(\diver)$
    with $\curl \mathbf{B} = J$ and $\diver \mathbf{B} = 0$ then the variational 
    formulation clearly holds.
    \end{proof}
If we now add the same additional constraints to both formulations of the problem 
then they will remain equivalent.


\subsection{Curve integral constraint}

From now on, we assume that the space of harmonic forms $\mathfrak{H}^1$ has dimension 
one.
We want to add now a curve integral constraint. We first want to rewrite the curve 
integral in variational form. 

Let $\Gamma$ be a closed curve with parametrization $\gamma:[0,|\Gamma|]$ s.t. 
$|\gamma'(t)| = 1$ and assume that $\gamma$ is bijective i.e. the curve does not 
intersect itself. Since $\Gamma$ is a closed curve 
it encompasses a set $K$ {\color{red} Add drawing} Let $\mathbf{n}$ be the 
unit normal of $K$. Then we know that $\mathbf{n} \perp \gamma'$. 

If we now take $\mathbf{B}$ that 
\begin{align*}
    n \times \mathbf{B} =   -\mathbf{B}\times n 
    = - (B_1 n_2 - B_2 n_1) = - \mathbf{B} \cdot \begin{pmatrix}n_2 \\ -n_1 \end{pmatrix}
    = \mathbf{B} \cdot R_{\pi/2}\mathbf{n}
\end{align*}
then $R_{\pi/2} \mathbf{n}$ is either $\gamma'$ or $-\gamma'$. Assume w.l.o.g.
that $R_{\pi/2} \mathbf{n} = \gamma'$ and thus 
\begin{align*}
    \mathbf{B} \cdot R_{\pi/2} \mathbf{n} = \mathbf{B} \cdot \gamma'.
\end{align*}

So we see that we can write 
\begin{align*}
    n \times \mathbf{B} = \mathbf{B} \cdot \gamma'
\end{align*}
and so the curve integral becomes

\begin{align*}
    \int_\Gamma \mathbf{B}\cdot dl= \int_0^{|\Gamma|} \mathbf{B}(\gamma(t)) \cdot \gamma'(t) \, dt 
    = \int_0^{|\Gamma|} n(\gamma(t)) \times \mathbf{B}(\gamma(t)) \, dt 
    = \int_\Gamma n \times \mathbf{B}.
\end{align*}

Assume now that $K$ is an annulus type domain. More precisely, there are two 
disjoint closed curve $\partial \Omega_{in}$ and $\partial \Omega_{out}$ that $\Omega$. 
$\Omega_\gamma$ as the set that is encompassed by $\partial \Omega_{in}$ and $\gamma$.
{\color{red} Drawing!}

Defie $\psi$ s.t. $\psi = 0$ on $\partial \Omega_{in}$ and $\psi = 1$ on $\Gamma$ 
and constant one outside
Then we observe 

\begin{align*}
    \int_\Omega \mathbf{\curl} \psi \cdot \mathbf{B} \, dx = 
    \int_\Omega \psi \, J \, dx - \int_\partial\Omega n\times \mathbf{B} \, dl
    = \int_\Omega \psi \, J \, dx - \int_\Gamma \mathbf{B}\cdot dl
\end{align*}

Note that even though right hand side requires some regularity for $\mathbf{B}$
the left hand side makes sense even if $\mathbf{B}$ is only in $L^2$!
So if we are in a situation where the we have curve integral given then 
we can add this constraint like this. 

Let us assume we are given that the curve integral 
\begin{align*}
    \int_\Gamma \mathbf{B} \cdot dl = C_0
\end{align*}
assuming this makes sense. Then we choose $\psi$ and then get 
the constraint
\begin{align*}
    \langle \mathbf{\curl} \psi, \mathbf{B} \rangle = \langle J, \psi \rangle - C_0.
\end{align*}
Note that there are not test functions involved since $\psi$ is fixed. 
We define $C_1 \vcentcolon=  \langle J, \psi \rangle - C_0$. Then to get a variational 
formulation we multiply \ref{} with an arbitrary $\mu \in \real$. Then we reformulate 
the mixed variational form slightly. 

Let $J \in L^2$, $\mathbf{p} \in \mathfrak{H}^1 \setminus \{0\}$. 
Find $\sigma \in H^1_0$, $\mathbf{B} \in H_0(\diver)$, $\lambda \in \real$ s.t.
\begin{align}
    \langle \sigma, \tau \rangle - \langle u, \mathbf{\curl}\tau \rangle 
    &=  -\langle J, \tau \rangle \quad \forall \tau \in H^1_0, \label{eq:first_eq_mixed_formulation_curve_integral}
    \\ \langle \mathbf{\curl}\sigma, \mathbf{v} \rangle + \langle \diver \mathbf{B}, \diver \mathbf{v} \rangle 
    + \langle \lambda \mathbf{p}, \mathbf{v} \rangle 
    &= 0 \quad \forall \mathbf{v} \in V^k, \label{eq:second_eq_mixed_formulation_curve_integral}
    \\ \mu \langle \bm{\curl} \psi, \mathbf{B} \rangle &= \mu C_1 \quad \forall \mu \in \real.
\end{align}
which gives us the variational formulation of the magnetostatic problem with curve integral 
constraint. We will study the well-posedness of this formulation next. 
Using the analogous reasoning as in \ref{} we see that the first two equations are still equivalent 
to the magnetostatic problem without additional constraint.

Defining $X \vcentcolon= H^1_0 \times H_0(\diver) \times \real$ and the 
bilinear form $a:X\times X \rightarrow \real$

\begin{align*}
    a(\sigma,\mathbf{B}, \lambda;\tau,\mathbf{v},\mu) 
    =   \langle \sigma, \tau \rangle - \langle u, \mathbf{\curl}\tau \rangle
        + \langle \mathbf{\curl}\sigma, \mathbf{v} \rangle + \langle \diver \mathbf{B}, \diver \mathbf{v} \rangle 
        + \langle \lambda \mathbf{p}, \mathbf{v} \rangle - \mu \langle \mathbf{\curl} \psi, \mathbf{B} \rangle.
\end{align*}
This allows us to rewrite \ref{} in the standard form
\begin{align*}
    a(\sigma,\mathbf{B},\lambda;\tau,\mathbf{v},\mu) = -\langle J, \tau \rangle - \mu C_1
        \quad \forall (\tau,\mathbf{v},\mu) \in X.
\end{align*}
Note that the bilinear form $a$ is not symmetric. 

As stated in the proof of the Poincare inequality \ref{} 
the $\veccurl| _{\mathfrak{Z}^\perp}: \mathfrak{Z}^\perp \rightarrow \mathfrak{B}^1$
is bijective and since it is bounded w.r.t. the $V$-norm -- which is the 
$H^1$-norm here -- due to Banach inverse theorem it is invertible and we denote this 
inverse $\veccurl^{-1}$. This is a slight abuse of notation since it is not really 
the inverse of the full $\veccurl$.

\begin{lemma}
    Assume $\veccurl \psi = \veccurl \psi_0 + c_\psi \mathbf{p}$. Assume w.l.o.g. $c_\psi >0$
    (otherwise we can choose the representative of $\mathfrak{H}^1$ with the opposite sign).
    Define $T:X \rightarrow X$ as 
    \begin{align*}
        T(\sigma, \mathbf{B},\lambda)
        = (\sigma - \frac{1}{c_P^2}\veccurl^{-1}, \veccurl \sigma + \mathbf{B} + \lambda \beta \mathbf{p},
            -\alpha \langle \mathbf{p}, \alpha \langle \mathfrak{p}, \mathbf{B} \rangle  
            + \frac{\lambda}{c_\psi} \rangle).
    \end{align*}
    Then $T$ is surjective. 
\end{lemma}
\begin{proof}
    Take $(\tau, \mathbf{v},\mu) \in X$ arbitrary. Then 
    Now we choose $\sigma = (1+1/c_P)^{-1} (\tau + (1/c_P^2) \veccurl^{-1} \mathbf{v})$ 
    and $\mathbf{B}_\mathfrak{B} = \mathbf{v}_\mathfrak{B} - \veccurl \sigma$. 
    So 
    \begin{align*}
        \sigma -  1/c_P^2 \veccurl^{-1}\mathbf{B}_\mathfrak{B} 
        = \sigma -  1/c_P^2 (\veccurl^{-1} \mathbf{v} - \sigma)
        = (1+1/c_P^2)\sigma - 1/c_P^2 \veccurl^{-1} \mathbf{v}
        = \tau.
    \end{align*}
    We simply choose $\mathbf{B}_\mathfrak{B^*} = \mathbf{v}_\mathfrak{B^*}$.
    For the harmonic part we observe for $\mathbf{v}_\mathfrak{H} = c_v p$
    Let us look at the system 
    \begin{align*}
        \begin{pmatrix}
            1 & \beta 
            \\ \alpha & 1/c_\psi
        \end{pmatrix}
        \begin{pmatrix}
            \kappa_u 
            \\ \lambda 
        \end{pmatrix}
        = 
        \begin{pmatrix}
            c_v 
            \\ \mu
        \end{pmatrix}
    \end{align*}
    Now since $c_\psi > 0$ and $\alpha < 0$, $\beta > 0$ we get 
    $1/c_\psi - \alpha \beta \neq 0$ and the system has a solution. 
    Then we see 
    \begin{align*}
        \mathbf{v}_\mathfrak{H} = c_v p = p(\kappa_u + \beta \lambda) 
        =  \mathbf{B}_\mathfrak{H} + \beta \lambda p
    \end{align*}
    and 
    \begin{align*}
        \mu = \alpha \kappa_u + 1/c_\psi \lambda
        = \alpha \kappa_u \lVert p \rVert^2 + 1/c_\psi \lambda 
        = \alpha \langle \mathbf{B}, \mathbf{p} \rangle + 1/c_\psi \lambda.
    \end{align*}
    And so in comining all that we arrive at 
    $T(\sigma,\mathbf{B}, \mathbf{p}) = (\tau, \mathbf{v}, \mathbf{p})$.
\end{proof}

\begin{theorem}
    $a$ satisfies a inf-sup condition with $\gamma$ depending on the Poincaré constant 
    as well as $\psi$.
\end{theorem}
\begin{proof}
    We will use T-coercivity to prove it. Choose $(\sigma, \mathbf{B},\lambda) \in X$ 
    arbitrary and define $\rho \vcentcolon= \veccurl^{-1} \mathbf{B}_{\mathfrak{B}}$
    \begin{align*}
        T(\sigma, \mathbf{B},\lambda)
        = (\sigma - \frac{1}{c_P^2}\rho, \veccurl \sigma + \mathbf{B} + \beta \mathbf{p},
            -\alpha \langle \mathbf{p}, \alpha \langle \mathfrak{p}, \mathbf{B} \rangle  + \frac{\lambda}{c_\psi} \rangle)
    \end{align*}
    with $\beta = \frac{3 c_1^2 c_P^2}{c_\psi^2}$ and $\alpha = -\frac{c_\psi}{4 c_1^2 c_P^2}$.
    Take $c_1 > 0$ s.t. $\lVert \veccurl \psi_0 \rVert \leq c_1$ (e.g. one could choose 
    $c_1 = \lVert \veccurl \psi_0 \rVert + 1)$.
    Then $T$ is surjective \ref{}. We split up $d\psi = d\psi_0 + c_\psi \mathfrak{p}$ to get 
    \begin{align*}
        &a(\sigma, \mathbf{B},\lambda;T(\sigma, \mathbf{B},\lambda))
        \\ &=\langle \sigma, \sigma - \frac{1}{c_P^2} \rho \rangle 
            - \langle \mathbf{B}, \veccurl \sigma - \frac{1}{c_P^2}\veccurl \rho \rangle
            + \langle \diver \mathbf{B}, \diver \veccurl 
        \\ &\quad + \sigma \diver \mathbf{B} + \beta \lambda \mathfrak{p}\rangle
            + \langle \lambda \mathbf{p}, \veccurl \sigma + \mathbf{B} + \beta \lambda \mathfrak{p}\rangle
            - (\alpha \langle \mathbf{B}, \mathbf{p} \rangle + \frac{\lambda}{c_\psi})
            \langle \mathbf{B} , \veccurl\psi \rangle
        \\ &= \lVert \sigma \rVert^2 - \frac{1}{c_P^2} \langle \sigma , \rho \rangle 
            + \frac{1}{c_P^2} \lVert B_\mathfrak{B} \rVert^2 + \lVert  \veccurl \sigma \rVert^2
            + \lVert \diver \mathbf{B} \rVert ^2 + \lambda ^2 \beta - \alpha c_\psi \lVert \mathbf{B}_\mathfrak{H} \rVert^2
        \\ &\quad- \alpha \langle \mathfrak{p}, \mathbf{B} \rangle \langle \mathbf{B}, \veccurl \psi_0 \rangle
            - \frac{\lambda}{c_\psi} \langle B_\mathfrak{B} , \veccurl \psi_0 \rangle
        \\ &...\geq \lVert \sigma \rVert^2 - 
            \left( \frac{1}{2} \lVert \sigma \rVert^2 
            + \frac{\lVert B_\mathfrak{B} \rVert^2}{2 c_P^2}  \right)
            + \frac{1}{c_P^2} \lVert B_\mathfrak{B} \rVert^2
            + \lVert  \veccurl \sigma \rVert^2 + \lVert \diver \mathbf{B} \rVert ^2
        \\ &\quad+ \lambda ^2 \beta - \alpha c_\psi \lVert \mathbf{B}_\mathfrak{H} \rVert^2
            - \left( \frac{\epsilon_1 \alpha^2 \lVert \mathbf{B}_\mathfrak{H} \rVert^2}{2} 
            + \frac{\lVert \mathbf{B}_\mathfrak{B} \rVert^2 \lVert  \veccurl \psi_0 \rVert^2}{2 \epsilon_1} \right)
            - \left( \frac{\lambda^2}{2 \epsilon_2 c_\psi^2} 
            + \frac{\epsilon_2 \lVert \mathbf{B}_\mathfrak{B} \rVert^2 \lVert  \veccurl \psi_0 \rVert^2}{2} \right)
    \end{align*}
    Choose $\epsilon_1 = 4 c_1^2 c_P^2$ to get 
    \begin{align*}
        &\frac{1}{2} \lVert \sigma \rVert^2 + \frac{1}{2 c_P^2} \lVert B_\mathfrak{B} \rVert^2
        + \lVert  \veccurl \sigma \rVert^2 + \lVert \diver \mathbf{B} \rVert ^2
        + \lambda ^2 \left( \beta - \frac{1}{2 \epsilon_2 c_\psi^2} \right) 
        \\ &\quad+ \lVert \mathbf{B}_\mathfrak{H} \rVert^2 
        \left( - \alpha c_\psi - \frac{4 c_1^2 c_P^2 \alpha^2}{2} \right)
        - \lVert B_\mathfrak{B} \rVert^2 \frac{\lVert  \veccurl \psi_0 \rVert^2}{8c_1^2 c_P^2}
        - \lVert B_\mathfrak{B} \rVert^2 \frac{ \epsilon_2 \lVert  \veccurl \psi_0 \rVert^2 }{2}
    \end{align*}
    Now choose $\epsilon_2 = \frac{1}{4 c_1^2 c_P^2}$ and plug in the definition of $\alpha$
    to get bound it from below with
    \begin{align*}
        &\frac{1}{2} \lVert \sigma \rVert^2 + \lVert B_\mathfrak{B} \rVert^2 
        \left( \frac{1}{2 c_P^2} - \frac{1}{8 c_P^2} 
        - \frac{\lVert  \veccurl \psi_0 \rVert^2}{8 c_1^2 c_P^2} \right)
        + \lVert  \veccurl \sigma \rVert^2 + \lVert \diver \mathbf{B} \rVert ^2
        + \lambda ^2 \left( \beta - \frac{4 c_1^2 c_P^2}{2 c_\psi^2} \right)
        \\ &\quad+ \lVert \mathbf{B}_\mathfrak{H} \rVert^2  \left( \frac{c_\psi^2}{4 c_1^2 c_P^2 }
        - \frac{ c_1^2 c_P^2 c_\psi^2}{8 c_1^4 c_P^4} \right)
    \end{align*}
    and finally by using .... and $\beta = \frac{3 c_1^2 c_P^2}{c_\psi^2}$
    \begin{align*}
        \frac{1}{2} \lVert \sigma \rVert^2 + \frac{1}{4 c_P^2}\lVert B_\mathfrak{B} \rVert^2 
        + \lVert  \veccurl \sigma \rVert^2
        + \frac{1}{2 c_P^2 B_{\mathfrak{B}^*}} + \frac{c_\psi}{8 c_1^2 c_P^2} 
        + \frac{1}{2} \lVert \diver \mathbf{B} \rVert ^2
        + \frac{c_1^2 c_P^2}{c_\psi^2}\lambda^2
        + \frac{c_\psi^2}{8 c_1^2 c_P^2} \lVert \mathbf{B}_\mathfrak{H} \rVert^2 
    \end{align*}
\end{proof}

\begin{theorem}[Stability]
    The system is stable. For solution $(\sigma, \mathbf{B},\mathbf{p}) \in X$
    we get 
    \begin{align*}
        \lVert \sigma \rVert _V + \lVert \mathbf{B} \rVert _V + |\lambda|
        \leq \frac{\lVert J \rVert + |C_1|}{\gamma}.
    \end{align*}
\end{theorem}
\begin{proof}
    The statement follows immediately from \ref{} and the fact that 
    \begin{align*}
        | l(\tau,\mathbf{v},\mu) |
        = | - \langle J, \tau \rangle - C_1 \mu | 
        \leq (\lVert J \rVert + C_1) \lVert \tau,\mathbf{v},\mu \rVert _X
    \end{align*}
    and thus $\lVert l \rVert _{X'} \leq \lVert J \rVert + | C_1 |$.
\end{proof}

\section{Discrete Hilbert complex}

In order to approximate the Hodge Laplacian problem we want to use finite elements.
We want to use them in a way that we can rebuild the structure of the Hilbert complex 
in our discretization. This section follows Sec. 5.2 in Arnold's book \cite{arnold}.

Recall situation

Let us assume the we have finite dimensional subspaces $V_h^k \subseteq V^k$. 
Then we define completely analogous to the continuous case,
\begin{align*}
    \mathfrak{Z}_h^k &\vcentcolon= \{ v \in V_h^k \mid d v = 0 \} = \ker d \cap V^k_h
    \\ \mathfrak{B}^k_h &\vcentcolon= \{ dv \mid v \in V_h^{k-1} \}.
\end{align*}
We can now also define the discrete harmonic forms. Now the situation is slightly 
different however. We will not use the continuous adjoint $d^*_k$ to define it.
Instead,
\begin{align*}
    \mathfrak{H}_h^k \vcentcolon= \{ v \in \mathfrak{Z}_h^k \mid v \perp \mathfrak{B}^k_h \}
        = \mathfrak{Z}_h^k \cap \mathfrak{B}_h^{k,\perp}.
\end{align*}
Notice that we have $\mathfrak{Z}_h^k \subseteq \mathfrak{Z}^k$ and 
$\mathfrak{B}_h^k \subseteq \mathfrak{b}^k$, but due to 
$\mathfrak{B}_h^{k,\perp} \supseteq \mathfrak{B}^{k,\perp}$ we have in general
\begin{align*}
    \mathfrak{H}^k = \mathfrak{Z}_h^k \cap \mathfrak{B}_h^{k,\perp} 
    \not\subseteq    \mathfrak{Z}^k \cap \mathfrak{B}^{k,\perp} = \mathfrak{H}^k.
\end{align*}


There are three crucial properties that are necessary for stability and convergence 
of the method. The first one is the common and reasonable assumption that 
-- as usual in finite element theory -- we want that the discrete spaces $V_h^k$
approximate the continuous ones $V^j$. This can be generally summarized that 
\begin{align*}
    \lim_{h \rightarrow 0} \inf_{v_h \in V_h^j} \lVert w - v_h \rVert = 0, \quad \forall w \in V^j.
\end{align*}
This is usually satisfied if we use established finite elements for a given space 
e.g. if we take Lagragian FE if $V = H^1$ or Raviart-Thomas if $V=H(\diver)$ \cite{}.

The next property is more restrictive. We require that $dV_h^{k-1} \subseteq V_h^k$ 
and $dV_h^j \subseteq V_h^{j+1}$. This shows that the we cannot simply use arbitrary 
discrete subspaces independent from one another. We say the spaces have to be 
compatible \cite{}. This property has a very nice
consequence. 
It shows that 
\begin{align*}
    V_h^{k-1} \xrightarrow{d^{k-1}} V_h^k \xrightarrow{d^k} V_h^{k+1}
\end{align*}
is itself a Hilbert complex and we can apply the general theory from 
Sec. \ref{sec:hilbert_complexes} directly to it. Let us do that.

Denote the restriction of $d^j$ to $V_h^j$ as $d_h^j$. Then as a linear map 
between finite spaces the adjoint -- denoted as $d_{j,h}^*: V_h^j \rightarrow V_h^{j-1}$ -- 
is everywhere defined. It is important to notice that in contrast to $d_h$ 
the adjoint $d^*_jh$ is not the restriction of the adjoint the continuous adjoint $d^*_j$.
In general, $V_h \not\subseteq V^*$ and so the continuous adjoint might not be 
well-defined for a given $v_h \in V_h$. 

So we obtain the Hilbert complex
\begin{align*}
    V_h^{k-1} \xrightarrow{d^{k-1}} V_h^{k} \xrightarrow{d^{k}} V_h^{k+1}
\end{align*}
and its dual complex
\begin{align*}
    V_h^{k-1} \xleftarrow{d^*_k} V_h^{k} \xleftarrow{d^*_{k+1}} V_h^{k+1}
\end{align*}
From the general Hilbert complex theory (Thm.\,\ref{thm:hodge_decomposition})
we thus obtain the \textit{discrete Hodge decomposition}
\begin{align*}
    V_h^j = \mathfrak{B}^j_h \stackrel{\perp}{\oplus} \mathfrak{H}^j_h \stackrel{\perp}{\oplus}
        \mathfrak{B}^*_{jh}.
\end{align*}
So we achieved our goal of getting a structure like in the continuous case 
for our discrete approximation. Especially the question how well the discrete harmonic 
forms approximate the contiinous one will be looked at more closely.


The third crucial assumption is the existence of \textit{bounded cochain projections} $\pi_h$. 
This is a projection that is a cochain map in the sense of cochain complexes \ref{} i.e. 
the following diagram commutes:
% \begin{tikzcd}
%     A \arrow[r, "\phi"] 
%     & B \arrow[d, "\psi" red] 
% \end{tikzcd}    
$\pi_h$ are either bounded in the $V$ or in the $W$-norm where  
$W$-boundedness implies $V$ boundedness. 
The cochain projection will play an important 
role in the stablity of the discrete system.

Let us now answer the question about the difference between discrete and continous 
harmonic forms. In order to do that we need some way to measure the "difference" 
between two subspaces.

\begin{definition}[Gap between subspaces]
    For a Banach space $W$ with subspaces $Z_1$ and $Z_2$. 
    Let $S_1$ and $S_2$ be the unit spheres in $Z_1$ and $Z_2$ respectively i.e.
    $S_1 = \{ z\in Z_1 \mid \lVert z \rVert _W = 1 \}$.
    Then we define 
    the gap between these subspaces as 
    \begin{align*}
        \gap(Z_1, Z_2) = \max\{ \sup_{z_1 \in S_1} \text{dist}{z_1, Z_2}, \sup_{z_2 \in S_2} \text{dist}{z_2, Z_1} \}
    \end{align*}
\end{definition}
This definition is from \cite[p.198]{kato perturbation theory} and defines a metric on the set of closed subspaces
of $W$ (see \ref[Remark p.198]{})
If $W$ is a Hilbert space -- as it is throughout this section -- and $Z_1$ and $Z_2$ are closed then
the $\gap(Z_1, Z_2) = \lVert P_{Z_1} - P_{Z_2} \rVert$ i.e. the difference in operator norm of the 
orthogonal projections onto $Z_1$ and $Z_2$. This gives us a measure of distance between 
spaces which we can now apply to the question of the difference of the difference between 
discrete and continous harmonic forms.

\begin{proposition}[Gap between harmonic forms]
    Assume that the discrete complex \ref{} admits a $V$-bounded cochain projection
    $\pi_h$. Then
    \begin{align*}
        \lVert (I - P_{\mathfrak{H}^k_h}) q \rVert _V \leq \lVert (I - \pi_h^k) q \rVert _V, 
            \forall q \in \mathfrak{H}^k 
        \\ \lVert (I - P_{\mathfrak{H}^k}) q_h \rVert _V 
        \leq \lVert (I - \pi_h^k)P_{\mathfrak{H}^k} q \rVert _V, \forall q \in \mathfrak{H}^k, 
                \forall q_h \in \mathfrak{H}^k_h 
    \end{align*}
    and then 
    \begin{align*}
        \gap (\mathfrak{H}, \mathfrak{H}_h) 
        \leq \sup_{q \in \mathfrak{H}, \lVert q \rVert = 1} \lVert (I - \pi_h^k) q \rVert_V
    \end{align*}
\end{proposition}
\begin{proof}
    See \cite[Thm.\,5.2]{arnold}. 
\end{proof}
{\color{red} Do not forget the continuous poincare inequality} This then implies 
that there is a quasi optimal kind of bound of the gap
\begin{proposition}[Discrete Poincare inequality]
    Assume that we have a $V$-bounded cochain projection $\pi_h$ for 
    the discrete Hilbert complex \ref{}. Then 
    \begin{align*}
        \lVert v \rVert _V \leq c_P \lVert \pi_h \rVert _V \lVert dv \rVert, 
            \quad \forall v \in \mathfrak{Z}_h^{k\perp}\cap V_h
    \end{align*}
    with $c_P$ being the Poincare constant from \ref{}.
\end{proposition}
\begin{proof}
    This indeed is a direct consequence of the existence of bounded cochain projections.
    Take $v_h \in \mathfrak{Z}_h^{k\perp}\cap V_h$ arbitrary. 
    Since $d (\mathfrak{Z}^{k,\perp} \cap V^k) = \mathfrak{B} \supseteq \mathfrak{B}_h$ we find 
    $z\in \mathfrak{Z}^{k\perp}\cap V_h$ s.t. $dz = dv$. We can apply now the continuous 
    Poincare inequality \ref{} to get $\lVert z \rVert _V \leq c_P \lVert dz \rVert _V = c_P \lVert dv_h \rVert _V$.
    Now we can combine the different assumptions about the discrete Hilbert complex teo get 
    $v_h - \pi_h z \in V_h^k$. Now we can use the fact that $\pi_h$ is a cochain map 
    and the fact that $\pi_h$ is a projection:
    \begin{align*}
        d\pi^k_h z = \pi^{k+1}_h dz = \pi^{k+1}_h dv_h = dv_h
    \end{align*}
    For the last equality we used also the fact that we have a discrete complex i.e. $d^k V^k_h \subseteq V^{k+1}_h$.
    That shows that $d(v_h - \pi_h z) = 0$ i.e. $(v_h - \pi_h z) \in \mathfrak{Z}_h^k$.
    Because $v_h \in \mathfrak{Z}_h^{k,\perp}$ by assumption we have 
    \begin{align*}
        0 = \langle v, v_h - \pi_h z \rangle = \langle v, v_h - \pi_h z \rangle + \langle dv, d(v_h - \pi_h z) \rangle
            = \langle v, v_h - \pi_h z \rangle _V
    \end{align*}
    so $v_h - \pi_h z$ is $V$ orthogonal to $v_h$. So 
    \begin{align*}
        \lVert v_h \rVert _V^2 = \langle v_h, \pi_h^k z \rangle _V + \langle v_h, v_h - \pi_h^k z\rangle _V 
        = \langle v_h, \pi_h^k z \rangle _V \leq \lVert \pi_h \rVert _V \lVert dv \rVert
        \stackrel{Poincare ineq.}{\leq} c_P \lVert \pi_h \rVert _V \lVert dv \rVert _V
    \end{align*}
\end{proof}

So we get the inf sup condition with $c_{P,h} = c_P \lVert \pi_h \rVert _V$ instead of $c_P$ 
and obtain well-posedness.

\section{Discretized magnetostatic problem}

Let us apply this discretized Hilbert complex to the 2D Hilbert complex \ref{}
to get $V^0_h \subseteq H^1_0$, $V^1_h \subseteq H_0(\diver)$ and $V^2_h \subseteq L^2$
with 
\begin{align*}
    V^0_h \xrightarrow{\veccurl} V^1_h \xrightarrow{\diver} V^2_h
\end{align*}
and the dual complex 
\begin{align*}
    V^0_h \xleftarrow{\widetilde{\curl}_h} V^1_h \xleftarrow{-\widetilde{\grad}_h} V^2_h
\end{align*}
where $\widetilde{\curl}_h$ is the adjoint of $\veccurl_h$ and corresponds thus to
weak form of $\curl$ and the same for $\widetilde{\grad}_h$.
Analogous to the continuous case we assume that $\dim \mathfrak{H}_h^1 = 1$ 
which is not unreasonable thanks to \ref{} for $h>0$ small enough.
The discretized version of the magnetostatic problem then states: 
Find $\mathbf{B}_h \in V_h^1$ s.t.
\begin{align*}
    \weakcurl_h \mathbf{B} = J \text{ and }
    \diver \mathbf{B} = 0.
\end{align*}
Note that the divergence is enforced strongly while the curl is only enforced weakly.
As explained in \ref{} we will add the curve integral constraint as in \ref{}.
This gives us the following discrete formulation. Find 
$\sigma_h \in V_h^0$, $\mathbf{B} \in V_h^1$ and $\lambda \in \real$ s.t.

\begin{align*}
    \langle \sigma_h, \tau_h \rangle - \langle \mathbf{B}_h, \mathbf{\curl}\tau_h \rangle 
    &=  -\langle J, \tau_h \rangle \quad \forall \tau_h \in V_h^0, 
    \\ \langle \mathbf{\curl}\sigma_h, \mathbf{v}_h \rangle + \langle \diver \mathbf{B}_h, \diver \mathbf{v}_h \rangle 
    + \langle \lambda \mathbf{p}_h, \mathbf{v}_h \rangle 
    &= 0 \quad \forall \mathbf{v}_h \in V^1_h, 
    \\ \mu \langle \bm{\curl} \psi, \mathbf{B}_h \rangle &= \mu C_1 \quad \forall \mu \in \real.
\end{align*}
Here we assume for simplicity that $\veccurl \psi \in V_h^1$. Since 
we can choose $\psi$ this is not unreasonable. 

Note that this trial and test space is indeed conforming, but we choose a discrete harmonic form 
$\mathbf{p}_h \in \mathfrak{H}^1_h$ so the resulting bilinear forms are different. 
The stability now follows from the exact same arguments as in \ref{} 
so we obtain a inf sup condition with a different constant $\gamma_h$ that involves 
$c_{P,h}$ from \ref{}. 

\section{Implementation in 2D}

\subsection{Splines}

For the discretization we will use the push forward of tensor product splines 
on a rectangular reference domain $\hat{\Omega}$. We use geometric degrees of freedom
which we will introduce below \ref{}. This section is a recollection 
 of \cite[Sec. 4.2]{broken FEEC framework on mapped multipatch} 
since we use the same method as presented in this paper also to fix notation. 

We will use two different types knot sequences, non-periodic and periodic ones. 
We choose a knot sequence $\{ \xi_i \}_{i=0}^{n+p}$ with 
$\xi_0 \leq \xi_1 \leq ... \xi_{n+p}$. We choose two types of sequences.

Let us define $x_0 < x_1 < ... < x_N$ as the physical knots which 
is our actual grid. We will stick to the equidistant case. Let $h$ be 
$x_{i+1} - x_i$.

For the non-periodic case we choose an \textit{open} knot sequence by 
$\xi_0 = ... = \xi_p = x_0$, $\xi_{p+l} = x_l$ for $l = 0,1,...,N$ and 
$\xi_{n} = \xi_{n+1} = ... = \xi_{n+p} = x_N$
\begin{align*}
    \xi_0 = ... = \xi_p < \xi_{p+1} < ... \xi_{n} = \xi_{n+1} = ... = \xi_{n+p}    
\end{align*}
and for the periodic case 
$\xi_0 = x_0-ph, \xi_1 = x_0-(p-1)h, ..., \xi_p = x_0$,
$\xi_{p+l} = x_l$ and $\xi_{n+l} = x_N + lh$ for $l = 0, ..., p$. 
Thus we always have $N+p = n$. 

\begin{align*}
    0 =  \xi_0 = \xi_1 = ... = \xi_p < \xi_{p+1} < \xi_{p+2} < ... < \xi_{n-1} < \xi_n 
    = \xi_{n+1} = ... = \xi_{n+p}
\end{align*}
with $n = N + p$ where $N$ is the number of cells.
Note that all the knot multiplicities in the interior are one and thus our spline 
space has maximal regularity. We then define 
$\mathcal{N}_i^q$ be the normalized B-spline \cite[Ref. 46, Def.4.19]{multipatch paper}.
We then define the spline spce $\mathbb{S}_q = \mathbb{S}_q(\bm{\xi}) = 
span{\mathcal{N}_i^q \mid i = 0, ..., n-1}$. Since we have maximal regularity 
we get that 
\begin{align*}
    \{ v \in C^{q-1} \mid v |_{\xi_{q+j, q+j+1}} \in \mathbb{P}_q \}.
\end{align*}
$\mathcal{N}_0^{p-1} $ vanishes.


We now can take tensor product of spline spaces. We use the notation 
with $\mathbf{q} \in \{p-1,p\}^2$ and we define with $\mathbf{i} \in [N_0 ] \times [ N_1 ]$
\begin{align*}
    \mathcal{N}_\mathbf{i}^\mathbf{q} \mathcal{N}_{i_1}^{q_1} \mathcal{N}_{i_2}^{q_2}.
\end{align*}
We write this as $\mathbb{S}_\mathbf{q} = \mathbb{S}_{q_1} \otimes \mathbb{S}_{q_2}$ 
The spline spaces used in the tensor product can also be periodic or only one of them 
can be periodic.
On $\hat{\Omega}$ we obtain the following discrete Hilbert complex 
\begin{align*}
    \mathbb{S}_{p,p} \xrightarrow{\veccurl} \mathbb{S}_{p-1,p} \xrightarrow{\diver} \mathbb{S}_{p-1,p-1}
\end{align*}
and we denote $\hat{V}^0 = \mathbb{S}_{p,p}$, $\hat{V}^1 = \mathbb{S}_{p-1,p}$ 
and $\hat{V}^2 = \mathbb{S}_{p-1,p-1}$. 

It is well-known that if we have a function $\mathbf{v}$ that is piecewise smooth 
then $\mathbf{v} \in H(\diver)$ i.i.f. the normal trace across element interfaces
agrees almost everywhere. Analogously for $\tau \in H^1$ i.i.f. the values agree 
on the interfaces almost everywhere. Since we always have $p \geq 2$ we thus know that 
all our tensor splines are at least continuous globally and thus we get 
$\hat{V}_h^j \subseteq \hat{V}^j$ for $j=0,1,2$ as desired. 

\subsection{Basis and degrees of freedom}


Let us now investigate the degrees of freedom. We will use geometric degrees of 
freedom i.e. each degree of freedom can be associated with some geometrical element of our 
domain. We define Greville points by 
\begin{align*}
    \zeta_i \vcentcolon= \frac{\xi_{i+1} + ... + \xi_{i+p}}{p}
\end{align*}
i.e. the knot averages for $i=0,...,n-1$. Then the spline interpolation at these 
points is well-defined (see \cite{I think that was in the isogeometric analysis book}).
Note that in the case periodic some Greville points lie outside of 
the physical grid. But since the function is periodic it can simply be extended periodically 
and then interpolated at these points. 

This gives us the following geometric elements nodes, edges and cells
\begin{align*}
    \hat{\mathtt{n}}_\mathbf{i} \vcentcolon= (\zeta_{i_1}, \zeta_{i_2}) for \mathbf{i} \in \hat{\mathcal{M}}^0
    \\ \hat{\mathtt{e}}_{d,\mathbf{i}} 
        \vcentcolon= [\hat{\mathtt{n}}_\mathbf{i}, \hat{\mathtt{n}}_{\mathbf{i} + \mathbf{e}_d}] 
        for (d, \mathbf{i}) \in \hat{\mathcal{M}}^1
    \\ \hat{\mathtt{n}}_\mathbf{i} \vcentcolon= [\hat{\mathtt{e}}_{1,\mathbf{i}}, \hat{\mathtt{e}}_{1,\mathbf{i}}]
        = [\zeta_{i_1}, \zeta_{i_1+1}] \times [\zeta_{i_2}, \zeta_{i_2+1}] 
        for \mathbf{i} \in \hat{\mathcal{M}}^2 
\end{align*}
with $[\cdot]$ being the convex hull. As before, $\mathbf{e}_d$ for $d = 1,2$ is the 
standard basis vector of $\real^2$. The set of multiindices are defined as
\begin{align*}
    \hat{\mathcal{M}}^0 \vcentcolon= [n-1]^2
    \\ \hat{\mathcal{M}}^1 \vcentcolon= \{ (d, \mathbf{i}=) \mid \mathbf{i} \in \hat{\mathcal{M}}^0, d\in \{1,2\} \}
    \\ \hat{\mathcal{M}}^2 \vcentcolon= [n-2]^2
\end{align*}
{\color{red} Does this stuff go through for periodic case?}

Now that we have defined the geometric elements we define the corresponding 
degrees of freedom as 
\begin{align*}
    \hat{\sigma_\mathbf{i}^0(v) } \vcentcolon= v(\hat{\mathtt{n}}_\mathbf{i}) for \mathbf{i} \in \hat{\mathcal{M}}^0
    % \\ \hat{\sigma_{\mathbf{d,i}^1(\mathbf{v})} } 
    %     \vcentcolon= \int_\hat{\mathtt{e}}_{d,\mathbf{i}} \mathbf{v} \cdot \mathbf{e}^\perp_d 
    %     for (d, \mathbf{i}) \in \hat{\mathcal{M}}^1
    \\ \hat{\sigma}_{\mathbf{i}}^2(v) 
        \vcentcolon= \int_{\hat{c}_{\mathbf{i}}} v 
        for (\mathbf{i}) \in \hat{\mathcal{M}}^2
\end{align*}
We $\mathbf{e}_d^\perp$ as the rotation of $\mathbf{e}_d$ by $\pi_2$ in counter clockwise 
direction i.e. $\mathbf{e}^\perp_1= \mathbf{e}_2$ and $\mathbf{e}^\perp_2= -\mathbf{e}_1$.  
Something about orientation


These degrees of freedome are unisolvent (explanation) i.e. 
with some ordering $\mu_0, \mu_1, ...$ of the indices of $\mathcal{M}^l$
\begin{align*}
    (\sigma^l_{\mu_0}, \sigma^l_{\mu_1}, ..., \sigma^l_{\mu_{|\mathcal{M}^l|}}):
    V^l_h \rightarrow \real^{|\mathcal{M}^l}
\end{align*}
is bijective.
And we can thus define our basis functions 
$\hat{\Lambda}^l_\mu$, $\mu \in \mathcal{M}^l$ as the basis which is dual to the 
degrees of freedom in the sense 
\begin{align*}
    \hat{\sigma}_\mu^l(\hat{\Lambda}^l_\nu) = \delta_{\mu,\nu} \quad for \mu, \nu \in \mathcal{M}^l.
\end{align*}

The question is now  on what function spaces these degrees of freedom are defined. We note first that 
the standard choice as described above with $\hat{V}^0 = H^1_0(\hat{\Omega})$ 
$\hat{V}^1 = H_0(\diver)$ and $\hat{V}^2 = L^2(\hat{\Omega})$ can not work 
because the evaluation at point values is not well-defined for $H^1$ in 2D since 
they can not be embedded into the continuous functions. Thus, 
we need to choose function spaces with higher regularity or integrability.

Let us define the spaces
\begin{align*}
    W^1_{1,2}(\hat{\Omega}) &\vcentcolon= \{ v \in L^1(\hat{\Omega}) \mid \partial_1 \partial_2 v
        \in L^1(\hat{\Omega}) \}
    \\  W^1_d(\hat{\Omega}) &\vcentcolon= \{ v \in L^1(\hat{\Omega}) \mid \partial_d v
        \in L^1(\hat{\Omega}) \}
\end{align*}
{\color{red} Why do we set the sequence to the spaces with $L^1$ sub instead of the intersection?}


This now gives us the discrete setting on the reference domain $\hat{\Omega}$ the idea 
is now to define the basis functions on the physical domain $\Omega$ by a 
push forward of the basis functions on the reference domain.
This is now the inverse of the analogous operations to 
\ref{} in 2D. 

We will stick to the single patch case meaning that there is a diffeomorphism 
$F: \hat{\Omega} \rightarrow \Omega$. Then we define 
the pullbacks 
\begin{align*}
    &\mathcal{P}_F^0: v \mapsto \hat{v} \vcentcolon= v \circ F
    \\ &\mathcal{P}_F^1: \mathbf{v} \mapsto \hat{\mathbf{v}} \vcentcolon= \det DF \,DF^{-1} \mathbf{v}
    \\ &\mathcal{P}_F^2: v \mapsto \hat{v} \vcentcolon= (\det DF) v
\end{align*}
which map functions on the physical domain $\Omega$ to functions on the reference domain 
$\hat{\Omega}$. Then we have the commuting properties
\begin{align*}
    \widehat{\veccurl} \mathcal{P}_F^0 v &= \mathcal{P}_F^1 \veccurl v
    \\ \widehat{\diver} \mathcal{P}_F^1 \mathbf{v} &= \mathcal{P}_F^1 \diver v
\end{align*}

Using the pullbacks we define the pushforwards as 
$\mathcal{F}^l \vcentcolon= (\mathcal{P}_F^l)^{-1}$ and then 
we get the basis functions on the physical domain
\begin{align*}
    \Lambda^l_{\mu} \vcentcolon= \mathcal{F}^l \hat{\Lambda}^l_{\mu}
\end{align*}
and then 
\begin{align*}
    V_h^l \vcentcolon= \text{span} \{ \Lambda^l_{\mu} \mid \mu \in \mathcal{M}^l \}
\end{align*}
are our discrete spaces. {\color{red} Approximation properties???}

Using the geometric degrees from \ref{} we can now construct the corresponding 
by 
\begin{align*}
    \sigma^l_\mu \vcentcolon= \hat{\sigma}_\mu^l \circ \mathcal{P}_F^l
\end{align*}
Then we have by construction that 
$\sigma^l_\mu(\Lambda^l_{\nu}) = \delta_{\mu,\nu}$. 

We can thus define the projection operators 
\begin{align*}
    \pi_h^l: U^l \rightarrow V_h^l, v \mapsto \sum_{\mu \in \mathcal{M}^l} \sigma^l_\mu \Lambda^l_\mu.
\end{align*}

Then these degrees of freedom commute with the corresponding differential operators 
as desired from \ref{}. They also correspond to geometric elements. 
$\sigma^0$ corresponds to point values in the physical domain, 
$\sigma^1$ to the fluxes through the image of edges and 
$\sigma^2$ to the integral over the mapped cells.

This very simple geometric interpretation gives us the ability to enforce 
the boundary condition directly by setting the corresponding degrees of freedom 
to zero. 

For $V_h^0 \subseteq H^1_0$ we want the trace on the boundary to vanish which 
means that we set the values at the boundary nodes to zero. Thus, 
for $\mathtt{n}_\mathbf{i}$ on the boundary we want 
$\sigma^0_{\mathbf{i}}(v) = 0$. 

For $V_h^1 \subseteq H_0(\diver)$ we want to have the normal trace zero. 
So when $\mathtt{e}_{d,\mathbf{i}}(\mathbf{v})$ is boundary edge we require
This is then achieved when $\sigma^1_{d,\mathbf{i}} (\mathbf{v}) = 0$.

We now define the spaces $\bar{V}_h^l$ which are the correspoding spaces 
without any boundary conditions. Then we define the 
projections $P_h^l: \bar{V}_h^l \rightarrow \bar{V}_h^l$ which set 
the boundary degrees of freedom to zero. They have a very simple 
matrix representation $\mathbb{P}_h^l$ 
$(\mathbb{P}_h^l)_{\mu, \nu} = 1 $ i.i.f. $\mu = \nu$ and $\mu$ does not 
correspond to a geometric element on the boundary. They are easily constructed 
by taking the identity matrix and setting the diagonal entries to zero that 
belong to boundary degrees of freedom.

On the lower level these basis functions are not used explicitely. Instead 
we use B-splines to compute the corresponding mass matrices etc. We will not 
go to deep into the details of implementation however. More details about the 
use of B-splines and the connection with the basis $\Lambda^l_\mu$ can be found 
in \cite[Sec.\,4.8]{multipatch_paper} 

\

\end{document}