\documentclass[../../master_thesis.tex]{subfiles}
\begin{document}
Wir betrachten das magnetostatische Problem auf äußeren Gebieten, 
welche das Komplement einer kompakten Menge sind. Im ersten Teil 
wird die Existenz und Eindeutigkeit des magnetostatischen Problems 
auf dem Komplement eines torusförmigen Gebietes bewiesen mit einer zusätzlichen
Bedingung in Form eines Kurvenintegrals unter passenden Annahmen. 
Im zweiten Teil liegt der Fokus auf der numerischen Annäherung von Lösungen 
des 2D magnetostatischen Problems auf einem Anulus-Gebiet mit einer Kurvenintegral-Bedingung.
Wir untersuchen die Idee, dieses Kurvenintegral mithilfe partieller Integration 
zu ersetzen und im Variationsproblem zu betrachten. Wir beweisen, dass das Problem wohl 
gestellt ist und eine A-Priori Abschätzung des Fehlers. Am Ende werden numerische Beispiele
präsentiert, welche die theoretischen Vorhersagen bestätigen.
\end{document}

% \begin{document}
% We investigate the magnetostatic problem on exterior domains 
% which are the complement of a compact set. In the first part, we prove 
% the existence and uniqueness of a solution of the magnetostatic problem posed on the
% complement of a toroidal domain with an additional curve integral constraint 
% under suitable assumptions.
% In the second part, we focus on the numerical approximation of solutions 
% of the 2D magnetostatic problem
% on an 
% annulus domain involving a curve integral constraint where we investigate 
% the idea to substitute this constraint using integration by parts and including it directly 
% in the variational formulation. We prove well-posedness and an a-priori estimate 
% of this formulation
% and present numerical examples in the end that confirm our theoretical 
% predictions.
% \end{document}